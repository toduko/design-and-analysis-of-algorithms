\documentclass{article}

\usepackage[utf8]{inputenc}
\usepackage[T2A]{fontenc}
\usepackage[english, bulgarian]{babel}
\usepackage{pgfplots}
\usepackage{amssymb}
\usepackage{hyperref, fancyhdr, lastpage, fancyvrb, tcolorbox, titlesec}
\usepackage{array, tabularx, colortbl}
\usepackage{tikz}
\usepackage{venndiagram}
\usepackage{amsthm, bm}
\usepackage{relsize}
\usepackage{amsmath,physics}
\usepackage{mathtools}
\usepackage{subcaption}
\usepackage{theoremref}
\usepackage{circuitikz}
\usepackage[a4paper, left=0.50in, right=0.50in, top=0.5in, bottom=1.0in]{geometry}
\usepackage{stmaryrd}
\usepackage[symbol]{footmisc}
\usepackage{minted}
\usepackage{enumitem}

\pgfplotsset{width=10cm,compat=1.9}

\newcommand{\N}{\mathbb{N}}
\newcommand{\R}{\mathbb{R}}
\newcommand{\F}{\mathcal{F}}

\ExplSyntaxOn
\NewDocumentCommand{\opair}{m}
{
  \langle\mspace{2mu}
  \clist_set:Nn \l_tmpa_clist { #1 }
  \clist_use:Nn \l_tmpa_clist {,\mspace{3mu plus 1mu minus 1mu}\allowbreak}
  \mspace{2mu}\rangle
}
\ExplSyntaxOff

\hypersetup{
  colorlinks=true,
  linktoc=all,
  linkcolor=blue
}

\theoremstyle{definition}
\newtheorem{definition}{Дефиниция}[section]
\newtheorem*{warning}{\textcolor{red}{Внимание}}
\theoremstyle{plain}
\newtheorem*{theorem}{Теорема}
\newtheorem*{invariant}{Инварианта}
\newtheorem{claim}[definition]{Твърдение}
\newtheorem{axiom}[definition]{Аксиома}
\newtheorem{lemma}[definition]{Лема}
\newtheorem{corollary}[definition]{Следствие}
\theoremstyle{remark}
\newtheorem*{remark}{Забележка}
\newtheorem{problem}{Задача}
\newtheorem*{solution}{Решение}
\theoremstyle{definition}

\setlength\parindent{0pt}

\title{Рекурентни уравнения}
\author{Тодор Дуков}
\date{}

\begin{document}
\maketitle

\section*{Защо са ни рекурентни уравнения?}

Те се появяват по естествен път, когато искаме да анализираме сложността на рекурсивни алгоритми.

Нека вземем за пример алгоритъма за двоично търсене:
\inputminted[linenos]{c++}{algorithms/binary_search.cpp}

При подаден масив $\mathtt{arr}$ с размер $\mathtt{n}$ и стойност $\mathtt{val}$, функцията $\mathtt{binary\_search(arr, 0, n - 1, val)}$ ще върне индекс на $\mathtt{arr}$, в който се намира $\mathtt{val}$, ако има такъв, иначе ще върне $\mathtt{-1}$.
Нека помислим каква е сложността на алгоритъма.
Управляващите параметри на рекурсията са $\mathtt{left}$ и $\mathtt{right}$.
Всеки път разликата между двете намалява двойно (като накрая когато $\mathtt{left = right}$ тя ще стане отрицателна).
Това означава, че в най-лошият случай сложността на алгоритъма може да се опише със следното рекурентно уравнение:
\begin{align*}
     & T(0) = 1                                          \\
     & T(n + 1) = T(\lfloor \frac{n + 1}{2} \rfloor) + 1
\end{align*}
Добавянето на $1$ е заради константните проверки, които се извършват преди рекурсивното извикване.
Технически не трябва да е $1$, защото проверките са няколко, но това няма да повлияе на асимптотичното поведение на $T(n)$.
В този случай лесно се вижда асимптотиката на $T(n)$:
\[
    T(n) = T(\lfloor \frac{n}{2} \rfloor) + 1 = T(\lfloor \frac{n}{4} \rfloor) + 1 + 1 = T(\lfloor \frac{n}{8} \rfloor) + 1 + 1 + 1 = \dots = T(0) + \underbrace{1 + \dots + 1}_{\text{около } \log(n) \text{ пъти}} \asymp \log(n)
\]

Така получаваме, че алгоритъмът има сложност $O(\log(n))$.
Обаче в общият случай далеч не е толкова лесно да се намери асимптотичното поведение на дадено рекурентно уравнение.
Целта ни ще бъде да развием по-богат апарат за асимптотичен анализ на рекурентните уравнения.

\section*{Начини за намиране на асимптотиката на рекурентни уравнения}

Начините се разделят на два типа:
\begin{enumerate}
    \item със решаване на уравнението
    \item без решаване на уравнението
\end{enumerate}

И двата начина са ценни.
Първият начин ни дава формула във явен вид, което може да ни е от полза.
Понякога обаче формулата във явен вид не е \textit{``красива''}, или изобщо не може да се намери такава.
Тогава идва на помощ вторият начин.
Той директно ни дава някаква \textit{``хубава''} формула, без да трябва да намираме в явен вид решение на рекурентното уравнение.
Проблема е обаче, че асимптотиката понякога е малко лъжлива -- алгоритъм със сложност $2^{2^{2^{1000}}}$ е асимптотично по-бавен от алгоритъм със сложност $n$, но практически вторият е по-бърз.

\pagebreak

Ще разгледаме следните методи (повечето от които са разглеждани по дискретна математика):
\begin{itemize}
    \item налучкване и доказване
    \item развиване (което преди малко показахме)
    \item методът с характеристичното уравнение
    \item Master теоремата
\end{itemize}

Нека разгледаме един пример с налучкване:
\begin{align*}
     & T(0) = 3                   \\
     & T(n + 1) = (n + 1)T(n) - n
\end{align*}

Започваме да разписваме:
\begin{center}
    \begin{tabular}{| c | c | c |}
        \hline
        $n$ & $T(n)$ & $n!$  \\
        \hline
        $0$ & $3$    & $1$   \\
        \hline
        $1$ & $3$    & $1$   \\
        \hline
        $2$ & $5$    & $2$   \\
        \hline
        $3$ & $13$   & $6$   \\
        \hline
        $4$ & $49$   & $24$  \\
        \hline
        $5$ & $241$  & $120$ \\
        \hline
        $6$ & $1441$ & $720$ \\
        \hline
    \end{tabular}
\end{center}

Вече лесно можем да покажем с индукция, че $T(n) = 2(n!) + 1$:
\begin{itemize}
    \item $T(0) = 3 = 2 \cdot 1 + 1 = 2 \cdot 0! + 1$
    \item $T(n + 1) = (n + 1)T(n) - n \stackrel{\text{(ИП)}}{=} (n + 1)(2(n!) + 1) - n = (n + 1)(2(n!)) + n + 1 - n = 2(n + 1)! + 1$
\end{itemize}
Накрая получаваме, че $T(n) \asymp n!$

Нека сега да видим как можем да използваме метода на характеристичното уравнение:
\[
    T(n) = 1 + \sum\limits_{i = 0}^{n - 1}T(i)    \text{ // функцията е добре дефинирана и за } 0
\]
Рекурентното уравнение, зададено в този вид, не може да се реши с този метод.
За това ще трябва да направим преобразувания:
\begin{align*}
     & T(0) = 0                                                                                                                                                                                                                        \\
     & T(n + 1) = 1 + \sum\limits_{i = 0}^{n}T(i) = 1 + T(n) + \underbrace{\sum\limits_{i = 0}^{n - 1}T(i)}_{T(n)} = 2T(n) + 1 = \underbrace{2T(n)}_{\text{хомогенна част}} + \underbrace{1 \cdot 1^{n + 1}}_{\text{нехомогенна част}}
\end{align*}

От хомогенната получаваме характеристичното уравнение $x - 2 = 0$ с единствен корен $2$, а от нехомогенната част получаваме единствен корен $1$.
Така:
\[
    T(n) = A \cdot 2^n + B \cdot 1^n \text{ за някои константи } A, B
\]
Вече няма нужда и да се намират константите -- ясно е че $T(n) \asymp 2^n$.
Kато използваме метода на характеристичното уравнение, не е нужно да намираме накрая константите за да разберем каква е асимптотиката.
Достатъчно е да вземем събираемото, която расте най-много (в случая $2^n$).

Нека сега разгледаме и последният начин:
\begin{theorem}[Master теоремата]
    Нека $a \geq 1, \: b > 1$ и $f \in \F$.
    Нека $T(n) = aT(\frac{n}{b}) + f(n)$ (мислете си, че $T(0) = 1$), където $\frac{n}{b}$ се интерпретира като $\lfloor \frac{n}{b} \rfloor$ или $\lceil \frac{n}{b} \rceil$.
    Тогава:
    \begin{itemize}
        \item[1 сл.] Ако $f(n) \preceq n^{\log_b(a) - \varepsilon}$ за някое $\varepsilon > 0$, то тогава $T(n) \asymp n^{log_b(a)}$
        \item[2 сл.] Ако $f(n) \asymp n^{log_b(a)}$, то тогава $T(n) \asymp n^{log_b(a)} \log(n)$
        \item[3 сл.] Ако са изпълнени следните условия:
            \begin{enumerate}
                \item $f(n) \succeq n^{\log_b(a) + \varepsilon}$ за някое $\varepsilon > 0$
                \item съществува $0 < c < 1$, за което от някъде нататък $a \cdot f(\frac{n}{b}) \leq c \cdot f(n)$,
            \end{enumerate}
            то тогава $T(n) \asymp f(n)$
    \end{itemize}
\end{theorem}

\pagebreak

Нека разгледаме рекурентното уравнение:
\[
    T(n) = 2T(\frac{n}{2}) + 1
\]
Тук $a = b = 2$, и $f(n) = 1$.
Също така $\log_b(a) = 1$, откъдето $f(n) = 1 \preceq n^{\log_b(a) - \varepsilon}$, за $\varepsilon \in (0, 1)$.
Така по 2 сл. на Master теоремата получаваме, че $T(n) \asymp n$.

\section*{Задачи}

\begin{problem}
Без да се използва Master теоремата да се намери асимптотиката на сложността по време на алгоритъма Merge sort т.е. на рекурентното уравнение:
\[
    T(n) = 2T(\frac{n}{2}) + n
\]
\end{problem}

\begin{problem}
Да се намери асимптотиката на следните рекурентни уравнения:
\begin{align*}
    T_1(n) & = 29T_1(\frac{n}{3}) + 2 \sum\limits_{i = 1}^n \frac{1}{i^2} & T_2(n) & = 29T_2(\frac{n}{3}) + 12n + \sqrt{n}              & T_3(n) & = T_3(n - 1) + \frac{n}{(n + 1)(n - 1)}            \\
    T_4(n) & = 29T_4(\frac{n}{3}) + (\sum\limits_{i = 1}^n \frac{1}{i})^4 & T_5(n) & = 29T_5(\frac{n}{3}) + 2 \sum\limits_{i = 1}^n i^2 & T_6(n) & = 29T_6(\frac{n}{3}) + n^{\sqrt{n}} + (\sqrt{n})^n \\
    T_7(n) & = T_7(\sqrt{n}) + n                                          & T_8(n) & = 29T_8(\frac{n}{3}) + \binom{2n}{2}               & T_9(n) & = 8T_9(n - 1) - T_9(n - 2) + 2n2^{2n} + 3n2^{3n}
\end{align*}
\end{problem}

\begin{problem}
Да се намери сложността по-време на следният алгоритъм:
\inputminted[linenos]{c++}{algorithms/alg.cpp}
\end{problem}

\end{document}
