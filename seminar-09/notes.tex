\documentclass{article}

\usepackage[utf8]{inputenc}
\usepackage[T2A]{fontenc}
\usepackage[english, bulgarian]{babel}
\usepackage{pgfplots}
\usepackage{amssymb}
\usepackage{hyperref, fancyhdr, lastpage, fancyvrb, tcolorbox, titlesec}
\usepackage{array, tabularx, colortbl}
\usepackage{tikz}
\usepackage{venndiagram}
\usepackage{amsthm, bm}
\usepackage{relsize}
\usepackage{amsmath,physics}
\usepackage{mathtools}
\usepackage{subcaption}
\usepackage{theoremref}
\usepackage{circuitikz}
\usepackage[a4paper, left=0.50in, right=0.50in, top=0.5in, bottom=1.0in]{geometry}
\usepackage{stmaryrd}
\usepackage[symbol]{footmisc}
\usepackage{minted}
\usepackage{enumitem}
\usepackage{systeme}

\pgfplotsset{width=10cm,compat=1.9}

\newcommand{\N}{\mathbb{N}}
\newcommand{\R}{\mathbb{R}}
\newcommand{\F}{\mathcal{F}}

\ExplSyntaxOn
\NewDocumentCommand{\opair}{m}
{
  \langle\mspace{2mu}
  \clist_set:Nn \l_tmpa_clist { #1 }
  \clist_use:Nn \l_tmpa_clist {,\mspace{3mu plus 1mu minus 1mu}\allowbreak}
  \mspace{2mu}\rangle
}
\ExplSyntaxOff

\hypersetup{
  colorlinks=true,
  linktoc=all,
  linkcolor=blue
}

\theoremstyle{definition}
\newtheorem{definition}{Дефиниция}[section]
\newtheorem*{warning}{\textcolor{red}{Внимание}}
\theoremstyle{plain}
\newtheorem*{theorem}{Теорема}
\newtheorem*{invariant}{Инвариант}
\newtheorem{claim}[definition]{Твърдение}
\newtheorem{axiom}[definition]{Аксиома}
\newtheorem{lemma}[definition]{Лема}
\newtheorem{corollary}[definition]{Следствие}
\theoremstyle{remark}
\newtheorem*{remark}{Забележка}
\newtheorem{problem}{Задача}
\newtheorem*{solution}{Решение}
\theoremstyle{definition}

\setlength\parindent{0pt}

\title{Прости алгоритми върху графи}
\author{Тодор Дуков}
\date{}

\begin{document}
\maketitle

\section*{Защо изобщо се занимаваме с графи?}

Графите са може би най-приложимата структура в областта на компютърните науки.
Тяхната моделираща мощ е несравнима с тази на останалите структури.
Те могат се използват за моделиране на:
\begin{itemize}
  \item приятелски връзки в социални мрежи
  \item пътни мрежи в навигационни системи
  \item йерархични системи
  \item биологични мрежи
\end{itemize}
Някои от задачите, които могат да решават са:
\begin{itemize}
  \item намиране на най-къс път от точка $A$ до точка $B$
  \item намиране на съвместима наредба на дадени задачи
  \item намиране на най-добро разписание на полети
  \item класифициране на уебсайтове по популярност
  \item маркетинг в социални мрежи
  \item валидация на текст
\end{itemize}

\section*{Как представяме графите в паметта?}

В зависимост от нашите цели графите могат да бъдат представени в паметта по различни начини.
Най-използваните начини са:
\begin{itemize}
  \item списък на съседство -- за всеки връх се пазят в списък съседите му (и теглата ако има такива).
  \item матрица на съседство -- пази се булева (може и числова ако графът е тегловен) таблица със всевъзможните комбинации от двойки върхове.
        Ако между два върха има ребро, то в съответната клетка пише единица (или теглото на реброто при тегловен граф), иначе нула.
  \item списък с ребрата -- множеството от ребра идва като списък. Обикновено ако графът е неориентиран се пази само една пермутация на реброто.
\end{itemize}

Матрицата на съседство се използва по-рядко.
Този подход е добър, когато графите са гъсти т.е. има много ребра в тях.
В противен случай ние заемаме много повече памет от колкото ни е нужна.
За сметка на това можем да проверим дали между два върха има ребро за константно време.

Списъците на съседство са по-пестеливи от към памет в средния случай, обаче за сметка на това по-бавно се проверява съседство между два върха.
Този подход е добър, когато графите са редки т.е. имат малко ребра в тях.
Също така ако по някаква причина ни трябва да изброяваме точно съседите на някакъв връх (да кажем за някакво обхождане), това очевидно е най-добрият начин.
В най-лошия случай заемаме двойно повече памет от подхода с матрицата.

Списъка с ребрата е най-пестеливия начин от тези три.
Пази се минималното количество нужна информация.
Проблемът тук е, че проверката за съседство и изброяването на съседи на даден връх са бавни.
Обаче това представяне все пак се използва, например когато искаме да построим МПД.

Накратко, сложностите са такива:
\begin{center}
  \begin{tabular}{|c|c|c|c|}
    \hline
    подход               & памет           & проверка за съседство & изброяване на съседите $N(v)$ на връх $v$ \\
    \hline
    списък на съседство  & $O(|V| + |E|)$  & $O(|V|)$              & $\Theta(|N(v)|)$                          \\
    \hline
    матрица на съседство & $\Theta(|V|^2)$ & $\Theta(1)$           & $O(|V|)$                                  \\
    \hline
    списък с ребра       & $\Theta(|E|)$   & $O(|E|)$              & $O(|E|)$                                  \\
    \hline
  \end{tabular}
\end{center}

\section*{Код на алгоритмите за обхождане на графи}

\inputminted[linenos]{c++}{algorithms/traversal.cpp}

\section*{Задачи}

\begin{problem}
Да се напише алгоритъм, който при подаден граф намира броя на свързаните компоненти.
\end{problem}

\begin{problem}
Да се напише алгоритъм, който при подаден граф проверява дали той е цикличен.
\end{problem}

\begin{problem}
Да се напише алгоритъм, който при подаден граф проверява дали той е дърво.
\end{problem}

\begin{problem}
Да се напише алгоритъм, който при подаден граф (може и ориентиран) и два негови върха намира най-късия път между тях.
\end{problem}

\begin{problem}
Да се напише алгоритъм, който при подаден граф (може и ориентиран), два негови върха и дължина намира броят на пътища между тях със съответната дължина.
\end{problem}

\begin{problem}
Да се напише алгоритъм, който при подаден граф проверява дали той е двуделен.
\end{problem}

\begin{problem}
Трябва да изпълним задачи $1, \dots, n$.
Обаче имаме допълнителни изисквания $R[1 \dots k]$ от вида $\opair{i, j}$, които казват \textit{``задача $i$ трябва да се изпълни преди задача $j$''}.
Да се напише алгоритъм, който при подадени изисквания намира последователност от задачи, която удовлетворява тези изисквания.
Ако няма такива, да се върне съобщение за грешка.
\end{problem}

\begin{problem}
В някакъв град има $n$ души с етикети от $0$ до $n - 1$.
Има слухове, че един от тези хора тайно е съдията на града.
Ако такъв човек има, то тогава:
\begin{itemize}
  \item съдията не вярва на никого
  \item всеки вярва на съдията, освен самия него
\end{itemize}
Масив на вярата за този град ще наричаме всеки масив $T[1 \dots k]$ от двойки $\opair{i, j}$ (за $0 \leq i, j < n$), които казват \textit{``човекът с етикет $i$ вярва на човека с етикет $j$''}.
Да се напише алгоритъм, който при подаден масив на вярата, връща етикета на съдията.
Ако няма съдия, да се върне $-1$.
\end{problem}

\end{document}
