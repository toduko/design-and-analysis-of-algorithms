\documentclass{article}

\usepackage[utf8]{inputenc}
\usepackage[T2A]{fontenc}
\usepackage[english, bulgarian]{babel}
\usepackage{pgfplots}
\usepackage{amssymb}
\usepackage{hyperref, fancyhdr, lastpage, fancyvrb, tcolorbox, titlesec}
\usepackage{array, tabularx, colortbl}
\usepackage{tikz}
\usepackage{venndiagram}
\usepackage{amsthm, bm}
\usepackage{relsize}
\usepackage{amsmath,physics}
\usepackage{mathtools}
\usepackage{subcaption}
\usepackage{theoremref}
\usepackage{circuitikz}
\usepackage[a4paper, left=0.50in, right=0.50in, top=0.5in, bottom=1.0in]{geometry}
\usepackage{stmaryrd}
\usepackage[symbol]{footmisc}
\usepackage{minted}
\usepackage{enumitem}
\usepackage{systeme}
\usepackage{forest}
\useforestlibrary{linguistics}

\pgfplotsset{width=10cm,compat=1.9}

\newcommand{\N}{\mathbb{N}}
\newcommand{\R}{\mathbb{R}}
\newcommand{\F}{\mathcal{F}}

\ExplSyntaxOn
\NewDocumentCommand{\opair}{m}
{
  \langle\mspace{2mu}
  \clist_set:Nn \l_tmpa_clist { #1 }
  \clist_use:Nn \l_tmpa_clist {,\mspace{3mu plus 1mu minus 1mu}\allowbreak}
  \mspace{2mu}\rangle
}
\ExplSyntaxOff

\hypersetup{
  colorlinks=true,
  linktoc=all,
  linkcolor=blue
}

\theoremstyle{definition}
\newtheorem{definition}{Дефиниция}[section]
\newtheorem*{warning}{\textcolor{red}{Внимание}}
\theoremstyle{plain}
\newtheorem*{theorem}{Теорема}
\newtheorem*{invariant}{Инвариант}
\newtheorem{claim}[definition]{Твърдение}
\newtheorem{axiom}[definition]{Аксиома}
\newtheorem{lemma}[definition]{Лема}
\newtheorem{corollary}[definition]{Следствие}
\theoremstyle{remark}
\newtheorem*{remark}{Забележка}
\newtheorem{problem}{Задача}
\newtheorem*{solution}{Решение}
\theoremstyle{definition}

\setlength\parindent{0pt}

\title{Още задачи върху графи}
\author{Тодор Дуков}
\date{}

\begin{document}
\maketitle

\section*{Най-къси пътища в тегловен граф}

Започваме с може би най-известния нетривиален графов алгоритъм -- алгоритъмът на Дийкстра за намиране на най-къси пътища в тегловен граф.
При него започваме с един стартов връх, и намираме най-късите пътища между този стартов връх и всеки достижим от него.
Ето как става това:
\inputminted[linenos]{c++}{algorithms/dijkstra.cpp}
Алгоритъмът има сложност по време $\Theta((|V| + |E|)\log(|V|))$ в най-лошия случай.

\section*{Структурата Union-find/Disjoint-union}

Ще разгледаме метод за поддържане на разбиване на множества от вида $\{ 0, \dots, n \}$.
Искаме да започнем от разбиването $\{ \{ 0 \}, \dots, \{ n \} \}$, и след това бързо да можем да обединяваме множества и да проверяваме дали някои $i, j$ попадат на едно и също място в разбиването.
За да може тези проверки и сливания да стават бързо, се използва структурата Union-find (понякога се нарича Disjoint-union).
Ще имаме единствена функция $\operatorname{unify}(i, j)$, която приема $i, j \in \{ 0, \dots, n \}$ и слива множествата от разбиването, в които $i$ и $j$ участват.
Ако преди това те са били в едно и също множество, то тогава накрая връщаме $\mathtt{false}$, иначе връщаме $\mathtt{true}$.

\pagebreak

Имплементацията е следната:
\inputminted[linenos]{c++}{structures/union_find.hpp}
Сложността на извикването на $\operatorname{unify}$ е $O(\alpha(n))$, където $\alpha$ е обратната функция на Акерман.
Тази функция расте изключително бавно -- $\alpha(n) \leq 4$ за $n < 10^{600}$.

\section*{Алгоритъмът на Крускал за намиране на минимално покриващо дърво}
Графът ще пазим като масив от ребра:
\inputminted[linenos]{c++}{structures/edge.hpp}

Алгоритъмът на Крускал работи по доста естествен начин.
Гледаме да приоритизираме ребрата с най-малки тегла.
Не да ги добавяме само ако биха образували цикъл.

Имплементацията е следната:
\inputminted[linenos]{c++}{algorithms/kruskal.cpp}
Сложността на алгоритъма по време е $\Theta(|E|\log(|E|))$ в най-лошия случай.

\section*{Задачи}

\begin{problem}
Да се напише алгоритъм, който при подаден ориентиран тегловен граф намира теглата на минималните пътища между всеки два върха.
\end{problem}

\begin{problem}
Да се напише алгоритъм, който при подаден тегловен ориентиран ацикличен граф и негов начален връх, намира най-късите пътище от този начален връх до всички достижими от него.
\end{problem}

\begin{problem}
Да се напише алгоритъм, който при подаден свързан цикличен граф с ребро, чието премахване ще превърне графа в дърво, връща това ребро.
\end{problem}

\begin{problem}
Нека е дадено крайно множество от променливи $X = \{ x_0, \dots x_k \}$.
Уравнения над $X$ ще наричаме всички низове от вида $x_i = x_j$ и $x_i \neq x_j$ за $0 \leq i, j \leq k$.
Да се напише алгоритъм, който при подадено множество от променливи $X$ и списък $E[1 \dots n]$ от уравнения над $X$, да проверява дали системата, съставена от списъка с уравнения има решение.
\end{problem}

\begin{problem}
Алис и Боб имат граф с върхове измежду $0$ и $n$ и три типа ребра:
\begin{itemize}
  \item ребрата от тип $1$ могат да бъдат траверсирани само от Алис
  \item ребрата от тип $2$ могат да бъдат траверсирани само от Боб
  \item ребрата от тип $3$ могат да бъдат траверсирани и от двамата
\end{itemize}
Да се напише алгоритъм, който при подаден масив от ребра $E[1 \dots k]$ т.е. тройки от типа $\opair{type, i, j}$ за някои $type \in \{1, 2, 3\}$ и $i, j \in \{ 0, \dots, n \}$, връща максималния брой ребра, които могат да се махнат, и графът пак да може да бъде обходен от Алис и Боб.
Ако някой от двамата не може да обходи първоначалния граф, да се върне $-1$.
\end{problem}

\begin{problem}
За всеки две точки $(x_1, y_1)$ и $(x_2, y_2)$ в $\mathbb{Z} \cross \mathbb{Z}$, цената на свързване на $X$ и $Y$ ще бъде $|x_1 - x_2| + |y_1 - y_2|$.
Да се напише алгоритъм, който при подаден масив $P[1 \dots n]$ от точки, намира минималната обща цена на свързване, за която между всеки две точки от $P[1 \dots n]$ ще има път.
\end{problem}

\end{document}
