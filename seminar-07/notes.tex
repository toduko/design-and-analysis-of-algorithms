\documentclass{article}

\usepackage[utf8]{inputenc}
\usepackage[T2A]{fontenc}
\usepackage[english, bulgarian]{babel}
\usepackage{pgfplots}
\usepackage{amssymb}
\usepackage{hyperref, fancyhdr, lastpage, fancyvrb, tcolorbox, titlesec}
\usepackage{array, tabularx, colortbl}
\usepackage{tikz}
\usepackage{venndiagram}
\usepackage{amsthm, bm}
\usepackage{relsize}
\usepackage{amsmath,physics}
\usepackage{mathtools}
\usepackage{subcaption}
\usepackage{theoremref}
\usepackage{circuitikz}
\usepackage[a4paper, left=0.50in, right=0.50in, top=0.5in, bottom=1.0in]{geometry}
\usepackage{stmaryrd}
\usepackage[symbol]{footmisc}
\usepackage{minted}
\usepackage{enumitem}

\pgfplotsset{width=10cm,compat=1.9}

\newcommand{\N}{\mathbb{N}}
\newcommand{\R}{\mathbb{R}}
\newcommand{\F}{\mathcal{F}}

\ExplSyntaxOn
\NewDocumentCommand{\opair}{m}
{
  \langle\mspace{2mu}
  \clist_set:Nn \l_tmpa_clist { #1 }
  \clist_use:Nn \l_tmpa_clist {,\mspace{3mu plus 1mu minus 1mu}\allowbreak}
  \mspace{2mu}\rangle
}
\ExplSyntaxOff

\hypersetup{
  colorlinks=true,
  linktoc=all,
  linkcolor=blue
}

\theoremstyle{definition}
\newtheorem{definition}{Дефиниция}[section]
\newtheorem*{warning}{\textcolor{red}{Внимание}}
\theoremstyle{plain}
\newtheorem*{theorem}{Теорема}
\newtheorem*{invariant}{Инвариант}
\newtheorem{claim}[definition]{Твърдение}
\newtheorem{axiom}[definition]{Аксиома}
\newtheorem{lemma}[definition]{Лема}
\newtheorem{corollary}[definition]{Следствие}
\theoremstyle{remark}
\newtheorem*{remark}{Забележка}
\newtheorem{problem}{Задача}
\newtheorem*{solution}{Решение}
\theoremstyle{definition}

\setlength\parindent{0pt}

\title{Приложения на сортиращите алгоритми}
\author{Тодор Дуков}
\date{}

\begin{document}
\maketitle

\section*{Обща информация за някои алгоритми за сортиране}

Понякога се оказва много удобно да сортираме входните данни, защото това ни носи полезна информация.
Затова е хубаво да се знаят различните алгоритми за сортиране и в какво те са добри.
Нека ги сравним по тяхната сложност, като:
\begin{itemize}
    \item $T_{avg}(n)$ е сложността по време в средния случай
    \item $M_{avg}(n)$ е сложността по памет в средния случай
    \item $T_{worst}(n)$ е сложността по време в най-лошия случай
    \item $M_{worst}(n)$ е сложността по памет в най-лошия случай
\end{itemize}

\begin{center}
    \begin{tabular}{| c | c | c | c | c |}
        \hline
        име                    & $T_{avg}(n)$        & $M_{avg}(n)$      & $T_{worst}(n)$      & $M_{worst}(n)$ \\
        \hline
        пирамидално сортиране  & $\Theta(n \log(n))$ & $\Theta(1)$       & $\Theta(n \log(n))$ & $\Theta(1)$    \\
        \hline
        сортиране чрез сливане & $\Theta(n \log(n))$ & $\Theta(n)$       & $\Theta(n \log(n))$ & $\Theta(n)$    \\
        \hline
        бързо сортиране        & $\Theta(n \log(n))$ & $\Theta(\log(n))$ & $\Theta(n^2)$       & $\Theta(n)$    \\
        \hline
    \end{tabular}
\end{center}

Въпреки че алгоритъмът за бързо сортиране е по-бавен в най-лошия случай от пирамидалното сортиране и сортирането чрез сливане, практически се оказва, че се справя по-добре.
Разбира се, другите сортировки също имат предимства.
Пирамидалното сортиране заема малко памет, което е много полезно във вградени системи.
Сортирането чрез сливане се оказва по-бързо, когато данните са много големи и се съхраняват във външни устройства (да кажем, на твърд диск).
Освен тези алгоритми има и други хубави алгоритми като сортиране чрез броене, но за съжаление ние няма да ги разглеждаме тук.

\section*{Два алгоритъма}

Ще започнем с два често срещани алгоритъма, които се възползват от това, че данните идват сортирани:
\begin{itemize}
    \item двоично търсене
    \item алгоритъма за задачата $2$-sum
\end{itemize}

Нека започнем с алгоритъма за двоично търсене:
\inputminted[linenos]{c++}{algorithms/binary_search.cpp}

При подаден сортиран масив от числа $\mathtt{arr}$ с размер $\mathtt{n}$ и стойност $\mathtt{val}$, функцията $\mathtt{binary\_search(arr, n, val)}$ ще върне индекс на $\mathtt{arr}$, в който се намира $\mathtt{val}$, ако има такъв, иначе ще върне $\mathtt{-1}$:
\begin{invariant}
    При всяко достигане на проверката за край на цикъла на ред $5$ имаме, че стойността $\mathtt{val}$ не се намира измежду двата масива $\mathtt{arr[0 \dots left - 1]}$ и $\mathtt{arr[right + 1 \dots n - 1]}$.
\end{invariant}

\textbf{База.}
При първото достигане имаме, че $\mathtt{left = 0}$ и $\mathtt{right = n - 1}$.
Наистина стойността $\mathtt{val}$ не се намира измежду двата масива $\mathtt{arr[0 \dots 0 -1]}$ и $\mathtt{arr[n - 1 + 1 \dots n - 1]}$.

\textbf{Поддръжка.}
Нека твърдението е изпълнено за някое непоследно достигане на проверката за край на цикъла.
Тогава $\mathtt{val}$ не се намира измежду $\mathtt{arr[0 \dots left - 1]}$ и $\mathtt{arr[right + 1 \dots n - 1]}$, и понеже достигането е непоследно, $\mathtt{left \leq right}$.
Тогава $\mathtt{mid = \Bigl\lfloor \frac{left + right}{2} \Bigr\rfloor}$, откъдето $\mathtt{left \leq mid \leq right}$.
Трябва да разгледаме следните три случая:
\begin{itemize}
    \item[1 сл.] $\mathtt{arr[mid] = val}$ -- това няма как да е изпълнено понеже достигането е нефинално
    \item[2 сл.] $\mathtt{arr[mid] < val}$ -- понеже $\mathtt{arr}$ е сортиран, няма как $\mathtt{val}$ да се намира измежду $\mathtt{arr[0 \dots mid]}$, откъдето $\mathtt{val}$ не се намира измежду $\mathtt{arr[0 \dots \underbrace{\mathtt{mid + 1}}_{left_{new}} - 1]}$ и $\mathtt{arr[right + 1 \dots n - 1]}$.
    \item[3 сл.] $\mathtt{arr[mid] > val}$ -- напълно дуален на 2 сл.
\end{itemize}

\textbf{Терминация.}
От цикъла винаги ще излезем, защото или ще открием $\mathtt{val}$, или $\mathtt{right - left}$ ще намалява, докато не стане отрицателна.
Излизането става по два начина:
\begin{itemize}
    \item не е изпълнено условието на ред $5$ т.е. $\mathtt{left > right}$ -- тогава $\mathtt{left - 1 \geq right}$ и понеже $\mathtt{val}$ не се намира измежду $\mathtt{arr[0 \dots left - 1]}$ и $\mathtt{arr[right + 1 \dots n - 1]}$, $\mathtt{val}$ не се намира във $\mathtt{arr[0 \dots n - 1]}$.
          Накрая алгоритъмът ще върне $\mathtt{-1}$, което наистина е желания резултат.
    \item изпълнено е условието на ред $9$ т.е. $\mathtt{arr[mid] = val}$ -- тогава алгоритъмът коректно връща $\mathtt{mid}$.
\end{itemize}

Нека сега разгледаме алгоритъма за задачата $2$-sum:
\inputminted[linenos]{c++}{algorithms/two_sum.cpp}

Ще покажем, че при подаден сортиран масив $\mathtt{arr}$ с размер $\mathtt{n}$ и число $\mathtt{target}$, функцията $\mathtt{two\_sum(arr, n, target)}$ разпознава дали има $\mathtt{0 \leq i < j \leq n - 1}$, за които:
\[
    \mathtt{arr[i] + arr[j] = target} \text{ // всяка такава двойка } \mathtt{(arr[i], arr[j])} \text{ ще наричаме \textit{диада}}
\]
\begin{invariant}
    При всяко достигане на проверката за край на цикъла на ред $5$, всички диади са в $\mathtt{arr[left \dots right]}$.
\end{invariant}

\textbf{База.}
При първото достигане имаме, че $\mathtt{left = 0}$ и $\mathtt{right = n - 1}$.
Тогава твърдението е тривиално изпълнено.

\textbf{Поддръжка.}
Нека твърдението е изпълнено за някое непоследно достигане на проверката за край на цикъла.
Тогава всички диади са в $\mathtt{arr[left \dots right]}$ и $\mathtt{left < right}$.
Разглеждаме три случая:
\begin{itemize}
    \item[1 сл.] $\mathtt{arr[left] + arr[right] = target}$ -- това няма как да е изпълнено понеже достигането е нефинално
    \item[2 сл.] $\mathtt{arr[left] + arr[right] < target}$ -- тогава понеже $\mathtt{arr}$ е сортиран, за всяко $\mathtt{left + 1 \leq i \leq right}$ имаме, че $\mathtt{arr[left] + arr[i] \leq arr[left] + arr[right] < target}$.
          Това означава, че $\mathtt{left}$ не участва в никоя диада във $\mathtt{arr[left \dots right]}$.
          Така получаваме, че всички диади се намират в $\mathtt{arr[\underbrace{\mathtt{left + 1}}_{left_{new}} \dots right]}$
    \item[3 сл.] $\mathtt{arr[left] + arr[right] > target}$ -- напълно дуален на 2 сл.
\end{itemize}

\textbf{Терминация.}
От цикъла винаги ще излезем, защото или ще открием $\mathtt{val}$, или $\mathtt{right - left}$ ще намалява, докато не стане $\mathtt{0}$.
Излизането става по два начина:
\begin{itemize}
    \item не е изпълнено условието на ред $5$ т.е. $\mathtt{left \geq right}$ -- тогава няма диади в $\mathtt{arr}$, и алгоритъмът коректно ще върне $\mathtt{false}$.
    \item изпълнено е условието на ред $9$ т.е. $\mathtt{arr[left] + arr[right] = target}$ -- тогава алгоритъмът връща $\mathtt{true}$ т.е. точно това, което искаме.
\end{itemize}

Двата алгоритъма са доста подобни, използват една често срещана техника за търсене в дадено множество от елементи.
Търсенето започва с цялото множество и то постепенно се смалява.
Разбира се, тук разгледаните алгоритми имат разлика в сложността, поради разликата в стесняването:
\begin{itemize}
    \item първият алгоритъм има сложност $O(\log(n))$ понеже винаги разликата между $\mathtt{left}$ и $\mathtt{right}$ намалява двойно
    \item вторият алгоритъм има сложност $O(n)$ понеже винаги разликата между $\mathtt{left}$ и $\mathtt{right}$ намалява с единица
\end{itemize}

\section*{Задачи}

\begin{problem}
Да се напише колкото се може по-бърз алгоритъм, който приема масив от различни цели числа $A[1 \dots n]$ с $n \geq 3$, за който има $1 < i < n$ такова, че $A[1 \dots i]$ е сортиран възходящо и $A[i \dots n]$ е сортиран низходящо, и връща това $i$.
След това да се докаже неговата коректност, и да се изследва сложността му по време.
\end{problem}

\begin{problem}
Да се напише колкото се може по-бърз алгоритъм, който при подаден сортиран целочислен масив $A[1 \dots n]$ и число $t$, което се намира в масива, връща най-малкият и най-големият индекс, на които $t$ се намира в $A[1 \dots n]$.
След това да се докаже неговата коректност, и да се изследва сложността му по време.
\end{problem}

\begin{problem}
Да се напише колкото се може по-бърз алгоритъм, който при подаден сортиран целочислен масив $A[1 \dots n]$ и числа $k \geq 2$ и $t$, връща дали има $0 \leq i_1 < i_2 < \dots < i_k \leq n - 1$, за които:
\[
    A[i_1] + A[i_2] + \dots + A[i_k] = t
\]
След това да се докаже неговата коректност, и да се изследва сложността му по време.
\end{problem}

\begin{problem}
За $a, b, x \in \mathbb{Z}$ казваме, че $a$ е по-близо от $b$ до $x$, ако:
\[
    |a - x| < |b - x| \lor (|a - x| = |b - x| \: \& \: a < b)
\]
Да се напише колкото се може по-бърз алгоритъм, който приема целочислен масив $A[1 \dots n]$, и връща най-близките $k$ на брой числа до $t$ в $A[1 \dots n]$.
След това да се докаже неговата коректност, и да се изследва сложността му по време.
Може ли да се напише по-бърз алгоритъм при предположение че $A[1 \dots n]$ е сортиран?
\end{problem}

\begin{problem}
Да се напише колкото се може по-бърз алгоритъм, който приема масив $A[1 \dots n]$, съставен от числата $0, 1$ и $2$, и го сортира.
След това да се докаже неговата коректност, и да се изследва сложността му по време.
\end{problem}

\begin{problem}
Да се напише колкото се може по-бърз алгоритъм, който приема масив $I[1 \dots n]$, съставен от двойки числа, които представят някакъв затворен интервал от цели числа ($[a, b]$ за някои $a, b \in \mathbb{Z}$), и връща нов масив, в който са сляти всички интервали с непразно сечение.
След това да се докаже неговата коректност, и да се изследва сложността му по време.
\end{problem}

\begin{problem}
Да се напише колкото се може по-бърз алгоритъм, който приема целочислен масив $A[1 \dots n]$, и връща нов масив от квадратите на $A[1 \dots n]$, който е сортиран.
След това да се докаже неговата коректност, и да се изследва сложността му по време.
\end{problem}

\begin{problem}
Да се напише колкото се може по-бърз алгоритъм, който приема целочислен масив $A[1 \dots 2n]$, и връща:
\[
    \min\{ \max \{ A'[2i] + A'[2i - 1] \mid 1 \leq i \leq n \} \mid A'[1 \dots n] \text{ е пермутация на } A[1 \dots n] \}
\]
След това да се докаже неговата коректност, и да се изследва сложността му по време.
\end{problem}

\begin{problem}
    Да се напише колкото се може по-бърз алгоритъм, който приема два целочислени масива $A[1 \dots n]$ и $B[1 \dots n]$, и връща:
    \[
        \min \{ \sum\limits_{i = 1}^n |A'[i] - B'[i]| \mid A'[1 \dots n] \text{ е пермутация на } A[1 \dots n] \text{ и } B'[1 \dots n] \text{ е пермутация на } B[1 \dots n]\}
    \]
    След това да се докаже неговата коректност, и да се изследва сложността му по време.
    \end{problem}

\end{document}
