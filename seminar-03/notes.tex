\documentclass{article}

\usepackage[utf8]{inputenc}
\usepackage[T2A]{fontenc}
\usepackage[english, bulgarian]{babel}
\usepackage{pgfplots}
\usepackage{amssymb}
\usepackage{hyperref, fancyhdr, lastpage, fancyvrb, tcolorbox, titlesec}
\usepackage{array, tabularx, colortbl}
\usepackage{tikz}
\usepackage{venndiagram}
\usepackage{amsthm, bm}
\usepackage{relsize}
\usepackage{amsmath,physics}
\usepackage{mathtools}
\usepackage{subcaption}
\usepackage{theoremref}
\usepackage{circuitikz}
\usepackage[a4paper, left=0.50in, right=0.50in, top=0.5in, bottom=1.0in]{geometry}
\usepackage{stmaryrd}
\usepackage[symbol]{footmisc}
\usepackage{minted}
\usepackage{enumitem}

\pgfplotsset{width=10cm,compat=1.9}

\newcommand{\N}{\mathbb{N}}
\newcommand{\R}{\mathbb{R}}
\newcommand{\F}{\mathcal{F}}

\ExplSyntaxOn
\NewDocumentCommand{\opair}{m}
{
  \langle\mspace{2mu}
  \clist_set:Nn \l_tmpa_clist { #1 }
  \clist_use:Nn \l_tmpa_clist {,\mspace{3mu plus 1mu minus 1mu}\allowbreak}
  \mspace{2mu}\rangle
}
\ExplSyntaxOff

\hypersetup{
  colorlinks=true,
  linktoc=all,
  linkcolor=blue
}

\theoremstyle{definition}
\newtheorem{definition}{Дефиниция}[section]
\newtheorem*{warning}{\textcolor{red}{Внимание}}
\theoremstyle{plain}
\newtheorem{theorem}[definition]{Теорема}
\newtheorem*{invariant}{Инварианта}
\newtheorem{claim}[definition]{Твърдение}
\newtheorem{axiom}[definition]{Аксиома}
\newtheorem{lemma}[definition]{Лема}
\newtheorem{corollary}[definition]{Следствие}
\theoremstyle{remark}
\newtheorem*{remark}{Забележка}
\newtheorem{problem}{Задача}
\newtheorem{solution}{Решение}
\theoremstyle{definition}

\setlength\parindent{0pt}

\title{Коректност на итеративни алгоритми}
\author{Тодор Дуков}
\date{}

\begin{document}
\maketitle

\section*{Какво имаме предвид под коректност?}

За целите на този курс един алгоритъм ще наричаме \textbf{коректен}, ако:
\begin{enumerate}
    \item завършва при всеки вход
    \item връща правилен изход при всеки вход
\end{enumerate}

\begin{remark}
    Въпреки че ние ще имаме това разбиране в курса, в реалния свят нещата не винаги се случват така:
    \begin{itemize}
        \item разглеждат се алгоритми, които могат и да не завършват за някои входове -- от теоритична гледна точка са интересни за хората, които се занимават с теория на изчислимостта
        \item разглеждат се алгоритми, които много често (но не винаги) връщат правилния вход -- обикновено това се прави с цел бързодействие
    \end{itemize}
\end{remark}

\section*{Едно \textit{``ново''} понятие}

Специално за итеративните алгоритми се въвежда ново понятие - \textbf{инварианта}.
Човек може да си мисли за инвариантата като на друго име на индукцията:
\begin{itemize}
    \item тук отново си имаме база
    \item индукционното предположение и индукционната стъпка се обединяват в \textit{``нова''} фаза, наречена \textbf{поддръжка}
    \item довършителните ще наричаме \textbf{терминация}
\end{itemize}

\begin{warning}
    Това, за което се използват инвариантите, е да се докаже коректността на ЕДИН итеративен цикъл, не на цял алгоритъм.
    Когато в алгоритъма ни има няколко цикъла, на всеки от тях трябва да съответства по една инварианта.
\end{warning}

\section*{Един простичък пример}

Нека разгледаме следния алгоритъм за степенуване на $2$:
\inputminted[linenos]{c++}{algorithms/pow2.cpp}

\begin{invariant}
    При всяко достигане на ред $5$ имаме, че $\mathtt{result = 2^i}$.
\end{invariant}

\textbf{База.}
Наистина при първото достигане на ред $5$ имаме, че $\mathtt{i = 0}$ и от там $\mathtt{result = 1 = 2^i}$.

\textbf{Поддръжка.}
Нека при някое непоследно достигане твърдението е изпълнено.
Тогава във следващото достигане на цикъла на $\mathtt{i}$ присвояваме $\mathtt{i + 1}$, и след това на $\mathtt{result}$ присвояваме $\mathtt{result * 2}$, като знаем, че преди $\mathtt{result}$ е бил $\mathtt{2^{i_{old}}}$.
Така е ясно, че $\mathtt{result}$ ще стане $\mathtt{2^{i_{old} + 1} = 2^i}$.

\textbf{Терминация.}
Ако се намираме в последното достигане на ред $5$, то тогава $\mathtt{i = n}$, откъдето ще върнем $\mathtt{result = 2^n}$.

\pagebreak

\section*{С инвариантите трябва да се внимава}

Един от често срещаните капани, в които попадат хората, е да не си формулират инвариантата добре.
Много е важно инварианта да дава достатъчна информация за това което наистина се случва в алгоритъма.
За целта ще разгледаме един пример:
\inputminted[linenos]{c++}{algorithms/selection_sort.cpp}

На интуитивно ниво е ясно какво прави кода.
Намира най-малкия елемент, и го слага на първо място.
След това намира втория най-малък елемент, и го слага на второ място, и т.н.

Нещо, което някои биха се пробвали да направят за първия цикъл, е следното:
\begin{center}
    \textit{При всяко достигане на ред 3 подмасивът $\mathtt{arr[0 \dots i - 1]}$ е сортиран.}
\end{center}

Проблемът с това твърдение, е че може много лесно да се измисли алгоритъм, за който това твърдение е изпълнено, и изобщо не сортира елементите в масива:
\inputminted[linenos]{c++}{algorithms/trust_me_it_sorts.cpp}

Очевидно този за този алгоритъм горната инварианта е изпълнена, но той е безсмислен.
Получаваме сортиран масив, но за сметка на това губим цялата информация, която сме имали за него.

Нещо друго, което е важно да се направи, е първо да се формулира инварианта за вътрешния цикъл, и после за външния, като тънкият момент тук е, че ще ни трябват допускания за първата инварианта.
Идеята е, че външния цикъл разчита на вътрешния да си свърши работата, и обратно вътрешния разчита (не винаги) на външния преди това да си е свършил работата.

Тук може да се направи следното (доказателството остава за упражнение на читателя):

\begin{invariant}[вътрешен цикъл]
    При всяко достигане на ред $7$ имаме, че $\mathtt{min\_index}$ е индексът на най-малкия елемент в масива $\mathtt{arr[i \dots j - 1]}$.
\end{invariant}

\begin{invariant}[външен цикъл]
    При всяко достигане на ред $3$ имаме, че масивът $\mathtt{arr[0 \dots i - 1]}$ съдържа сортирани първите $\mathtt{i}$ по-големина елементи на $\mathtt{arr}$, като останалите се намират в $\mathtt{arr[i \dots n - 1]}$.
\end{invariant}

Обикновено в доказателството на коректност на алгоритми най-трудното е да се формулира инвариантата.
Ако човек има добре формулирана инварианта, доказателството и на първо място възможно, а на второ -- по-лесно.

\section*{Задачи}

\begin{problem}
Да се:
\begin{enumerate}
    \item напише алгоритъм, който сумира числата в един масив
    \item докаже неговата коректност
    \item изследва сложността му по време и памет
\end{enumerate}
\end{problem}

\pagebreak

\begin{problem}
Даден е следният алгоритъм:
\inputminted[linenos]{c++}{algorithms/alg.cpp}

\begin{enumerate}
    \item Какво връща той? Отговорът да се обоснове.
    \item Каква е неговата сложност по време и памет?
\end{enumerate}
\end{problem}

\begin{problem}
Даден е следният алгоритъм:
\inputminted[linenos]{c++}{algorithms/fibonacci.cpp}

Да се докаже, че $\mathtt{fib(n)}$ връща $\mathtt{n}$-тото число на Фибоначи.
\end{problem}

\begin{problem}
Даден е следният: алгоритъм:
\inputminted[linenos]{c++}{algorithms/mult.cpp}

Да се докаже че при вход $\mathtt{n \times n}$ матрици $\mathtt{A, B}$ и $\mathtt{C}$, функцията $\mathtt{mult(A, B, C, n)}$ записва в $\mathtt{C}$ произведението на $\mathtt{A}$ и $\mathtt{B}$.
Да се намери сложността му по-време и памет.
\end{problem}

\end{document}
