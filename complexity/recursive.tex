\section{Защо са ни рекурентни уравнения?}

Те се появяват по естествен път, когато искаме да анализираме сложността на рекурсивни алгоритми.

Нека вземем за пример алгоритъма за двоично търсене:
\lstinputlisting{algorithms/binary-search-rec.txt}

При подаден сортиран целочислен масив $A[1 \dots n]$, негови индекси $l, r$ и цяло число $v$, функцията $\mathtt{BinarySearch}(A[1 \dots n], 1, n, v)$ ще върне индекс на $A[1 \dots n]$, в който се намира $v$, ако има такъв, иначе ще върне $-1$.
Нека помислим каква е сложността на алгоритъма.
Управляващите параметри на рекурсията са $l$ и $r$.
Всеки път разликата между двете намалява двойно (като накрая когато $l = r$ тя ще стане отрицателна).

Това означава, че в най-лошия случай сложността на алгоритъма може да се опише със следното рекурентно уравнение:
\begin{align*}
     & T(0) = 2 \text{ // заради ред } 3 \text{ и } 4                                                          \\
     & T(n + 1) = T(\lfloor \frac{n + 1}{2} \rfloor) + 5 \text{ // заради проверките и рекурсивното извикване}
\end{align*}

В този случай лесно се вижда асимптотиката на $T(n)$:
\begin{align*}
    T(n) & = T(\lfloor \frac{n}{2} \rfloor) + 5                                                                                                         \\
         & = T(\lfloor \frac{n}{4} \rfloor) + 5 + 5                                                                                                     \\
         & = T(\lfloor \frac{n}{8} \rfloor) + 5 + 5 + 5 = \dots = T(0) + \underbrace{5 + \dots + 5}_{\text{около } \log(n) \text{ пъти}} \asymp \log(n)
\end{align*}

Така получаваме, че алгоритъмът има сложност $O(\log(n))$.
Обаче в общият случай далеч не е толкова лесно да се намери асимптотичното поведение на дадено рекурентно уравнение.
Целта ни ще бъде да развием по-богат апарат за асимптотичен анализ на рекурентните уравнения.

\section{Начини за намиране на асимптотиката на рекурентни уравнения}

Начините се разделят на два типа:
\begin{itemize}
    \item със решаване на уравнението;
    \item без решаване на уравнението.
\end{itemize}

И двата начина са ценни.
Първият начин ни дава формула във явен вид, което може да ни е от полза.
Понякога обаче формулата във явен вид не е \textit{``красива''}, или изобщо не може да се намери такава.
Тогава идва на помощ вторият начин.
Той директно ни дава някаква \textit{``хубава''} формула, без да трябва да намираме в явен вид решение на рекурентното уравнение.
Проблема е обаче, че асимптотиката понякога е малко лъжлива -- алгоритъм със сложност $2^{2^{2^{1000}}}$ е асимптотично по-бавен от алгоритъм със сложност $n$, но практически вторият е по-бърз.

Ще разгледаме следните методи (повечето от които са разглеждани по дискретна математика):
\begin{itemize}
    \item налучкване и доказване
    \item развиване (което преди малко показахме)
    \item методът с характеристичното уравнение
    \item мастър-теоремата
\end{itemize}

Нека разгледаме един пример с налучкване:
\begin{align*}
     & T(0) = 3                   \\
     & T(n + 1) = (n + 1)T(n) - n
\end{align*}

Започваме да разписваме:
\begin{center}
    \begin{tabular}{| c | c | c |}
        \hline
        $n$ & $T(n)$ & $n!$  \\
        \hline
        $0$ & $3$    & $1$   \\
        \hline
        $1$ & $3$    & $1$   \\
        \hline
        $2$ & $5$    & $2$   \\
        \hline
        $3$ & $13$   & $6$   \\
        \hline
        $4$ & $49$   & $24$  \\
        \hline
        $5$ & $241$  & $120$ \\
        \hline
        $6$ & $1441$ & $720$ \\
        \hline
    \end{tabular}
\end{center}

Вече лесно можем да покажем с индукция, че $T(n) = 2(n!) + 1$:
\begin{itemize}
    \item В базата имаме, че $T(0) = 3 = 2 \cdot 1 + 1 = 2 \cdot 0! + 1$.
    \item За индуктивната стъпка:
          \begin{align*}
              T(n + 1) & = (n + 1)T(n) - n \stackrel{\text{(ИП)}}{=} (n + 1)(2(n!) + 1) - n \\
                       & = (n + 1)(2(n!)) + n + 1 - n = 2(n + 1)! + 1
          \end{align*}
\end{itemize}
Накрая получаваме, че $T(n) \asymp n!$

Нека сега да видим как можем да използваме метода на характеристичното уравнение:
\[
    T(n) = 1 + \sum\limits_{i = 0}^{n - 1}T(i)    \text{ // функцията е добре дефинирана и за } 0.
\]
Рекурентното уравнение, зададено в този вид, не може да се реши с този метод.
За това ще трябва да направим преобразувания:
\begin{align*}
    T(0)     & = 1                                                                                                                                      \\
    T(n + 1) & = 1 + \sum\limits_{i = 0}^{n}T(i) = 1 + T(n) + \sum\limits_{i = 0}^{n - 1}T(i)                                                           \\
             & = T(n) + \underbrace{\left( 1 + \sum\limits_{i = 0}^{n - 1}T(i) \right)}_{T(n)} = 2T(n) + 1 = \underbrace{2T(n)}_{\text{хомогенна част}}
\end{align*}

Имаме само хомогенна част, от която получаваме характеристичното уравнение $x - 2 = 0$ с единствен корен $2$.
Така:
\[
    T(n) = A \cdot 2^n \text{ за някоя константи } A.
\]
Вече няма нужда и да се намира константата -- ясно е че $T(n) \asymp 2^n$.
Kато използваме метода на характеристичното уравнение, не е нужно да намираме накрая константите за да разберем каква е асимптотиката.
Достатъчно е да вземем събираемото, която расте най-много. В случая е ясно, че това е $2^n$.

\newpage

Нека сега разгледаме и последният начин:
\begin{theorem}[Мастър-теорема]
    Нека $a \geq 1, \: b > 1$ и $f \in \calF$.
    Нека $T(n) = aT(\frac{n}{b}) + f(n)$, където $\frac{n}{b}$ се интерпретира като $\lfloor \frac{n}{b} \rfloor$ или $\lceil \frac{n}{b} \rceil$.
    Тогава:
    \begin{itemize}
        \item[1 сл.] Ако $f(n) \preceq n^{\log_b(a) - \varepsilon}$ за някое $\varepsilon > 0$, то тогава $T(n) \asymp n^{log_b(a)}$.
        \item[2 сл.] Ако $f(n) \asymp n^{log_b(a)}$, то тогава $T(n) \asymp n^{log_b(a)} \log(n)$.
        \item[3 сл.] Ако са изпълнени следните условия:
              \begin{enumerate}
                  \item $f(n) \succeq n^{\log_b(a) + \varepsilon}$ за някое $\varepsilon > 0$; и
                  \item съществува $0 < c < 1$, за което от някъде нататък $a \cdot f(\frac{n}{b}) \leq c \cdot f(n)$,
              \end{enumerate}
              то тогава $T(n) \asymp f(n)$.
    \end{itemize}
\end{theorem}

Нека разгледаме рекурентното уравнение:
\[
    T(n) = 2T(\frac{n}{2}) + 1.
\]
Тук $a = b = 2$, и $f(n) = 1$.
Също така $\log_b(a) = 1$, откъдето $f(n) = 1 \preceq n^{\log_b(a) - \varepsilon}$, за $\varepsilon \in (0, 1)$.
Така по 1 сл. на мастър-теоремата получаваме, че $T(n) \asymp n$.