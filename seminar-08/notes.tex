\documentclass{article}

\usepackage[utf8]{inputenc}
\usepackage[T2A]{fontenc}
\usepackage[english, bulgarian]{babel}
\usepackage{pgfplots}
\usepackage{amssymb}
\usepackage{hyperref, fancyhdr, lastpage, fancyvrb, tcolorbox, titlesec}
\usepackage{array, tabularx, colortbl}
\usepackage{tikz}
\usepackage{venndiagram}
\usepackage{amsthm, bm}
\usepackage{relsize}
\usepackage{amsmath,physics}
\usepackage{mathtools}
\usepackage{subcaption}
\usepackage{theoremref}
\usepackage{circuitikz}
\usepackage[a4paper, left=0.50in, right=0.50in, top=0.5in, bottom=1.0in]{geometry}
\usepackage{stmaryrd}
\usepackage[symbol]{footmisc}
\usepackage{minted}
\usepackage{enumitem}
\usepackage{systeme}

\pgfplotsset{width=10cm,compat=1.9}

\newcommand{\N}{\mathbb{N}}
\newcommand{\R}{\mathbb{R}}
\newcommand{\F}{\mathcal{F}}

\ExplSyntaxOn
\NewDocumentCommand{\opair}{m}
{
  \langle\mspace{2mu}
  \clist_set:Nn \l_tmpa_clist { #1 }
  \clist_use:Nn \l_tmpa_clist {,\mspace{3mu plus 1mu minus 1mu}\allowbreak}
  \mspace{2mu}\rangle
}
\ExplSyntaxOff

\hypersetup{
  colorlinks=true,
  linktoc=all,
  linkcolor=blue
}

\theoremstyle{definition}
\newtheorem{definition}{Дефиниция}[section]
\newtheorem*{warning}{\textcolor{red}{Внимание}}
\theoremstyle{plain}
\newtheorem*{theorem}{Теорема}
\newtheorem*{invariant}{Инвариант}
\newtheorem{claim}[definition]{Твърдение}
\newtheorem{axiom}[definition]{Аксиома}
\newtheorem{lemma}[definition]{Лема}
\newtheorem{corollary}[definition]{Следствие}
\theoremstyle{remark}
\newtheorem*{remark}{Забележка}
\newtheorem{problem}{Задача}
\newtheorem*{solution}{Решение}
\theoremstyle{definition}

\setlength\parindent{0pt}

\title{Общи задачи}
\author{Тодор Дуков}
\date{}

\begin{document}
\maketitle

\section*{Примери за задачи с числа}

Искаме да напишем алгоритъм, който при подадена двойка от естествени числа $\opair{a, b}$, където $b \leq a$, да върне $\gcd(a, b)$ т.е. това число $d \in \mathbb{N}$, за което:
\begin{itemize}
  \item $d \mid a$ и $d \mid b$
  \item ако $d_1 \mid a$ и $d_1 \mid b$, то $d_1 \mid d$
\end{itemize}

За целта ще покажем, че за всяко $a, b, q, r \in \mathbb{N}$, където $0 < b \leq a, \: a = bq + r$ и $r \in \{ 0, \dots, b - 1 \}$, е изпълнено, че:
\[
  \gcd(a, b) = \gcd(b, r)
\]
Първо имаме, че $\gcd(a, b) \mid a$ и $\gcd(a, b) \mid b$, откъдето понеже $a = bq + r$, имаме $\gcd(a, b) \mid r$.
Тогава $\gcd(a, b) \mid \gcd(b, r)$.
От друга страна $\gcd(b, r) \mid b$ и $\gcd(b, r) \mid r$, откъдето $\gcd(b, r) \mid bq + r = a$.
Така $\gcd(b, r) \mid \gcd(a, b)$.
Накрая можем да заключим $\gcd(a, b) = \gcd(b, r)$ от $\gcd(a, b) \mid \gcd(b, r)$ и $\gcd(b, r) \mid \gcd(a, b)$.

На базата на това наблюдение се получава следният алгоритъм (кръстен на Евклид):
\inputminted[linenos]{c++}{algorithms/euclid.cpp}

Ще покажем с индукция относно $\mathtt{b}$, че за всяко $\mathtt{a \geq b}$, е изпълнено, че:
\[
  \mathtt{euclid(a, b) = gcd(a, b)}
\]
\begin{itemize}
  \item В базовия случай имаме $\mathtt{euclid(a, 0) = a = gcd(a, 0)}$
  \item Нека $\mathtt{b > 0}$ и $\mathtt{a = bq + r}$ за някои $\mathtt{r \in \{ 0, \dots, b - 1 \}}$ и $\mathtt{q}$ и нека твърдението е изпълнено за всяко $\mathtt{b' < b}$.
        Тогава:
        \[
          \mathtt{euclid(a, b) = euclid(b, r) \stackrel{\text{(ИП)}}{=} gcd(b, r) = gcd(a, b)}
        \]
\end{itemize}

Алгоритъмът терминира, понеже управляващият параметър $\mathtt{b}$ винаги намаля, докато не стане $\mathtt{0}$.
На пръсти ще покажем, че алгоритъмът има сложност по време и памет $O(\log(a))$.
Нека видим, че ако $a > b$, то $\bmod(a, b) < \frac{a}{2}$:
\begin{itemize}
  \item[1 сл.] ако $b \leq \frac{a}{2}$, то $\bmod(a, b) < \frac{a}{2}$ и сме готови
  \item[2 сл.] ако пък $b > \frac{a}{2}$, то тогава $a - b < \frac{a}{2}$, откъдето $\bmod(a, b) = \bmod(a - b, b) = a - b < \frac{a}{2}$
\end{itemize}
Тогава през на всеки две стъпки на алгоритъма от вход $\opair{a, b}$, отиваме до вход $\opair{\bmod(a, b), \bmod(b, \bmod(a, b))}$ и левият аргумент става поне два пъти по-малък.
Тъй като левият аргумент е горна граница за десния, то той също ще намаля рязко.
Дълбочината на дървото на рекурсията зависи само от броя на рекурсивните извиквания.
Понеже те са логаритмично много и всяко едно от тях заема константна памет, сложността по памет е също $O(\log(a))$.

Втората задача за числа е скрита в задача за масиви:

Имаме един масив $\mathtt{A[1 \dots n]}$, който съдържа всички числа от $\mathtt{1}$ до $\mathtt{n}$, с изключение на едно от тях, което е заместено с друго число от $\mathtt{1}$ до $\mathtt{n}$.
Искаме да напишем колкото се може по-бърз алгоритъм, който при вход такъв масив $\mathtt{A}$ връща наредената двойка $\mathtt{\opair{dup, miss}}$ от дупликата и липсващото число.
Оказва се, че тази задача може да се реши за линейно време и константна памет.
За целта трябва да се забележат следните факти:
\begin{itemize}
  \item $\mathtt{dup - miss = \sum\limits_{i = 1}^n A[i] - \sum\limits_{i = 1}^n i = \sum\limits_{i = 1}^n A[i] - \frac{n(n + 1)}{2} =: S_{1, A}}$
  \item $\mathtt{(dup + miss)(dup - miss) = dup^2 - miss^2 = \sum\limits_{i = 1}^n A[i]^2 - \sum\limits_{i = 1}^n i^2 = \sum\limits_{i = 1}^n A[i]^2 - \frac{n(n + 1)(2n + 1)}{6} =: S_{2, A}}$
\end{itemize}

\pagebreak

От тях получаваме следната система от уравнения:
\begin{center}
  \begin{tabular}{|l}
    $\mathtt{dup - miss = S_{1, A}}$ \\
    $\mathtt{dup + miss = \frac{S_{2, A}}{S_{1, A}}}$
  \end{tabular}
\end{center}

Най-сложното, което трябва да направим, е да пресметнем $\mathtt{S_{1, A}}$ и $\mathtt{S_{2, A}}$:
\inputminted[linenos]{c++}{algorithms/find_missing_and_duplicate.cpp}

Остава само да се докаже следното твърдение:

\begin{invariant}
  При всяко достигане на проверката за край на цикъла на ред $9$ имаме, че:
  \[
    \mathtt{arr\_sum = \sum\limits_{k = 0}^{i - 1} arr[k] \text{ и } arr\_sq\_sum = \sum\limits_{k = 0}^{i - 1} arr[k]^2}
  \]
\end{invariant}

\textbf{База.}
При първото достигане имаме, че:
\begin{itemize}
  \item $\mathtt{arr\_sum = 0 = \sum\limits_{k = 0}^{0 - 1} arr[k]}$
  \item $\mathtt{arr\_sq\_sum = 0 = \sum\limits_{k = 0}^{0 - 1} arr[k]^2}$
\end{itemize}

\textbf{Поддръжка.}
Нека твърдението е изпълнено за някое непоследно достигане на проверката за край на цикъла.
Тогава влизайки в тялото на цикъла:
\begin{itemize}
  \item $\mathtt{arr\_sum}$ става $\mathtt{arr\_sum + arr[i] \stackrel{\text{(ИП)}}{=} \sum\limits_{k = 0}^{i - 1} arr[k] + arr[i] = \sum\limits_{k = 0}^{i} arr[k] = \sum\limits_{k = 0}^{i_{new} - 1} arr[k]}$
  \item $\mathtt{arr\_sq\_sum}$ става $\mathtt{arr\_sq\_sum + arr[i]^2 \stackrel{\text{(ИП)}}{=} \sum\limits_{k = 0}^{i - 1} arr[k]^2 + arr[i]^2 = \sum\limits_{k = 0}^{i} arr[k]^2 = \sum\limits_{k = 0}^{i_{new} - 1} arr[k]^2}$
\end{itemize}

\textbf{Терминация.}
В последното достигане на проверката за край на цикъла имаме, че $\mathtt{i = n}$, откъдето:
\[
  \mathtt{arr\_sum = \sum\limits_{i = 0}^{n - 1} arr[i] \text{ и } arr\_sq\_sum = \sum\limits_{i = 0}^{n - 1} arr[i]^2}
\]

Имайки това твърдение, остава само да се отбележи, че $\mathtt{dup}$ и $\mathtt{miss}$ наистина са решения на горната система от уравнения (със съответните преиндексирания).
Накрая нека за пълнота с индукция по $n \in \mathbb{N}$ покажем, че:
\[
  \sum\limits_{i = 0}^n i = \frac{n(n + 1)}{2} \text{ и } \sum\limits_{i = 0}^n i^2 = \frac{n(n + 1)(2n + 1)}{6}
\]
\begin{itemize}
  \item $\sum\limits_{i = 0}^0 i = 0 = \frac{0(0 + 1)}{2}$ и $\sum\limits_{i = 0}^0 i^2 = \frac{0(0 + 1)(2 \cdot 0 + 1)}{6}$ \checkmark
  \item $\sum\limits_{i = 0}^{n + 1} i = \sum\limits_{i = 0}^{n} i + (n + 1) \stackrel{\text{(ИП)}}{=} \frac{n(n + 1)}{2} + (n + 1) = \frac{n(n + 1)}{2} + \frac{2(n + 1)}{2} = \frac{(n + 1)(n + 2)}{2}$
  \item $\sum\limits_{i = 0}^{n + 1} i^2 = \sum\limits_{i = 0}^{n} i^2 + (n + 1)^2 \stackrel{\text{(ИП)}}{=} \frac{n(n + 1)(2n + 1)}{6} + (n + 1)^2 = \frac{n(n + 1)(2n + 1)}{6} + \frac{6(n + 1)^2}{6} = \frac{(n + 1)(2n^2 + n + 6n + 6)}{6} = \frac{(n + 1)(n + 2)(2(n + 1) + 1)}{6}$
\end{itemize}

\section*{Задачи}

\begin{problem}
Да се напише алгоритъм $\mathtt{find\_primes(n)}$, който приема естествено число $\mathtt{n}$ и връща булев масив $\mathtt{P[2 \dots n]}$, за който е изпълнено, че за всяко $\mathtt{2 \leq i \leq n}$:
\[
  \mathtt{P[i] \text{ е истина } \iff i \text{ е просто}}
\]
След това да се докаже неговата коректност и да се изследва сложността му по време и памет.
\end{problem}

\begin{problem}
Масив на Монж ще наричаме всеки двумерен масив $\mathtt{A[1 \dots n, 1 \dots m]}$ от естествени числа, за който:
\[
  \mathtt{A[p, q] + A[s, t] \leq A[p, t] + A[s, q] \text{ за всички } 1 \leq p, s \leq n \text{ и } 1 \leq q, t \leq n}
\]
Да се напише алгоритъм $\mathtt{test\_mongeness(A)}$ със линейна сложност, който приема двумерен масив $\mathtt{A[1 \dots n, 1 \dots m]}$ и проверява дали е масив на Монж.
След това да се докаже неговата коректност и да се изследва сложността му по време и памет.
\end{problem}

\begin{problem}
Да се напише колкото се може по-бърз алгоритъм $\mathtt{split(A)}$, който приема масив $\mathtt{A[1 \dots n]}$ от цели числа и го подрежда така, че всички отрицателни числа да са вляво от всички неотрицателни.
След това да се докаже неговата коректност и да се изследва сложността му по време и памет.
\end{problem}

\begin{problem}
Да се напише колкото се може по-бърз алгоритъм $\mathtt{find\_products(A)}$, който приема масив $\mathtt{A[1 \dots n]}$ от цели числа и връща масив $\mathtt{B[1 \dots n]}$, такъв че за всяко $\mathtt{1 \leq i \leq n}$:
\[
  \mathtt{B[i] = \prod\limits_{\substack{\mathtt{k = 1} \\ \mathtt{k \neq i}}}^{n} A[i]}
\]
След това да се докаже неговата коректност и да се изследва сложността му по време и памет.
\end{problem}

\begin{problem}
Да се напише колкото се може по-бърз алгоритъм $\mathtt{find\_max\_product\_pair(A)}$, който приема масив $\mathtt{A[1 \dots n]}$ от цели числа и връща двойка $\mathtt{\opair{i, j}}$ от различни индекси, за които $\mathtt{A[i] * A[j]}$ е максимално.
След това да се докаже неговата коректност и да се изследва сложността му по време и памет.
\end{problem}

\begin{problem}
Даден е следният алгоритъм:
\inputminted[linenos]{c++}{algorithms/alg.cpp}
Какво връща $\mathtt{alg(arr, n)}$ където $\mathtt{arr}$ е масив от цели числа с дължина $\mathtt{n}$?
Обосновете отговора си.
\end{problem}

\begin{problem}
Да се напише колкото се може по-бърз алгоритъм, който при подадено крайно множество от точки $P \subseteq \mathbb{Z} \cross \mathbb{Z}$, намира $\max \{ |x_1 - x_2| + |y_1 - y_2| \mid (x_1, y_1), (x_2, y_2) \in P \}$.
След това да се докаже неговата коректност и да се изследва сложността му по време и памет.
\end{problem}

\end{document}
