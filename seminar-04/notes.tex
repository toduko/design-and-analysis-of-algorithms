\documentclass{article}

\usepackage[utf8]{inputenc}
\usepackage[T2A]{fontenc}
\usepackage[english, bulgarian]{babel}
\usepackage{pgfplots}
\usepackage{amssymb}
\usepackage{hyperref, fancyhdr, lastpage, fancyvrb, tcolorbox, titlesec}
\usepackage{array, tabularx, colortbl}
\usepackage{tikz}
\usepackage{venndiagram}
\usepackage{amsthm, bm}
\usepackage{relsize}
\usepackage{amsmath,physics}
\usepackage{mathtools}
\usepackage{subcaption}
\usepackage{theoremref}
\usepackage{circuitikz}
\usepackage[a4paper, left=0.50in, right=0.50in, top=0.5in, bottom=1.0in]{geometry}
\usepackage{stmaryrd}
\usepackage[symbol]{footmisc}
\usepackage{minted}
\usepackage{enumitem}

\pgfplotsset{width=10cm,compat=1.9}

\newcommand{\N}{\mathbb{N}}
\newcommand{\R}{\mathbb{R}}
\newcommand{\F}{\mathcal{F}}

\ExplSyntaxOn
\NewDocumentCommand{\opair}{m}
{
  \langle\mspace{2mu}
  \clist_set:Nn \l_tmpa_clist { #1 }
  \clist_use:Nn \l_tmpa_clist {,\mspace{3mu plus 1mu minus 1mu}\allowbreak}
  \mspace{2mu}\rangle
}
\ExplSyntaxOff

\hypersetup{
  colorlinks=true,
  linktoc=all,
  linkcolor=blue
}

\theoremstyle{definition}
\newtheorem{definition}{Дефиниция}[section]
\newtheorem*{warning}{\textcolor{red}{Внимание}}
\theoremstyle{plain}
\newtheorem{theorem}[definition]{Теорема}
\newtheorem*{invariant}{Инвариант}
\newtheorem{claim}[definition]{Твърдение}
\newtheorem{axiom}[definition]{Аксиома}
\newtheorem{lemma}[definition]{Лема}
\newtheorem{corollary}[definition]{Следствие}
\theoremstyle{remark}
\newtheorem*{remark}{Забележка}
\newtheorem{problem}{Задача}
\newtheorem*{solution}{Решение}
\theoremstyle{definition}

\setlength\parindent{0pt}

\title{Още итеравивни алгоритми}
\author{Тодор Дуков}
\date{}

\begin{document}
\maketitle

\section*{Подход при задачи с вече даден алгоритъм}

Нека разгледаме следния алгоритъм:
\inputminted[linenos]{c++}{algorithms/foo.cpp}
Питаме се какво връща той?

Обикновено в такива задачи трябва да се изпробва алгоритъма върху няколко стойности.
Можем да забележим, че:
\begin{itemize}
    \item $\mathtt{foo(0) \text{ връща } 0}$
    \item $\mathtt{foo(1) \text{ връща } 1}$
    \item $\mathtt{foo(2) \text{ връща } 8}$
    \item $\mathtt{foo(3) \text{ връща } 27}$
    \item $\mathtt{foo(4) \text{ връща } 64}$
    \item $\mathtt{foo(5) \text{ връща } 125}$
    \item $\mathtt{foo(6) \text{ връща } 216}$
    \item $\mathtt{foo(7) \text{ връща } 343}$
    \item $\mathtt{foo(8) \text{ връща } 512}$
\end{itemize}

Вече можем да видим какво прави алгоритъмът -- $\mathtt{foo(a) \text{ връща } a^3}$.
Нека сега докажем това:
\begin{invariant}
    При всяко достигане на проверката за край на цикъла на ред $5$ имаме, че:
    \begin{itemize}
        \item $\mathtt{x = 6 (1 + i)}$
        \item $\mathtt{y = 3i^2 + 3i + 1}$
        \item $\mathtt{z = i^3}$
    \end{itemize}
\end{invariant}

\textbf{База.}
При първото достигане имаме, че:
\begin{itemize}
    \item $\mathtt{i = 0}$
    \item $\mathtt{x = 6 = 6 \cdot 1 = 6 (1 + 0) = 6 (1 + i)}$
    \item $\mathtt{y = 1 = 3 \cdot 0^2 + 3 \cdot 0 + 1 = 3i^2 + 3i + 1}$
    \item $\mathtt{z = 0 = 0^3 = i^3}$
\end{itemize}

\textbf{Поддръжка.}
Нека твърдението е изпълнено за някое непоследно достигане на проверката за край на цикъла.
Тогава при влизане в тялото на цикъла:
\begin{itemize}
    \item $\mathtt{z \text{ ще стане } z + y \stackrel{\text{(ИП)}}{=} i^3 + 3i^2 + 3i + 1 = (\underbrace{\mathtt{i + 1}}_{\text{ново } i})^3}$
    \item $\mathtt{y \text{ ще стане } y + x \stackrel{\text{(ИП)}}{=} 3i^2 + 3i + 1 + 6 + 6i = 3(\underbrace{\mathtt{i + 1}}_{\text{ново } i})^2 + 3(\underbrace{\mathtt{i + 1}}_{\text{ново } i}) + 1}$
    \item $\mathtt{x \text{ ще стане } x + 6 \stackrel{\text{(ИП)}}{=} 6(1 + i) + 6 = 6(1 + \underbrace{\mathtt{i + 1}}_{\text{ново } i})}$
\end{itemize}

\textbf{Терминация.}
В последното достигане на проверката за край на цикъла имаме, че $\mathtt{i = a}$, и тогава на ред $12$ алгоритъмът ще върне $\mathtt{z = a^3}$.

Нека разгледаме още един такъв пример:
\inputminted[linenos]{c++}{algorithms/quux.cpp}

Искаме да видим какво връща $\mathtt{quux(arr, n)}$ където $\mathtt{arr}$ е масив от цели числа а $\mathtt{n}$ е броят на неговите елементи.
Тук най-трудното, което трябва да се направи, е да се определи какво връщат $\mathtt{bar}$ и $\mathtt{foo}$:
\begin{itemize}
    \item $\mathtt{bar(n) = \sqrt{n ^ 2} = | n |}$:
          \begin{itemize}
              \item ако $\mathtt{n \geq 0}$, то $\mathtt{\sqrt{n^2} = n = | n |}$
              \item ако $\mathtt{n < 0}$, то $\mathtt{\sqrt{n^2} = -n = | n |}$
          \end{itemize}
    \item $\mathtt{foo(x, y) = \frac{x + y + |x - y|}{2}  = \max\{ x, y \}}$:
          \begin{itemize}
              \item ако $\mathtt{x \geq y}$, то $\mathtt{\frac{x + y + |x - y|}{2} = \frac{x + y + x - y}{2} = \frac{2x}{2} = x = \max\{ x, y \}}$
              \item ако $\mathtt{x < y}$, то $\mathtt{\frac{x + y + |x - y|}{2} = \frac{x + y + y - x}{2} = \frac{2y}{2} = y = \max\{ x, y \}}$
          \end{itemize}
\end{itemize}

Като знаем какво правят $\mathtt{bar}$ и $\mathtt{foo}$, можем да забележим, че $\mathtt{quux(arr, n)}$ дава най-големия елемент на $\mathtt{arr}$:
\begin{invariant}
    При всяко достигане на проверката за край на цикъла на ред $15$ имаме, че $\mathtt{a = \max arr[0 \dots i - 1]}$.
\end{invariant}

\textbf{База.}
При първото достигане имаме, че $\mathtt{a = arr[0] = \max arr[0 \dots 1 - 1] = \max arr[0 \dots i - 1]}$.

\textbf{Поддръжка.}
Нека твърдението е изпълнено за някое непоследно достигане на проверката за край на цикъла.
Тогава като влезем в тялото на цикъла, променливата $\mathtt{a}$ става:
\begin{align*}
    \mathtt{foo(a, arr[i]) \stackrel{\text{(ИП)}}{=} foo(\max arr[0 \dots i - 1], arr[i])} & = \mathtt{\max (\max arr[0 \dots i - 1], arr[i])}                                                  \\
                                                                                           & =\mathtt{\max arr[0 \dots i] = \max arr[i \dots \underbrace{\mathtt{i + 1}}_{\text{ново } i} - 1]}
\end{align*}

\textbf{Терминация.}
При последното достигане на проверката за край на цикъла на ред $15$ променливата $\mathtt{i}$ ще бъде $\mathtt{n}$, откъдето функцията ще върне точно $\mathtt{a = \max arr[0 \dots n - 1]}$, с което сме готови.

\section*{Задачи}

\begin{problem}
Даден е следният алгоритъм:
\inputminted[linenos]{c++}{algorithms/num_slopes.cpp}

Да се докаже, че при подаден масив от цели числа $\mathtt{arr}$ със размер $\mathtt{n}$, функцията $\mathtt{num\_slopes(arr, n)}$ връща броят на ненамаляващи подмасиви на $\mathtt{arr}$.
\end{problem}


\begin{problem}
Даден е следният алгоритъм:
\inputminted[linenos]{c++}{algorithms/kadane.cpp}

Какво връща $\mathtt{kadane(arr, n)}$, където $\mathtt{arr}$ е масив от цели числа с дължина $\mathtt{n}$?
Обосновете отговора си.
\end{problem}

\begin{problem}
Даден е следният алгоритъм:
\inputminted[linenos]{c++}{algorithms/find_majority.cpp}

Да се докаже, че при подаден масив от цели числа $\mathtt{arr}$ със размер $\mathtt{n}$, в който има елемент с повече от $\mathtt{\lfloor \frac{n}{2} \rfloor}$ срещания,
функцията $\mathtt{find\_majority(arr, n)}$ ще върне точно този елемент.
\end{problem}

\end{document}
