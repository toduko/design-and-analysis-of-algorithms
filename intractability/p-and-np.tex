
\chapter{Алгоритмична неподатливост}

\section{Класове на сложност \textbf{P} и \textbf{NP}}

При решаването на различни видове задачи, които ни интересуват, е естествено да се опитаме да ги класифицираме по това колко са \textit{``сложни''}.
Това сме го виждали вече -- в курса по ЕАИ сме класифицирали различни езици спрямо това каква машина може да решава въпроса за принадлежност към съответния език.
Тук ще направим нещо подобно, разликата ще бъде в това, че ще се интересуваме от това за какво време се решава една задача.
\begin{itemize}
      \item Класът на сложност \textbf{P} ще бъде множеството от всички изчислителни задачи за разпознаване, за които съществува алгоритъм с полиномиална времева сложност при най-лоши входни данни.
      \item Класът на сложност \textbf{NP} ще бъде множеството от всички изчислителни задачи за разпознаване, за които съществува алгоритъм с полиномиална времева сложност при всякакви входни данни, проверяващ отговора ``ДА'' на задачата с помощта на допълнителен параметър, наречен \textbf{сертификат}, който зависи от входните данни на задачата и чиято дължина е полиномиална спрямо тяхната.
\end{itemize}

\begin{remark}
      За класа \textbf{NP} има и алтернативна дефиниция -- в нея не е нужен сертификат, но се допуска алгоритъмът да е недетерминиран.
      Идеята е, че той едновременно \textit{``познава''} правилния отговор и го верифицира.
\end{remark}

\newpage

Нека дадем пример за сертификат.
Да кажем, че търсим отговор на въпроса:
\begin{center}
      \textit{Вярно ли е, че уравнението $a_0 + a_1 x + \dots + a_n x^n = 0$ има реален корен?}
\end{center}
По-лесно ще бъде да верифицираме някое предложено решение, т.е. да отговорим на въпроса:
\begin{center}
      \textit{Вярно ли е, че реалното число $x_0$ е корен на уравнението $a_0 + a_1 x + \dots + a_n x^n = 0$?}
\end{center}
В него имаме още един параметър -- предложеното решение $x_0$.
В този случай това ще бъде сертификатът.
Въпреки че в примера е направено така, не е нужно сертификатът да се използва.
Той е само помощен и съществуването му е нужно само при отговор ``ДА''.
Ясно е, че при отговор ``НЕ'' няма и как да има сертификат.

Неформално казано, в класа \textbf{P} се намират задачите, които се решават \textit{``бързо''} (за полиномиално време), а в класа \textbf{NP} се намират задачите, чиито решения се верифицират \textit{``бързо''}.
Лесно може да се види, че $\mathbf{P} \subseteq \mathbf{NP}$.
Ако една задача може да се реши за полиномиално време без да използва сертификат, то тогава тя може да се реши и със използване на сертификат.

\section{Няколко важни задачи за класа \textbf{NP}}

Следните задачи за изключително важни за класа \textbf{NP} (по-късно ще разберем защо):
\begin{itemize}
      \item Задачата \textbf{SAT}:

            \textbf{Вход:} Съждителна формула $\varphi$ в конюнктивна нормална форма.

            \textbf{Въпрос:} Има ли оценка, в която $\varphi$ е вярна?
      \item Задачата \textbf{3SAT}:

            \textbf{Вход:} Съждителна формула $\varphi$ в конюнктивна нормална форма, при която във всяка дизюнктивна клауза участват точно три литерала.

            \textbf{Въпрос:} Има ли оценка, в която $\varphi$ е вярна?
      \item Задачата \textbf{SubsetSum}:

            \textbf{Вход:} Масив $A[1 \dots n]$ от положителни числа и естествено число $s$.

            \textbf{Въпрос:} Има ли $I \subseteq \{ 1, \dots, n \}$, за което $\sum\limits_{i \in I} A[i] = s$?
      \item Задачата \textbf{2Partition}:

            \textbf{Вход:} Масив $A[1 \dots n]$ от положителни числа.

            \textbf{Въпрос:} Има ли $I \subseteq \{ 1, \dots, n \}$, за което $\sum\limits_{i \in I} A[i] = \sum\limits_{i \in \{ 1, \dots, n \} \setminus I} A[i]$?
      \item Задачата \textbf{VertexCover}:

            \textbf{Вход:} Граф $G = \opair{V, E}$ и естествено число $k$.

            \textbf{Въпрос:} Има ли $X \subseteq V$, за което $|X| \leq k$ и $X$ съдържа поне един край на всяко ребро от $E$?
      \item Задачата \textbf{DominatingSet}:

            \textbf{Вход:} Граф $G = \opair{V, E}$ и естествено число $k$.

            \textbf{Въпрос:} Има ли $X \subseteq V$, за което $|X| \leq k$ и всеки връх от $V \setminus X$ е съседен на някой от $X$?
      \item Задачата \textbf{Clique}:

            \textbf{Вход:} Граф $G = \opair{V, E}$ и естествено число $k$.

            \textbf{Въпрос:} Има ли клика $X \subseteq V$, за която $|X| \geq k$?
      \item Задачата \textbf{Anticlique}:

            \textbf{Вход:} Граф $G = \opair{V, E}$ и естествено число $k$.

            \textbf{Въпрос:} Има ли антиклика $X \subseteq V$, за която $|X| \geq k$?
      \item Задачата \textbf{HamiltonianPath}:

            \textbf{Вход:} Граф $G = \opair{V, E}$ и два върха $s, e \in V$.

            \textbf{Въпрос:} Има ли Хамилтонов път в $G$ от $s$ до $e$?
      \item Задачата \textbf{SubgraphIsomorphism}:

            \textbf{Вход:} Два графа $G_1 = \opair{V_1, E_1}$ и $G_2 = \opair{V_2, E_2}$.

            \textbf{Въпрос:} Има ли подграф $G$ на $G_1$, който е изоморфен на $G_2$?
      \item Задачата \textbf{TSP}:

            \textbf{Вход:} Тегловен граф $G = \opair{V, E, w}$ и естествено число $k$.

            \textbf{Въпрос:} Има ли Хамилтонов път в $G$ с тегло ненадвишаващо $k$?
\end{itemize}

Нека покажем за някои от тях, че попадат в класа \textbf{NP}.

Да започнем със \textbf{SAT}.
Трябва да предложим полиномиален алгоритъм, който с помощта на сертификат проверява за отговор ``ДА''.
Сертификатът ще бъде оценката.
Оценката може да се представи като масив от променливи $V[1 \dots k]$, чиито членове са променливите, които имат стойност ``истина''.
Ясно е, че тази кодировка е с полиномиална относно формулата дължина.
При дадена оценка лесно можем да видим дали формулата е вярна:
\VerbatimInput[numbersep=3pt, frame=single, numbers=left,commandchars=\\\{\},codes={\catcode`$=3}]{algorithms/sat.txt}
Този алгоритъм очевидно работи за полиномиално време.
Наистина, той проверява дали $V[1 \dots k]$ задава оценка, в която $\varphi$ е вярна.
Така \textbf{SAT} е в класа \textbf{NP}.
Тъй като \textbf{3SAT} е просто по-лесна версия на \textbf{SAT}, тя също попада в класа \textbf{NP}.

Нека сега видим, че \textbf{SubsetSum} е в класа \textbf{NP}.
Тук сертификатът ще бъде масив $I[1 \dots k]$, който ще представя множество от индекси за входния масив.
При дадено множество от индекси лесно можем да проверим дали е изпълнено условието за сумата:
\VerbatimInput[numbersep=3pt, frame=single, numbers=left,commandchars=\\\{\},codes={\catcode`$=3}]{algorithms/subset-sum.txt}
Този алгоритъм очевидно работи за полиномиално време.
Наистина, той проверява дали $I[1 \dots k]$ задава подмножество $I$ на $\{ 1, \dots, n \}$, за което $\sum\limits_{i \in I} A[i] = s$.
Така \textbf{SubsetSum} е в класа \textbf{NP}.

Да видим сега, че \textbf{VertexCover} е в класа \textbf{NP}.
Тук сертификатът ще бъде масив $X[1 \dots n]$, който ще представя множеството от върховете, които ще образуват потенциално върхово покритие.
При дадено множество от върхове лесно можем да видим дали то е върхово покритие с размер най-много $k$:
\VerbatimInput[numbersep=3pt, frame=single, numbers=left,commandchars=\\\{\},codes={\catcode`$=3}]{algorithms/vertex-cover.txt}
Този алгоритъм очевидно работи за полиномиално време.
Наистина, той проверява дали $X[1 \dots n]$ задава върхово покритие на $G$ с най-много $k$ елемента.
Така \textbf{VertexCover} е в класа \textbf{NP}.

Сега ще покажем, че \textbf{Clique} е в класа \textbf{NP}.
Тук сертификатът ще бъде масив $X[1 \dots n]$, който ще представя множеството от върховете, които ще образуват потенциална клика.
При дадено множество от върхове лесно можем да видим дали то е клика с размер поне $k$:
\VerbatimInput[numbersep=3pt, frame=single, numbers=left,commandchars=\\\{\},codes={\catcode`$=3}]{algorithms/clique.txt}
Този алгоритъм очевидно работи за полиномиално време.
Наистина, той проверява дали $X[1 \dots n]$ задава клика в $G$ с поне $k$ елемента.
Така \textbf{Clique} е в класа \textbf{NP}.