\section{$\mathbf{P = NP}$ или $\mathbf{P \neq NP}$? Колко пък да е трудно?}

За съжаление все още няма отговор на този въпрос -- не се знае дали $\mathbf{P = NP}$  или $\mathbf{P \neq NP}$.
Най-доброто, което имаме до момента, са няколко задачи в класа \NP, които са изключително важни.
Тези задачи в някакъв смисъл характеризират целия клас.
Тях ще ги наричаме \NP-пълни.
Формалната дефиниция е следната, една изчислителна задача \textbf{X} е \NP-пълна, ако:
\begin{itemize}
    \item \textbf{X} е в класа \NP;
    \item всяка задача \textbf{Y} в класа \NP{ }може алгоритмично да се сведе до задачата \textbf{X} за полиномиално време.
\end{itemize}
Фактът, че \textbf{Y} се свежда до \textbf{X} за полиномиално време ще бележим с $\mathbf{Y} \leq_p \mathbf{X}$\footnote{На други места вместо $\leq_p$ се използват означенията $\leq^p_m$ и $\propto_p$.}.
Лесно се вижда, че $\leq_p$ е транзитивна.
Когато второто условие е изпълнено (без да е непременно изпълнено първото) казваме, че задачата \textbf{X} е \NP-трудна.

Интересното за \NP-пълните задачи е следното:
\begin{itemize}
    \item ако покажем, че която и да е \NP-пълна задача се решава за полиномиално време, то тогава $\mathbf{P = NP}$;
    \item ако покажем, че която и да е \NP-пълна задача не може да се реши за полиномиално време, то тогава $\mathbf{P \neq NP}$.
\end{itemize}
Всички \NP-пълни задачи са еквивалентни в смисъл, че всяка може да се сведе до другата за полиномиално време.
Така имайки полиномиално алгоритъм, който решава някоя \NP-пълна задача \textbf{X}, то тогава можем да решим всяка задача \textbf{Y} от \NP{} за полиномиално време:
\VerbatimInput[numbersep = 3pt, frame=single, numbers=left,commandchars=\\\{\},codes={\catcode`$=3}]{algorithms/solve-in-ptime.txt}

Обаче ние нито имаме такъв алгоритъм, нито не знаем, че такъв алгоритъм няма.
Това са задачи, които хем не можем да решим ефективно, хем не можем да покажем, че такова решение няма.
Най-доброто, което можем да направим, е да покажем, задачата е еквивалентна по трудност на други задачи, за които други много умни хора не са се сетили.

\section{\textit{``Основната''} \NP-пълнa задача}

По-принцип е трудно директно да се показва \NP-пълнота, ако се кара по дефиницията.
Това, което обикновено се прави, е се показва по дефиниция, че една задача е \NP-пълна (все трябва да започнем от някъде), след което се правят полиномиални редукции от задачи, за които знаем, че са \NP-пълни, към задачите, която искаме да покажем, че са \NP-пълни.
Тук се възползваме от транзитивността на $\leq_p$.
Разбира се, това само би показало \NP-трудност, трябва и да се провери принадлежност към класа \NP.
Задачата, от която обикновено се почва е тази за удовлетворимост/изпълнимост.

\begin{theorem}[Кук-Левин]
    Задачата \textbf{SAT} е \NP-пълна.
\end{theorem}

След това се показва, че $\mathbf{SAT} \leq_p \mathbf{3SAT}$.
Ние вече знаем, че тя е в класа \NP, така че ако направим тази редукция, ще излезе, че \textbf{3SAT} е \NP-пълна задача.
Можем да направим следната редукция -- за всеки дизюнкт $D$ във входната формула $\varphi$:
\begin{itemize}
    \item ако в $D$ участва точно един литерал $L$, то тогава избираме нови променливи $x$ и $y$, и заменяме $D$ с конюнкцията на дизюнктите $(L \lor x \lor y), (L \lor x \lor \overline{y}), (L \lor \overline{x} \lor y)$ и $(L \lor \overline{x} \lor \overline{y})$;
    \item ако в $D$ участват два литерала $L_1, L_2$, то тогава избираме нова променлива $x$, и заменяме $D$ с конюнкцията на дизюнктите $(L_1 \lor L_2 \lor x)$ и $(L_1 \lor L_2 \lor \overline{x})$;
    \item ако в $D$ участват три литерала, не променяме $D$;
    \item ако в $D$ участват литералите $L_1, \dots, L_n$, където $n > 3$, то тогава избираме нови променливи $x_1, \dots, x_{n - 3}$, и заменяме $D$ с конюнкцията на дизюнктите
          \[
              (L_1 \lor L_2 \lor x_1), (L_3 \lor \overline{x_1} \lor x_2), (L_4 \lor \overline{x_2} \lor x_3), \dots, (L_{n - 2} \lor \overline{x_{n - 4}} \lor x_{n - 3}), (L_{n - 1} \lor L_n \lor \overline{x_{n - 3}}).
          \]
\end{itemize}
Накрая ще получим като резултат формула $\psi$, която не е еквивалентна на $\varphi$, но е изпълнима т.с.т.к. $\varphi$ е изпълнима.
На читателя оставяме да провери, че това е вярно.
Остана само да проверим, че всичко това става за полиномиално време.
Дължината на $\psi$ ще бъде полиномиална спрямо тази на $\varphi$, защото:
\begin{itemize}
    \item дизюнктите с един литерал се заменят с четири дизюнкта с три литерала;
    \item дизюнктите с два литерала се заменят с два дизюнкта с три литерала;
    \item дизюнктите с три литерала не се променят;
    \item дизюнктите с $n > 4$ литерала се заменят с $n - 2$ дизюнкта с три литерала.
\end{itemize}

\newpage

По-съмнителното е търсенето на нови променливи, но и това няма как да е прекалено бавно, защото броят на променливите, които участват в една формула е по-малък или равен на дължината ѝ.
Тъй като ние ще получим формула с полиномиална на $\varphi$ дължина, в нея ще участват най-много полиномиално на $\varphi$ променливи.
Това означава, че ако всеки път търсим неизползваната променлива с най-малък индекс, няма да търсим прекалено дълго.
С това получаваме, че \textbf{3SAT} е \NP-трудна задача, и понеже сме показвали че е в класа \NP, то тя е и \NP-пълна.

\section{Класически \NP-пълни задачи}

Ще покажем няколко класически примери за \NP-пълни задачи.
Тъй като за тях сме доказвали, че са в класа \NP, ни остава само да покажем, че са \NP-трудни.
Нека сега покажем, че $\mathbf{3SAT} \leq_p \mathbf{Clique}$.
Нека $\varphi$ е формула в 3КНФ със $n$ на брой дизюнкта.
За полиномиално време ще построим граф $G$, за които $\varphi$ е изпълнима т.с.т.к. $G$ съдържа $n$-клика.
За всеки дизюнкт $(L_1(x) \lor L_2(y) \lor L_3(z))$, който участва във $\varphi$, в графа $G$ има върховете от вида $\{ \opair{x, v_x}, \opair{y, v_y}, \opair{z, v_z} \}$, където $v_x, v_y, v_z \in \{ \T, \F \}$ и интерпретирайки $t$ като $v_t$ за $t \in \{ x, y, z \}$, дизюнкта $(L_1(x) \lor L_2(y) \lor L_3(z))$ се оценява като $\T$.
Ребра ще има между тези множества, които не си противоречат т.е. няма променлива $x$, за която $\opair{x, \T}$ да участва в едното множество и $\opair{x, \F}$ да участва в другото.
По построение е очевидно, че в $G$ има $n$ клика т.с.т.к. $\varphi$ е изпълнима.
В едната посока кликата ни казва точно как да оценим променливите, а в другата посока от оценката можем да извлечем кликата.
Тъй като за всеки дизюнкт получаваме най-много $2^3$ върхове, то тогава конструкцията е полиномиална.
С това показахме, че \textbf{Clique} е \NP-трудна задача, откъдето е и \NP-пълна.

Сега нека да видим, че $\mathbf{Clique} \leq_p \mathbf{VertexCover}$.
За входния граф $G = \opair{V, E}$ строим $\overline{G} = \opair{V, \overline{E}}$, където:
\[
    \overline{E} = \{ (u, v) \mid u, v \in V \: \& \: (u, v) \notin E \}.
\]
Този граф очевидно се строи за време $\Theta(|V|^2)$.

Оказва се, че за всяко $X \subseteq V$ е изпълнено, че:
\begin{center}
    $X$ е клика в $G$ $\iff$ $V \setminus X$ е върхово покритие в $\overline{G}$.
\end{center}

\newpage

$(\Rightarrow)$
Нека $X$ е клика в $G$.
Тогава което и ребро $(u, v) \in \overline{E}$ да вземем, $u \notin X$ или $v \notin X$.
Ако $u, v \in X$, то тогава тъй като $(u, v) \in \overline{E}$, то тогава $(u, v) \notin E$, което противоречи с факта, че $X$ е клика.
Така наистина $V \setminus X$ е върхово покритие в $\overline{G}$.

$(\Leftarrow)$
Нека $X$ не е клика в $G$.
Тогава има два върха $u, v \in X$, за които $(u, v) \notin E$.
Но тогава $(u, v) \in \overline{E}$, откъдето $V \setminus X$ не е върхово покритие в $\overline{G}$.

Псевдокода на редукцията би изглеждал така:
\VerbatimInput[numbersep = 3pt, frame=single, numbers=left,commandchars=\\\{\},codes={\catcode`$=3}]{algorithms/clique-to-vertex-cover.txt}
С това показахме, че \textbf{VertexCover} е \NP-трудна задача, откъдето е и \NP-пълна.

Сега ще покажем, че $\mathbf{3SAT} \leq_p \mathbf{SubsetSum}$.
Нека е дадена формула $\varphi$ в 3КНФ, в която участват променливите $x_1, \dots, x_m$ и дизюнктите $D_1, \dots, D_l$.
Ще построим масив $A[1 \dots n]$ и число $k$, за които:
\begin{center}
    $\varphi$ е изпълнима $\iff$ има $S \subseteq \{ 1, \dots, n \}$, за което $k = \sum\limits_{t \in S} A[t]$.
\end{center}
За всяка променлива $x_i$ строим числа $t_i, f_i$ със $m + l$ цифри по следния начин:
\begin{itemize}
    \item $i$-тата цифра на $t_i$ и $f_i$ e $1$;
    \item за $m + 1 \leq j \leq m + l$, $j$-тата цифра на $t_i$ е $1$ ако $x_i$ участва в $D_{j - m}$;
    \item за $m + 1 \leq j \leq m + l$, $j$-тата цифра на $f_i$ е $1$ ако $\overline{x_i}$ участва в $D_{j - m}$;
    \item всички други цифри на $t_i$ и $f_i$ са $0$.
\end{itemize}
Очевидно всички тези числа могат да се построят за полиномиално на дължината на $\varphi$ време.

Сега за всеки дизюнкт $D_j$ строим числа $h_j$ и $g_j$ със $m + l$ цифри по следния начин:
\begin{itemize}
    \item $(m + j)$-тата цифра на $h_j$ и $g_j$ е $1$;
    \item всички други цифри на $h_j$ и $g_j$ са $0$.
\end{itemize}

\newpage
Очевидно всички тези числа могат да се построят за полиномиално на дължината на $\varphi$ време.

Накрая нека $k = \underbrace{1 \dots 1}_{m \text{ пъти}} \underbrace{3 \dots 3}_{l \text{ пъти}}$.
Това число също се строи за полиномиално на дължината на $\varphi$ време.

Накрая получаваме вход $[t_1, f_1, \dots t_m, f_m, h_1, g_1, \dots, h_l, g_l]$ и $k$ за \textbf{SubsetSum}.

Нека $\varphi$ е изпълнима т.е. е вярна при оценка $v$.
Тогава за $1 \leq i \leq m$, избираме $t_i$ ако $v(x_i) = \T$, иначе избираме $f_i$.
Тъй като тези числа не се бият в първите $m$ цифри, те ще бъдат точно като в $k$, стига да не сумираме прекалено големи цифри от дясната част.
За останалите $l$ цифри на $k$, можем да забележим, че в една формула ще са верни между $1$ и $3$ променливи.
Тогава сумирайки избраните числа за всяко $m + 1 \leq j \leq m + l$ знаем, че $j$-тата цифра ще е между $1$ и $3$.
Така можем просто да изберем достатъчно числа от $h_j$ и $g_j$, че да добутаме до $3$.

Обратно, нека имаме $I \subseteq \{ 1, \dots, 2m + 2l \}$, за което $k = \sum\limits_{i \in I} A[i]$.
От структурата на числото $k$ лесно се вижда, че за всяко $1 \leq i \leq m$ точно едно от $t_i$ и $f_i$ участва в $\{ A[i] \mid i \in I \}$.
Нека $v(x_i) = \T$ при $t_i \in S$ и $v(x_i) = \F$ иначе.
Ако има такъв дизюнкт $D_j$, който при оценката $v$ се остойностява като $\F$, то тогава $(m + j)$-тата цифра на $k$ щеше да е строго по-малка от $3$.
Следователно всички дизюнкти във $\varphi$ са верни при оценка $v$, откъдето $\varphi$ е изпълнима.

Псевдокода на редукцията би изглеждал така:
\VerbatimInput[numbersep = 3pt, frame=single, numbers=left,commandchars=\\\{\},codes={\catcode`$=3\catcode`_=8}]{algorithms/3sat-to-subset-sum.txt}
С това показахме, че \textbf{SubsetSum} е \NP-трудна задача, откъдето е и \NP-пълна.