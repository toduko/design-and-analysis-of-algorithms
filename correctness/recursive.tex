\section{Примери за рекурсивни алгоритми}

Нека разгледаме следният алгоритъм:
\lstinputlisting{algorithms/maximum.txt}

Очевидно при параметър целочислен масив $A[1 \dots n]$, функцията $\mathfrak{M}(A[1 \dots n])$ връща $\max A[1 \dots n]$.
Ще докажем това с индукция по $n$:
\begin{itemize}
    \item В базата имаме, че $\mathfrak{M}(A[1 \dots n]) = -\infty = \max [] = \max A[1 \dots 0]$ където със $[]$ означаваме празният масив.
    \item За индуктивната стъпка имаме, че:
          \begin{align*}
              \mathfrak{M}(A[1 \dots n + 1]) & = \max(A[n + 1], \mathfrak{M}(A[1 \dots n])) \stackrel{\text{(ИП)}}{=} \max(A[n + 1], \max A[1 \dots n]) \\
                                             & = \max A[1 \dots n + 1].
          \end{align*}
\end{itemize}

Тук управляващият параметър на рекурсията $n$ винаги намалява с $1$, докато не стигне $0$, където ще приключи алгоритъмът.
В по нататъчните разсъждения ще смятаме завършването на алгоритъма за очевидно.

Сложността на алгоритъма се описва със рекурентното уравнение:
\[
    T(n) = T(n - 1) + 1 \text{ // базата няма да я пишем}.
\]

Директно се вижда, че $T(n) = \sum\limits_{i = 0}^n 1 = n + 1 \asymp n$.

Да видим един малко по-сложен пример -- за бързо степенуване:
\lstinputlisting{algorithms/exp-rec.txt}

С пълна индукция относно $y$ ще покажем, че $\mathcal{P}(x, y) = x^y$:
\begin{itemize}
    \item $\mathcal{P}(x, 0) = 1 = x^0 = 1$ // тук се уговаряме, че $0^0 = 1$.
    \item $\mathcal{P}(x, 2y + 1) = x \cdot \mathcal{P}(x, y) \cdot \mathcal{P}(x, y) \stackrel{\text{(ИП)}}{=} x \cdot x^y \cdot x^y = x^{2y + 1}$.
    \item $\mathcal{P}(x, 2y + 2) = \mathcal{P}(x, y + 1) \cdot \mathcal{P}(x, y + 1) \stackrel{\text{(ИП)}}{=} x^{y + 1} \cdot x^{y + 1} = x^{2y + 2}$.
\end{itemize}

Сложността на този алгоритъм може да се опише със следното рекурентно уравнение:
\[
    T(n) = T(\frac{n}{2}) + 1.
\]
От мастър-теоремата следва, че:
\[
    T(n) \asymp \log(n).
\]

\section{Трик за бързо пресмятане на членове на някои рекурентни редици}

Нека вземем за пример редицата на Фибоначи:
\begin{align*}
     & F(0) = 0                   \\
     & F(1) = 1                   \\
     & F(n + 2) = F(n + 1) + F(n)
\end{align*}
Човек може да забележи, че:
\[
    \underbrace{\begin{pmatrix}
            1 & 1 \\
            1 & 0
        \end{pmatrix}}_{\mathfrak{F}^*}
    \cdot
    \begin{pmatrix}
        F(n + 1) \\
        F(n)
    \end{pmatrix}
    =
    \begin{pmatrix}
        F(n + 2) \\
        F(n + 1)
    \end{pmatrix}.
\]
След това с индукция лесно се показва, че:
\[
    (\mathfrak{F}^*)^n
    \cdot
    \begin{pmatrix}
        F(1) \\
        F(0)
    \end{pmatrix}
    =
    \begin{pmatrix}
        F(n + 1) \\
        F(n)
    \end{pmatrix}.
\]

\newpage

За да сметнем $F(n)$, можем да направим бързо степенуване на матрицата, аналогична на тази с числата:
\lstinputlisting{algorithms/fibonacci-fast.txt}

Коректността и сложността на алгоритъма оставяме на читателя (напълно аналогично е на предния алгоритъм).

В общия случай ще имаме рекурентно уравнение от вида:
\[
    T(n + k + 1) = a_k T(n + k) + \dots + a_1 T(n + 1) + a_0 T(n) \text{, където } a_0, \dots, a_k, k \text{ са константи}.
\]

Тогава отново с индукция човек лесно може да покаже, че:
\[
    \begin{pmatrix}
        a_k    & a_{k - 1} & a_{k - 2} & \dots  & a_2    & a_1    & a_0    \\
        1      & 0         & 0         & \dots  & 0      & 0      & 0      \\
        0      & 1         & 0         & \dots  & 0      & 0      & 0      \\
        \vdots & \vdots    & \vdots    & \ddots & \vdots & \vdots & \vdots \\
        0      & 0         & 0         & \dots  & 0      & 1      & 0      \\
    \end{pmatrix}^n
    \cdot
    \begin{pmatrix}
        T(k)     \\
        T(k - 1) \\
        T(k - 2) \\
        \vdots   \\
        T(0)
    \end{pmatrix}
    =
    \begin{pmatrix}
        T(n + k)     \\
        T(n + k - 1) \\
        T(n + k - 2) \\
        \vdots       \\
        T(n)
    \end{pmatrix}.
\]

\newpage