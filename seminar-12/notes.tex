\documentclass{article}

\usepackage[utf8]{inputenc}
\usepackage[T2A]{fontenc}
\usepackage[english, bulgarian]{babel}
\usepackage{pgfplots}
\usepackage{amssymb}
\usepackage{hyperref, fancyhdr, lastpage, fancyvrb, tcolorbox, titlesec}
\usepackage{array, tabularx, colortbl}
\usepackage{tikz}
\usepackage{venndiagram}
\usepackage{amsthm, bm}
\usepackage{relsize}
\usepackage{amsmath,physics}
\usepackage{mathtools}
\usepackage{subcaption}
\usepackage{theoremref}
\usepackage{circuitikz}
\usepackage[a4paper, left=0.50in, right=0.50in, top=0.5in, bottom=1.0in]{geometry}
\usepackage{stmaryrd}
\usepackage[symbol]{footmisc}
\usepackage{minted}
\usepackage{enumitem}
\usepackage{systeme}
\usepackage{forest}
\useforestlibrary{linguistics}

\pgfplotsset{width=10cm,compat=1.9}

\newcommand{\N}{\mathbb{N}}
\newcommand{\R}{\mathbb{R}}
\newcommand{\T}{\mathbb{T}}
\newcommand{\F}{\mathbb{F}}

\ExplSyntaxOn
\NewDocumentCommand{\opair}{m}
{
  \langle\mspace{2mu}
  \clist_set:Nn \l_tmpa_clist { #1 }
  \clist_use:Nn \l_tmpa_clist {,\mspace{3mu plus 1mu minus 1mu}\allowbreak}
  \mspace{2mu}\rangle
}
\ExplSyntaxOff

\hypersetup{
  colorlinks=true,
  linktoc=all,
  linkcolor=blue
}

\newcommand{\dc}[1]{\tikz \node[draw, circle, inner sep=0pt, minimum size=9mm]{#1};}
\newcommand{\dt}[1]{\tikz \node[circle, inner sep=0pt, minimum size=9mm]{#1};}

\theoremstyle{definition}
\newtheorem{definition}{Дефиниция}[section]
\newtheorem*{warning}{\textcolor{red}{Внимание}}
\theoremstyle{plain}
\newtheorem*{theorem}{Теорема}
\newtheorem*{invariant}{Инвариант}
\newtheorem{claim}[definition]{Твърдение}
\newtheorem{axiom}[definition]{Аксиома}
\newtheorem{lemma}[definition]{Лема}
\newtheorem{corollary}[definition]{Следствие}
\theoremstyle{remark}
\newtheorem*{remark}{Забележка}
\newtheorem{problem}{Задача}
\newtheorem*{solution}{Решение}
\theoremstyle{definition}

\setlength\parindent{0pt}

\title{Още задачи върху динамично програмиране}
\author{Тодор Дуков}
\date{}

\begin{document}
\maketitle

\section*{Два интересни примера}

За първия пример, нека имаме някакъв краен речник $D$ от непразни думи над $\Sigma = \{ a, \dots, z \}$.
Ще искаме при подаден низ над $\Sigma$ да видим дали той може да се разбие на думи от речника (с възможни повторения).
Нека например да вземем речника $D = \{ \text{mango}, \text{i}, \text{icecream}, \text{like}, \text{with} \}$.
Тогава думата ``ilikeicecreamwithmango'' може да се разбие на ``i like icecream with mango''.

Нека е подаден един низ $\alpha \in \Sigma^*$.
Ако $\alpha = \varepsilon$, то тогава очевидно можем да получим $\alpha$ чрез конкатенация на думи от $D$.
Ако $\alpha \neq \varepsilon$ и $\alpha$ се получава чрез конкатенация на думи от $D$, то тогава има $\beta \in \Sigma^*$ и $\gamma \in D$, за които е изпълнено, че $\alpha = \beta \gamma$ и $\beta$ може да се получи чрез думи от $D$.
Но $\gamma \neq \varepsilon$, т.е. успяхме да сведем задачата за $\alpha$ до задача за $\beta$, като $|\beta| < |\alpha|$.
Нека сега да формализираме тези разсъждения.
Нека $\alpha = \alpha_1 \dots \alpha_n$, където $\alpha_i \in \Sigma$.
Тогава булевата функция $\operatorname{WB}_{D, \alpha}(i)$, която казва дали $\alpha_1 \dots \alpha_i$ може да се разбие на думи от $D$, изглежда така:
\[
  \operatorname{WB}_{D, \alpha}(i) = \begin{cases}
    \T                                                                                                                                 & \text{, ако } i = 0 \\
    \bigvee\limits_{\beta \in D} (\alpha_{i - |\beta| + 1} \dots \alpha_i = \beta \: \& \: \operatorname{WB}_{D, \alpha}(i - |\beta|)) & \text{, иначе}
  \end{cases}
\]

На нас това, което ни трябва, е $\operatorname{WB}_{D, \alpha}(|\alpha|)$.
С малко мислене върху това как се пресмята $\operatorname{WB}_{D, \alpha}$, човек може да стигне до следното итеративно решение:
\inputminted[linenos]{c++}{algorithms/word_break.cpp}
Ако $|\alpha| = n, |D| = m$, и $\max \{ |\beta| \mid \beta \in D \} = k$, то това решение има сложност по време $O(n \cdot m \cdot k)$, понеже за всяко $1 \leq i \leq n$ и за всяка дума $\beta \in D$ (те са $m$ на брой) проверяваме дали $\alpha_{i - |\beta| + 1} \dots \alpha_i = \beta$, което става за време $O(k)$, и дали $\operatorname{WB}_{D, \alpha}(i - |\beta|) = \T$, което поради наличието на масива $\mathtt{wb}$ става за константно време.
Този масив доста помага, ако не го бяхме ползвали, сложността щеше да се опише (горе долу) с рекурентното уравнение:
\[
  T(n) = \sum\limits_{i = 1}^{n} (m \cdot k \cdot T(n - i)) + \Theta(1).
\]
Сложността по памет очевидно е $\Theta(n)$, заради допълнителния масив, който заделяме.

\pagebreak

Вторият пример ще бъде под формата на игра.
Наредени са $n$ монети със стойности съответно $v_1, \dots, v_n$.
Редуваме се с опонент да теглим една монета от избран от краищата на редицата, докато монетите не свършат.
Накрая всеки човек печели толкова, колкото е изтеглил.
В случай че играем първи, каква печалба можем да си гарантираме?

Първо да започнем с двата най-прости типа игри -- с една или две монети.
В играта с една монета е ясно, че най-голямата печалба, която можем да си гарантираме, е стойността на монетата.
В игра с две монети със стойности съответно $v_1, v_2$, най-голямата гарантирана печалба е очевидно $\max \{ v_1, v_2 \}$.

Нека сега в общия случай имаме следната партия:
\begin{center}
  \dc{$v_1$} \dc{$v_2$} \dt{$\dots$} \dc{$v_{n - 1}$} \dc{$v_n$}
\end{center}
Ще се опитаме да сведем тази по-сложна партия, до няколко по-прости.
Имаме две възможности:
\begin{enumerate}
  \item Ако изберем първата монета, опонента ще трябва да избира в конфигурацията:
        \begin{center}
          \dc{$v_2$} \dc{$v_3$} \dt{$\dots$} \dc{$v_{n - 1}$} \dc{$v_n$}
        \end{center}
        Той също има две възможности:
        \begin{enumerate}
          \item Ако опонента избере първата монета, ни оставя в конфигурацията:
                \begin{center}
                  \dc{$v_3$} \dc{$v_4$} \dt{$\dots$} \dc{$v_{n - 1}$} \dc{$v_n$}
                \end{center}
                Тук можем да си мислим, че ние сме си заделили на страна печалба $v_1$, опонента си е заделил на страна печалба $v_2$, и играта започва наново в горната конфигурация.
          \item Ако пък избере последната монета, ни оставя в конфигурацията:
                \begin{center}
                  \dc{$v_2$} \dc{$v_3$} \dt{$\dots$} \dc{$v_{n - 2}$} \dc{$v_{n - 1}$}
                \end{center}
                Тук можем да си мислим, че ние сме си заделили на страна печалба $v_1$, опонента си е заделил на страна печалба $v_n$, и играта започва наново в горната конфигурация.
        \end{enumerate}
  \item Ако изберем последната монета, опонента ще трябва да избира в конфигурацията:
        \begin{center}
          \dc{$v_1$} \dc{$v_2$} \dt{$\dots$} \dc{$v_{n - 2}$} \dc{$v_{n - 1}$}
        \end{center}
        Той също има две възможности:
        \begin{enumerate}
          \item Ако опонента избере първата монета, ни оставя в конфигурацията:
                \begin{center}
                  \dc{$v_2$} \dc{$v_3$} \dt{$\dots$} \dc{$v_{n - 2}$} \dc{$v_{n - 1}$}
                \end{center}
                Тук можем да си мислим, че ние сме си заделили на страна печалба $v_n$, опонента си е заделил на страна печалба $v_1$, и играта започва наново в горната конфигурация.
          \item Ако пък избере последната монета, ни оставя в конфигурацията:
                \begin{center}
                  \dc{$v_1$} \dc{$v_2$} \dt{$\dots$} \dc{$v_{n - 3}$} \dc{$v_{n - 2}$}
                \end{center}
                Тук можем да си мислим, че ние сме си заделили на страна печалба $v_n$, опонента си е заделил на страна печалба $v_{n - 1}$, и играта започва наново в горната конфигурация.
        \end{enumerate}
\end{enumerate}

Нека $\operatorname{MP}(i, j)$ е максималната печалба, която може да се спечели от първия играч (в първоначалната игра), ако може да събира монети със стойности съответно $v_i, \dots, v_j$.
По предните разсъждения можем да пресметнем тази печалба рекурсивно така:
\[
  \operatorname{MP}(i, j) = \begin{cases}
    v_i                                                                                                                                                                      & \text{, ако } i = j     \\
    \max \{ v_i, v_j\}                                                                                                                                                       & \text{, ако } i = j + 1 \\
    \max \{ v_i + \min \{ \operatorname{MP}(i + 2, j) , \operatorname{MP}(i + 1, j - 1) \}, v_j + \min \{ \operatorname{MP}(i + 1, j - 1) , \operatorname{MP}(i, j - 2) \}\} & \text{, иначе}
  \end{cases}
\]
Във хода на втория играч минимизираме, защото той печели възможно най-много, когато ние печелим възможно най-малко.
За задача на читателя оставяме да пресметне $\operatorname{MP}(1, n)$ итеративно.

\section*{Задачи}

\begin{problem}
Формула ще наричаме всеки низ от вида $B_0 \sigma_1 B_1 \sigma_2 B_2 \dots B_{n - 1} \sigma_n B_n$, където $B_i \in \{ \T, \F \}$ и $\sigma_i \in \{ \lor, \land, \oplus \}$.
Например низът $\T \land \T \oplus \F \lor \T$ е формула.
В зависимост от това как слагаме скобите, оценката на израза може да е различна.
Например изразът $(\T \land (\T \oplus (\F \lor \T)))$ се остойностява като $\F$, докато изразът $(\T \land ((\T \oplus \F) \lor \T)))$ се остойностява като $\T$, въпреки че и двата израза се получават от примерната формула.
Да се напише колкото се може по-бърз алгоритъм, който при подадена формула да върне броят на различни скобувания, за които съответния израз се остойностява като $\T$.
След това да се докаже неговата коректност, и да се изследва сложността му по време.
\end{problem}

\begin{problem}
Имаме професионален крадец, който иска да ограби къщите в дадена улица.
Проблемът е, че ако той ограби две съседни къщи, алармата ще се активира и полицията ще дойде.
Той не иска това, защото в такъв случай няма да спечели нищо.
Да се напише колкото се може по-бърз алгоритъм, който при подаден масив от естествени числа $L[1 \dots n]$, където $L[i]$ е печалбата от къща $i$, връща максималната печалба на крадеца.
След това да се докаже неговата коректност, и да се изследва сложността му по време.
\end{problem}

\begin{problem}
Да се напише колкото се може по-бърз алгоритъм, който при подадена булева матрица $M[1 \dots n, 1 \dots m]$ намира най-голямото квадратче в $M$, съставено от $1$, и връща неговото лице.
След това да се докаже неговата коректност, и да се изследва сложността му по време.
\end{problem}

\begin{problem}
Един целочислен масив $A[1 \dots n]$ наричаме аритметичен, ако $n \geq 3$ и за всяко $1 \leq i \leq n - 2$ е изпълнено, че $A[i + 2] - A[i + 1] = A[i + 1] - A[i]$.
Да се напише колкото се може по-бърз алгоритъм, който при подаден целочислен масив $A[1 \dots n]$ намира броя на подредиците на $A$, които са аритметични масиви.
След това да се докаже неговата коректност, и да се изследва сложността му по време.
\end{problem}

\begin{problem}
Имаме $n$ на брой къщи, които искаме да боядисаме със цветове $c_1, c_2, c_3$, като не може две съседни къщи да имат еднакъв цвят.
Масив на цените ще наричаме всеки двумерен масив от положителни числа $P[1 \dots n, 1 \dots 3]$, където $P[i, j]$ ще бъде цената за боядисване на къща $i$ със цвят $c_j$.
Да се напише колкото се може по-бърз алгоритъм, който при подаден двумерен масив от положителни числа, намира минималната цена за боядисване на всички къщи.
След това да се докаже неговата коректност, и да се изследва сложността му по време.
\end{problem}

\begin{problem}
Върху един масив от естествени числа $A[1 \dots n]$ можем да прилагаме следните две операции:
\begin{itemize}
  \item да увеличим $A[i]$ с единица за някое $1 \leq i \leq n$;
  \item да намалим $A[i]$ с единица за някое $1 \leq i \leq n$.
\end{itemize}
Да се напише колкото се може по-бърз алгоритъм, който при подаден масив от естествени числа $A[1 \dots n]$ връща минималния брой операции, които са нужни, за да стане масива монотонно ненамаляващ.
След това да се докаже неговата коректност, и да се изследва сложността му по време.
\end{problem}

\begin{problem}
Нека $n, l, r \in \N$ и $l \leq r$.
Един масив $A[1 \dots n]$ ще наричаме $(n, l, r)$-интересен, ако:
\begin{itemize}
  \item $l \leq A[i] \leq r$ за всяко $1 \leq i \leq n$;
  \item $\sum\limits_{i = 1}^n A[i] \equiv 0 \; (\operatorname{mod} 3)$.
\end{itemize}
Да се напише колкото се може по-бърз алгоритъм, който при подадени $n, l, r \in \N$, за които $l \leq r$, връща броя на $(n, l, r)$-интересни масиви.
След това да се докаже неговата коректност, и да се изследва сложността му по време.
\end{problem}

\end{document}
