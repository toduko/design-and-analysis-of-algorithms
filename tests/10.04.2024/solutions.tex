\documentclass{article}
\usepackage[utf8]{inputenc}
\usepackage[T2A]{fontenc}
\usepackage[english, bulgarian]{babel}
\usepackage{amssymb}
\usepackage{hyperref, fancyhdr, lastpage, fancyvrb, tcolorbox, titlesec}
\usepackage{array, tabularx, colortbl}
\usepackage{tikz}
\usepackage{venndiagram}
\usepackage{amsthm, bm}
\usepackage{relsize}
\usepackage{amsmath,physics}
\usepackage{mathtools}
\usepackage{subcaption}
\usepackage{theoremref}
\usepackage{circuitikz}
\usepackage[a5paper, left=0.50in, right=0.50in, top=0.50in, bottom=0.50in]{geometry}
\usepackage{stmaryrd}
\usepackage{forest}
\usepackage{cancel}
\usepackage{minted}
\usepackage{titling}
\usetikzlibrary{automata, arrows, positioning, shapes}
\useforestlibrary{linguistics}
\pagenumbering{gobble}
\ExplSyntaxOn
\NewDocumentCommand{\opair}{m}
 {
  \langle\mspace{2mu}
  \clist_set:Nn \l_tmpa_clist { #1 }
  \clist_use:Nn \l_tmpa_clist {,\mspace{3mu plus 1mu minus 1mu}\allowbreak}
  \mspace{2mu}\rangle
}
\ExplSyntaxOff

\hypersetup{
    colorlinks=true,
    linktoc=all,
    linkcolor=blue
}

\setlength\parindent{0pt}

\newtheorem{problem}{Задача}
\theoremstyle{definition}
\newtheorem*{solution}{Решение}

\begin{document}

\begin{center}
    \Large{Решения на задачите от първото малко контролно по ДАА на група 5, проведено на 10.04.2024 г.}
\end{center}

\vspace*{4mm}

\begin{problem} {\bf (50 т.)}
Дефинираме \textit{вълнист масив} индуктивно:
\begin{itemize}
    \item Всеки едноелементен масив е вълнист.
    \item Масив $A[1 \dots n]$ (където $n > 1$) е вълнист, ако
          \begin{enumerate}
              \item масивът $A[1 \dots \lfloor \frac{n}{2} \rfloor]$ е сортиран, а
              \item масивът $A[\lfloor \frac{n}{2} \rfloor + 1 \dots n]$ е вълнист.
          \end{enumerate}
\end{itemize}
Да се състави алгоритъм с линейна сложност по време, който приема вълнист масив и го сортира.
Коректността на алгоритъма да се обоснове формално и да се изследва сложността по време.
\end{problem}

\begin{solution}
    Решението е проста модификация на алгоритъма за сортиране чрез сливане:
    \begin{Verbatim}[frame=single, numbers=left,commandchars=\\\{\},codes={\catcode`$=3}]
Sort(A[1..n] - вълнист масив):
    if n = 1:
        return

    Sort(A[$\mathtt{\lfloor\frac{n}{2}\rfloor}$ + 1..n])
    Merge(A[1..$\mathtt{\lfloor\frac{n}{2}\rfloor}$], A[$\mathtt{\lfloor\frac{n}{2}\rfloor}$ + 1..n])
    \end{Verbatim}
    Тук $\mathtt{Merge}$ е алгоритъмът за сливане на сортирани масиви, който е изучаван на лекции.
    Ще покажем коректността на $\mathtt{Sort}$ с пълна индукция по $\mathtt{n}$:
    \begin{itemize}
        \item ако $\mathtt{n = 1}$, то тогава $\mathtt{A[1..n]}$ е сортиран и алгоритъмът коректно веднага ще приключи работа;
        \item ако $\mathtt{n > 1}$, то тогава на ред $5$ по ИП ще сортираме вълнистия масив $\mathtt{A[\lfloor\frac{n}{2}\rfloor + 1..n]}$ и с извикването на ред $6$ ще слеем масивите $\mathtt{A[1..\lfloor\frac{n}{2}\rfloor]}$ и $\mathtt{A[\lfloor\frac{n}{2}\rfloor + 1..n]}$, което ще сортира $\mathtt{A[1..n]}$.
    \end{itemize}
    Времевата сложност се описва със следното рекурентно уравнение:
    \begin{center}
        $T(n) = T(\frac{n}{2}) + \Theta(n)$ (заради извикванията $\mathtt{Sort}$ и $\mathtt{Merge}$).
    \end{center}
    По мастър-теоремата излиза, че $T(n) \asymp n$.
\end{solution}

\textbf{Критерии за оценяване:}
\begin{itemize}
    \item за правилен алгоритъм -- $20$ точки;
    \item за доказателство на коректността на алгоритъма -- $20$ точки;
    \item за обосновка на сложността по време -- $10$ точки.
\end{itemize}

\pagebreak

\begin{problem} {\bf (70 т.)}
Подредете по асимптотично нарастване следните \\ функции:
\begin{align*}
    f_1(n) & = n!             & f_2(n) & = \sum\limits_{k = 1}^{n} \ln(k)         & f_3(n) & = \sum\limits_{k = 1}^{n^3} \frac{1}{k} & f_4(n) & = (\ln(n))^{\ln(n)} \\
    f_5(n) & = 3^{n \sqrt{n}} & f_6(n) & = \sum\limits_{k = 1}^{n!} \frac{1}{k^2} & f_7(n) & = \ln(\ln(n))                           & f_8(n) & = 2^{n^2}.
\end{align*}
При всяко сравнение на две функции се обосновете формално.
\end{problem}

\begin{solution}
    Окончателната наредба е следната:
    \[
        f_6(n) \stackrel{(1)}{\prec} f_7(n) \stackrel{(2)}{\prec} f_3(n) \stackrel{(3)}{\prec} f_2(n) \stackrel{(4)}{\prec} f_4(n) \stackrel{(5)}{\prec} f_1(n) \stackrel{(6)}{\prec} f_5(n) \stackrel{(7)}{\prec} f_8(n).
    \]
    Доказателство:
    \begin{itemize}
        \item[$(1)$] $f_7(n) = \ln(\ln(n)) \succ 1 \asymp \sum\limits_{k = 1}^{n!} \frac{1}{k^2} = f_6(n)$.
        \item[$(2)$] $f_3(n) = \sum\limits_{k = 1}^{n^3} \frac{1}{k} \asymp \ln(n^3) = 3 \ln(n) \asymp \ln(n) \succ f_7(n)$.
        \item[$(3)$] $f_2(n) = \sum\limits_{k = 1}^{n} \ln(k) = \ln(n!) \asymp n \ln(n) \succ \ln(n) \asymp f_3(n)$.
        \item[$(4)$] $f_4(n) = (\ln(n))^{\ln(n)} = n^{\ln(\ln(n))} \succ n \ln(n) \asymp f_2(n)$.
        \item[$(5)$] $\ln(f_1(n)) \asymp n \ln(n) \succ \ln(n) \ln(\ln(n)) \asymp \ln(f_4(n))$, откъдето $f_1(n) \succ f_4(n)$.
        \item[$(6)$] $\ln(f_5(n)) \asymp n \sqrt{n} \succ n \ln(n) \asymp \ln(f_1(n))$, откъдето $f_5(n) \succ f_1(n)$.
        \item[$(7)$] $\ln(f_8(n)) \asymp n^2 \succ n \sqrt{n} \asymp \ln(f_5(n))$, откъдето $f_8(n) \succ f_5(n)$.
    \end{itemize}
\end{solution}

\textbf{Критерии за оценяване:}
\begin{itemize}
    \item за всяко правилно сравнение от $(1)$ до $(7)$ -- по 5 точки;
    \item за обосновка на всяко сравнение от $(1)$ до $(7)$ -- по 5 точки.
\end{itemize}

\end{document}