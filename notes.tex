\documentclass[oneside, 12pt]{book}

\usepackage[utf8]{inputenc}
\usepackage[T2A]{fontenc}
\usepackage[english, bulgarian]{babel}
\usepackage{pgfplots}
\usepackage{amssymb}
\usepackage{hyperref, fancyhdr, lastpage, fancyvrb, tcolorbox, titlesec}
\usepackage{array, tabularx, colortbl}
\usepackage{tikz}
\usepackage{venndiagram}
\usepackage{amsthm, bm}
\usepackage{relsize}
\usepackage{amsmath,physics}
\usepackage{mathtools}
\usepackage{subcaption}
\usepackage{theoremref}
\usepackage{circuitikz}
\usepackage{geometry}
\usepackage{stmaryrd}
\usepackage[symbol]{footmisc}
\usepackage{minted}
\usepackage{enumitem}
\usepackage{listings}
\usepackage{systeme}
\usepackage{forest}
\useforestlibrary{linguistics}

\pgfplotsset{width=10cm,compat=1.9}

\newcommand{\N}{\mathbb{N}}
\newcommand{\Z}{\mathbb{Z}}
\newcommand{\R}{\mathbb{R}}
\newcommand{\T}{\mathbb{T}}
\newcommand{\F}{\mathbb{F}}
\newcommand{\calF}{\mathcal{F}}
\newcommand{\calT}{\mathcal{T}}
\newcommand{\NP}{\textbf{NP}}
\newcommand{\arr}{\operatorname{array}(\mathbb{Z})}


\ExplSyntaxOn
\NewDocumentCommand{\opair}{m}
 {
  \langle\mspace{2mu}
  \clist_set:Nn \l_tmpa_clist { #1 }
  \clist_use:Nn \l_tmpa_clist {,\mspace{3mu plus 1mu minus 1mu}\allowbreak}
  \mspace{2mu}\rangle
}
\ExplSyntaxOff

\hypersetup{
    colorlinks=true,
    linktoc=all,
    linkcolor=blue
}

\lstset{basicstyle=\ttfamily,
        breaklines=true,mathescape=true,numbers=left,
        inputencoding=utf8,extendedchars=true,frame=single}

\theoremstyle{definition}
\newtheorem*{definition}{Дефиниция}
\newtheorem*{warning}{\textcolor{red}{Внимание}}
\theoremstyle{plain}
\newtheorem{theorem}{Теорема}[section]
\newtheorem{invariant}[theorem]{Инвариант}
\newtheorem{claim}[theorem]{Твърдение}
\newtheorem{axiom}[theorem]{Аксиома}
\newtheorem{lemma}[theorem]{Лема}
\newtheorem{corollary}[theorem]{Следствие}
\theoremstyle{remark}
\newtheorem*{remark}{Забележка}
\newtheorem{problem}{Задача}[chapter]
\newtheorem*{solution}{Решение}
\theoremstyle{definition}

\pagestyle{fancy}

\lhead{\leftmark}
\rhead{}

\setlength\parindent{0pt}

\begin{document}

\begin{titlepage}
    \title{Записки за упражнения по ДАА}
    \author{Тодор Дуков}
    \date{\today}
\end{titlepage}

\maketitle

\tableofcontents

\chapter{Въведение в алгоритмите и асимптотичния анализ}

\section{Що е то алгоритъм?}

Алгоритмите се срещат навсякъде около нас:
\begin{itemize}
  \item рецептите са алгоритми за готвене;
  \item сутрешното приготвяне;
  \item придвижването от точка A до точка B;
  \item търсенето на книга в библиотеката.
\end{itemize}

Въпреки това е трудно да се даде формална дефиниция на това какво точно е алгоритъм.
На ниво интуиция, човек може да си мисли, че това просто е някакъв последователен списък от стъпки/инструкции, които човек/машина трябва да изпълни.
Други начини човек да си мисли за алгоритмите, са:
\begin{itemize}
  \item програми -- обикновено така се реализират алгоритми;
  \item машини на Тюринг, крайни (стекови) автомати или формални граматики;
  \item частично рекурсивни функции.
\end{itemize}

Един програмист в ежедневието си постоянно пише алгоритми за да решава различни задачи/проблеми.
Една задача може да се решава по много начини, някои по-добри от други.
Добрият програмист, освен че ще намери решение на проблема, той ще намери най-доброто решение (или поне достатъчно добро за неговите цели).

\section{Какво означава добро решение?}

Хубаво е човек да се води по следните (неизчерпателни) критерии:
\begin{itemize}
  \item решението трябва да е коректно -- ако алгоритъмът работи само през 50\% от времето, най-вероятно можем да се справим по-добре;
  \item решението трябва да е бързо -- ако алгоритъмът ще завърши работа след като всички звезди са измрели, то той практически не ни върши работа;
  \item решението трябва да заема малко памет -- ако алгоритъмът по време на своята работа се нуждае от повече памет, колкото компютърът може да предостави, за нас този алгоритъм е безполезен;
  \item решението трябва да е просто -- това е може би най-маловажният критерии от тези, но въпреки това е хубаво когато човек може, да пише чист и разбираем код, който лесно се разширява.
\end{itemize}

За да можем да сравняваме алгоритми в зависимост от това колко големи ресурси (време и памет) използват, трябва първо да можем да ``измерваме'' тези ресурси.

\section{Как мерим времето и паметта?}

Когато пишем алгоритми, имаме няколко базови инструкции (за които предварително сме се уговорили), които ще наричаме \textbf{атомарни инструкции}.
Тяхното извикване ще отнеме една единица време.
\textbf{Време за изпълнение} ще наричаме броят на извикванията на атомарните инструкции по време на изпълнение на програмата.
Също така числата и символите ще бъдат нашите \textbf{атомарни типове данни}, и ще заемат една единица памет.
\textbf{Паметта}, която една програма заема, ще наричаме максималния брой на единици от атомарни типове данни по време на изпълнение, без да броим входните данни.
Обикновено времето и паметта зависят от размера на подадените входни данни.
Това означава, че можем да си мислим за времето и паметта като функции на размера на входа.
Подходът, който ще изберем, е да сравняваме функциите за време/памет на различните алгоритми асимптотично.
Интересуваме се не толкова от конкретните стойности, а от поведението им, когато размерът на входа клони към безкрайност.

\section{Основни дефиниции}
\footnotetext[1]{точност до константен множител и константно събираемо}
Множеството от функции, които ще анализираме, е
\[
  \calF = \{ f \mid f : \R^{\geq 0} \rightarrow \R \: \& \: (\exists n_0 > 0) (\forall n \geq n_0) (f(n) > 0) \}.
\]


\begin{definition}
  За всяка функция $f \in \calF$ дефинираме:
  \begin{align*}
    \Theta(f) = \{ g \in \calF \: \mid & (\exists c_1 > 0)(\exists c_2 > 0)                                                        \\
                                       & (\exists n_0 \in \N)(\forall n \geq n_0)(c_1 \cdot f(n) \leq g(n) \leq c_2 \cdot f(n))\}.
  \end{align*}

\end{definition}
Може да тълкуваме $\Theta(f)$ като:
\begin{center}
  \textit{``множеството от функциите, които растат\footnotemark[1] със скоростта на $f$''.}
\end{center}


Нека вземем за пример $f(n) = 3n + 1$ и $g(n) = n + 200$:

\begin{tikzpicture}
  \begin{axis}[axis lines = left, xlabel = \(n\), ylabel = {\(h(n)\)}]
    \addplot[domain = 0:1000, color=red]{x+200};
    \addlegendentry{$g(n)$};
    \addplot[domain = 0:1000, color=blue]{(3*x)+1};
    \addlegendentry{$f(n) (c_2 = 1)$};
    \addplot[domain = 0:1000, color=green]{((3*x)+1)/4};
    \addlegendentry{$\frac{f(n)}{4} (c_1 = \frac{1}{4})$};
  \end{axis}
\end{tikzpicture}

На картинката се вижда как от един момент нататък, функцията $f$ остава ``заключена`` между $c_1 \cdot g$ и $c_2 \cdot g$.
Точно заради това $g \in \Theta(f)$.
\begin{remark}
  Вместо да пишем $g \in \Theta(f)$, ще пишем $g = \Theta(f)$ или $g \asymp f$.
\end{remark}

\newpage

\begin{definition}
  За всяка функция $f \in \calF$ дефинираме:
  \begin{align*}
    O(f) = \{ g \in \calF \: \mid (\exists c > 0)(\exists n_0 \in \N)(\forall n \geq n_0)(g(n) \leq c \cdot f(n))\}.
  \end{align*}
\end{definition}
Може да тълкуваме $O(f)$ като:
\begin{center}
  \textit{``множеството от функциите, които не растат\footnotemark[1] по-бързо от $f$''.}
\end{center}


Тук заслабваме условията от $\Theta(f)$ като искаме само горната граница.

За пример човек може да вземе $f(n) = n^2$ и $g(n) = n$:

\begin{tikzpicture}
  \begin{axis}[axis lines = left, xlabel = \(n\), ylabel = {\(h(n)\)}]
    \addplot[domain = 0:10, color=red]{x*x};
    \addlegendentry{$f(n)$};
    \addplot[domain = 0:10, color=blue]{x};
    \addlegendentry{$g(n)$};
  \end{axis}
\end{tikzpicture}

\begin{remark}
  Вместо да пишем $g \in O(f)$, ще пишем $g = O(f)$ или $g \preceq f$.
\end{remark}

\begin{definition}
  За всяка функция $f \in \calF$ дефинираме:
  \begin{align*}
    o(f) = \{ g \in \calF \: \mid (\forall c > 0)(\exists n_0 \in \N)(\forall n \geq n_0)(g(n) < c \cdot f(n))\}.
  \end{align*}
\end{definition}
Може да тълкуваме $o(f)$ като:
\begin{center}
  \textit{``множеството от функциите, които растат\footnotemark[1] по-бавно от $f$''.}
\end{center}
Разликата между $O(f)$ и $o(f)$ е строгото неравенство и универсалният квантор в началото.
Лесно се вижда, че $o(f) \subseteq O(f)$.
Тук изключваме функциите от същия порядък.
\begin{remark}
  Вместо да пишем $g \in o(f)$, ще пишем $g = o(f)$ или $g \prec f$.
\end{remark}

\begin{definition}
  За всяка функция $f \in \calF$ дефинираме:
  \begin{align*}
    \Omega(f) = \{ g \in \calF \: \mid (\exists c > 0)(\exists n_0 \in \N)(\forall n \geq n_0)(c \cdot f(n) \leq g(n))\}.
  \end{align*}
\end{definition}
Може да тълкуваме $\Omega(f)$ като:
\begin{center}
  \textit{``множеството от функциите, които не растат\footnotemark[1] по-бавно от $f$''.}
\end{center}
Това е дуалното множество на $O(f)$.
\begin{remark}
  Вместо да пишем $g \in \Omega(f)$, ще пишем $g = \Omega(f)$ или $g \succeq f$.
\end{remark}

\begin{definition}
  За всяка функция $f \in \calF$ дефинираме:
  \begin{align*}
    \omega(f) = \{ g \in \calF \: \mid (\forall c > 0)(\exists n_0 \in \N)(\forall n \geq n_0)(c \cdot f(n) < g(n))\}.
  \end{align*}
\end{definition}
Може да тълкуваме $\omega(f)$ като:
\begin{center}
  \textit{``множеството от функциите, които растат\footnotemark[1] по-бързо от $f$''.}
\end{center}
Това е дуалното множество на $o(f)$.
\begin{remark}
  Вместо да пишем $g \in \omega(f)$, ще пишем $g = \omega(f)$ или $g \succ f$.
\end{remark}

\begin{warning}
  Не всички функции от $\calF$ са сравними по релациите $\prec, \preceq$ или $\asymp$.

  За пример човек може да вземе функциите $f(n) = n$ и $g(n) = n^{1 + \sin(n)}$.
  Лесно се вижда, че функцията $g(n)$ ``плава'' между $n^0 = 1$ и $n^2$ т.е. няма нито как да расте по-бързо, нито как да расте по-бавно.
\end{warning}

Въпреки това, тези релации са сравнително хубави.
\begin{claim}
  Следните свойства са в сила:
  \begin{itemize}
    \item $\asymp$ е релация на еквивалентност;
    \item $\prec$ и $\succ$ са транзитивни и антирефлексивни;
    \item $\preceq$ и $\succeq$ са транзитивни и рефлексивни.
  \end{itemize}
\end{claim}

Доказателството на това твърдение оставяме за упражнение на читателя.
То е една елементарна разходка из дефинициите.

\newpage

\section{Полезни свойства}

Тук ще изброим няколко свойства, които много често се ползват в задачите:
\begin{itemize}
  \item Нека $f, g \in \calF$ и $\lim\limits_{n \rightarrow \infty} \frac{f(n)}{g(n)} = l$ (тук искаме границата да съществува).
        Тогава: \\
        --- ако $l = 0$, то $f \prec g$; \\
        --- ако $l = \infty$, то $f \succ g$; \\
        --- в останалите случаи $f \asymp g$.
  \item $f + g \asymp \max\{f, g\}$ за всяко $f, g \in \calF$.
  \item $c \cdot f \asymp f$ за всяко $f \in F$ и $c > 0$.
  \item $f \asymp g \iff f^c \asymp g^c$ за всяко $f, g \in F$ и $c > 0$.
  \item $O(f) \cap \Omega(f) = \Theta(f)$ за всяко $f \in \calF$.
  \item $o(f) \cap \omega(f) = O(f) \cap \omega(f) = o(f) \cap \Omega(f) = \varnothing$ за всяко $f \in \calF$.
  \item $f \prec g \iff g \succ f$ и $f \preceq g \iff g \succeq f$ за всяко $f, g \in \calF$.
  \item ако $f \prec g$, то $c^f \prec c^g$ за всяко $f, g \in \calF$ и $c > 1$.
  \item ако $\log_c(f) \prec \log_c(g)$, то $f \prec g$ за всяко $f, g \in \calF$ и $c > 1$.
  \item ако $c^f \asymp c^g$, то $f \asymp g$ за всяко $f, g \in \calF$ и $c > 1$.
  \item ако $f \asymp g$, то $\log_c(f) \asymp \log_c(g)$ за всяко $f, g \in \calF$ и $c > 1$.
  \item тъй като $\log_a(n) = \frac{\log_b(n)}{\log_b(a)}$, то $\log_a(n) \asymp \log_b(n)$ -- вече ще пишем само $\log(n)$ като ще имаме предвид $\log_2(n)$.
  \item $n! \asymp \sqrt{n} \frac{n^n}{e^n}$ - апроксимация на Стирлинг.
  \item $\log(n!) \asymp n \log(n)$.
  \item $\log(n) \prec n^k \prec 2^n \prec n! \prec n^n \prec 2^{n^2}$ за всяко $k \geq 1$.
\end{itemize}

\section{Задачи}

\begin{problem}
Да се сравнят асимптотично следните двойки функции:
\begin{enumerate}
  \item $f(n) = \log(\log(n))$ и $g(n) = \log(n)$;
  \item $f(n) = 5n^3$ и $g(n) = n \sqrt{n^9 + n^5}$;
  \item $f(n) = n 5^n$ и $g(n) = n^ 2 3^n$;
  \item $f(n) = n^n$ и $g(n) = 3^{n^2}$;
  \item $f(n) = 3^{n^2}$ и $g(n) = 2^{n^3}$.
\end{enumerate}
\end{problem}

\begin{problem}
Да се докаже, че $\sum\limits_{i = 0}^n i^k \asymp n^{k+1}$.
\end{problem}

\begin{problem}
Да се подредят по асимптотично нарастване следните функции:
\begin{align*}
  f_1(n) & = n^2                                       & f_2(n)    & = \sqrt{n}       & f_3(n)    & = \log^2(n)        & f_4(n)    & = \sqrt{\log(n)!}                         \\
  f_5(n) & = \sum\limits_{k = 2}^{\log(n)} \frac{1}{k} & f_6(n)    & = \log(\log(n))  & f_7(n)    & = 2^{2^{\sqrt{n}}} & f_8(n)    & = \binom{\binom{n}{3}}{2}                 \\
  f_9(n) & =2^{n^2}                                    & f_{10}(n) & = 3^{n \sqrt{n}} & f_{11}(n) & =2^{\binom{n}{2}}  & f_{12}(n) & = \sum\limits_{k = 1}^{n^2} \frac{1}{2^k}.
\end{align*}
\end{problem}
\chapter{Анализ на сложността на итеративни алгоритми}

\section{Как анализираме един алгоритъм по сложност?}

Нека започнем с един прост пример:
\lstinputlisting{algorithms/find.txt}

Да кажем, че искаме да проверим броя на инструкциите, която тази функция ще изпълни, преди да приключи работата си.
Точен отговор не може да се даде.
В зависимост от това къде се намира $v$ във $A[1 \dots n]$, алгоритъмът може да приключи много бързо или много бавно.
Можем да дадем горна и долна граница на бързодействието.

Ако $v$ се намира в началото, то ще сме направили само следните $4$ операции:
\begin{itemize}
  \item да инициализираме променливата $i$ със $0$;
  \item да проверим верността на $i < n$;
  \item да проверим верността на $A[i] = v$;
  \item да върнем $i$ т.е. $1$.
\end{itemize}

Нека сега да помислим какво ще стане в най-лошия случай (обикновено от тези ще се интересуваме) -- $v$ не участва в $A[1 \dots n]$.
Тогава $n$ пъти ще изпълним следните $3$ операции:
\begin{itemize}
  \item проверяваме верността на $i \leq n$;
  \item проверяваме верността на $A[i] = v$;
  \item увеличаваме $i$ с $1$.
\end{itemize}
Освен тези $3n$ операции, преди всичко трябва да инициализираме променливата $i$ със $1$, да се направи последната проверка на верността на $i \leq n$ (която ще ни изкара от цикъла), и да върнем $-1$.
Общо излизат $3n + 3$ операции.

Така виждаме, че в зависимост от входните данни, алгоритъмът приключва работа за поне $4$ стъпки и най-много $3n + 3$ стъпки.
Такъв алгоритъм ще казваме, че има сложност по време $O(n)$.
Разбира се, няма да е грешно и да кажем, че алгоритъмът има сложност по време $\Omega(1)$, но това не ни дава никаква информация, защото всеки алгоритъм има такава сложност.
Също така, понеже не използваме допълнителни променливи, алгоритъмът ни има константна сложност по памет или сложност по памет $\Theta(1)$.

По-общо казано, се интересуваме от асимптотиката на $T(n)$, където $T(n)$ е броят елементарни инструкции, които алгоритъмът извиква по време на своето изпълнение, при вход с размер $n$ в най-лошият случай.

Тук вход с големина $n$ може да означава различни неща.
Ако входът е някакъв масив или множество, то под размер ще разбираме броят на елементи.
Ако пък входът е число, то под размер можем да разбираме самата стойност на числото или дължината на двоичния запис.

\section{Предимствата и недостатъците на този вид анализ}

Най-голямото предимство на асимптотичния анализ, е неговата простота.
Вместо да влачим някакви константни множители и събираеми, имаме колкото се може по-проста формула, която да описва сложността на нашия алгоритъм.
Това дали един алгоритъм работи със две или три стъпки по-бързо/бавно не ни интересува особено много.
При много голям вход те ще работят практически еднакво.
В някакъв смисъл това ни помага да виждаме по-голямата картинка.
Един алгоритъм може да бъде по-бърз от друг, но от по-бърз алгоритъм до по-бърз алгоритъм има голяма разлика.

Нека вземем за пример следната таблица:
\begin{center}
  \begin{tabular}{|c|c|c|c|c|}
    \hline
    $n$       & $\lceil\log_2(n)\rceil$ & $n$       & $n^2$           & $2^n$                    \\
    \hline
    $1$       & $0$                     & $1$       & $1$             & $2$                      \\
    \hline
    $10$      & $4$                     & $10$      & $100$           & $1024$                   \\
    \hline
    $100$     & $7$                     & $100$     & $10000$         & число със $31$ цифри     \\
    \hline
    $10000$   & $13$                    & $10000$   & $100000000$     & число със $3011$ цифри   \\
    \hline
    $1000000$ & $20$                    & $1000000$ & $1000000000000$ & число със $301030$ цифри \\
    \hline
  \end{tabular}
\end{center}

Алгоритъм със сложността $n^2$ ще е по-бавен от алгоритъм със сложност $n$, обаче скока в бързината е много по-малък от този между $2^n$ и $n^2$.

Този подход обаче си има своите недостатъци.
Нека разгледаме два алгоритъма със сложности по време съответно $n$ и $2^{2^{2^{2^{1024}}}}$.
Ние ведната ще се втурнем да кажем, че първият алгоритъм е по-лош.
Той е с линейна сложност, а вторият алгоритъм има константна сложност.
Обаче преди вторият алгоритъм даде отговор, всички звезди ще умрат т.е. няма да доживем да чуем този отговор.
Разбира се, от някъде нататък, за много голями входни данни, първият алгоритъм наистина ще работи по-бавно, но ние никога няма да работим с толкова големи данни.
Тогава на практика, първият алгоритъм е по-добър, нищо че асимптотично се води по-лош.
Нас това няма да ни интересува в курса по ДАА.

\section{Сложност по време на някои алгоритми}

Нека видим сложността на алгоритъма за сортиране по метода на мехурчето:
\lstinputlisting{algorithms/bubble-sort.txt}

В най-лошия случай сложността $T(n)$ на функцията {\tt Sort} е следната:
\[
  T(n) = \sum\limits_{i = 1}^{n - 1} \sum\limits_{j = 1}^{n - i - 1} 1 = \sum\limits_{i = 1}^{n - 1} (n - i - 1) = (n - 2) + (n - 3) + \dots + 0 = \frac{(n - 2)(n - 1)}{2} \asymp n^2.
\]

По принцип $T(n)$ трябва да е сума от $4$, а не от $1$, но такъв константен брой операции, дори и приложени неконстантен брой пъти, не влияят на асимптотичното поведение.

Нека сега разгледаме следният алгоритъм за степенуване:
\lstinputlisting{algorithms/exp.txt}

Той се възползва от простата идея, че за да сметнем да кажем $3^8$, можем вместо $8$ пъти да умножаваме числото $3$, да представим $3^8$ като $3^4 \cdot 3^4$.
Тогава $3^4$ можем да сметнем веднъж, и да го умножим със себе си.
Пак можем да представим $3^4$ като $3^2 \cdot 3^2$ и да пресметнем $3^2$ само веднъж и да го умножим със себе си.
Така при по-голяма стойност на $y$ си спестяваме много работа.
С уговорката, че умножението е атомарна операция, сложността по време $T(n)$ ($n$ е стойността на $y$) на функцията {\tt Exp} е следната:
\[
  T(n) = \sum\limits_{\substack{i = n \\ i \leftarrow \frac{i}{2}}}^1 1 = \underbrace{1 + \dots + 1}_{\substack{\text{колкото пъти} \\\text{можем да} \\ \text{делим целочислено} \\ n \text{ на } 2 \text{ преди} \\ \text{да получим } 0}} = \underbrace{1 + \dots + 1}_{\text{около } \log(n) \text{ пъти}} \asymp \log(n).
\]

\newpage

\section{Задачи}

\begin{problem}
Да се определи сложността по време за функцията:
\lstinputlisting{algorithms/task1.txt}
\end{problem}

\begin{problem}
Да се определи сложността по време за функцията:
\lstinputlisting{algorithms/task2.txt}
\end{problem}

\begin{problem}
Да се определи сложността по време за функцията:
\lstinputlisting{algorithms/task3.txt}
\end{problem}
\chapter{Коректност на итеративни алгоритми}

\section{Какво имаме предвид под коректност?}

За целите на този курс един алгоритъм ще наричаме \textbf{коректен}, ако завършва при всякакви входни данни и връща правилен резултат при всякакви входни данни

\begin{remark}
    Въпреки че ние ще имаме това разбиране в курса, на практика тези изисквания невинаги са изпълнени:
    \begin{itemize}
        \item разглеждат се алгоритми, които могат и да не завършват за някои входни данни -- от теоретична гледна точка са интересни за хората, които се занимават с теорията на изчислимостта;
        \item разглеждат се алгоритми, които много често (но не винаги) връщат правилния резултат -- обикновено това се прави с цел бързодействие.
    \end{itemize}
\end{remark}

\section{Едно \textit{``ново''} понятие}

Специално за итеративните алгоритми се въвежда ново понятие - \textbf{инвариант}.
Това са специални твърдения, свързани с цикъла.

В най-общият случай (за алгоритми) се формулират по следния начин:
\begin{center}
    ``При $k$-тото достигане на ред $l$ (ако има няколко инструкции казваме преди/след коя се намираме) в алгоритъма $\mathfrak{A}$ е изпълнено \textit{някакво твърдение, зависещо от $k$ и променливите, използвани в $\mathfrak{A}$}''.
\end{center}
Доказателството на такива твърдения протича с добре познатата индукция.
Първо доказваме базата т.е. какво се случва при първото достигане на цикъла.
Индуктивното предположение и индуктивната стъпка се обединяват в \textit{``нова''} фаза, наречена \textbf{поддръжка}.
Довършителните разсъждения, които по принцип се намират след доказването на твърдението чрез индукция, ще наричаме \textbf{терминация}.
Накрая показваме, че винаги ще излезнем от цикъла (\textbf{финитност}).
Обикновено това ще го смятаме за очевидно (най-вече за $\mathtt{for}$-цикли).

\begin{warning}
    Това, за което се използват инвариантите, е да се докаже коректността на ЕДИН цикъл, не на цял алгоритъм.
    Когато в алгоритъма ни има няколко цикъла, на всеки от тях трябва да съответства по един инвариант.
\end{warning}

\section{Пример}

Нека разгледаме следния алгоритъм за степенуване на $2$:
\lstinputlisting{algorithms/pow2.txt}

\begin{invariant}
    При всяко достигане на проверката за край на цикъла (на ред $4$) е изпълнено, че $r = 2^{i - 1}$.
\end{invariant}
\begin{proof}
    \phantom{1}

    \textbf{База.}
    Наистина при първото достигане имаме, че $i = 1$ и от там $r = 1 = 2^{i - 1}$.

    \textbf{Поддръжка.}
    Нека при някое непоследно достигане твърдението е изпълнено.
    Тогава преди следващото достигане на проверката на $r$ присвояваме $2r$, като знаем, че преди $r$ е бил $2^{i - 1}$, и след това на $i$ присвояваме $i + 1$.
    Така е ясно, че при новото достигане на проверката $r$ ще стане $2 \cdot 2^{i_{old} - 1} = 2^{i_{old}} = 2^{i - 1}$.

    \textbf{Терминация.}
    Ако е изпълнено условието за край на цикъла, то тогава $i = n + 1$, откъдето ще върнем $r = 2^{(n + 1) - 1} = 2^n$.

    \textbf{Финитност.}
    Величината $n - i$ започва с $n - 1$, и намалява с $1$, докато не стигне $-1$, когато ще излезнем от цикъла.
    Следователно алгоритъмът винаги завършва.
\end{proof}

\section{С инвариантите трябва да се внимава}

Един от често срещаните капани, в които попадат хората, е да не си формулират инвариантът добре.
Много е важно инвариант да дава достатъчна информация за това което наистина се случва в алгоритъма.
За целта ще разгледаме един пример:
\lstinputlisting{algorithms/selection-sort.txt}

На интуитивно ниво е ясно какво прави кода.
Намира най-малкия елемент, и го слага на първо място.
След това намира втория най-малък елемент, и го слага на второ място, и т.н.

Нещо, което някои биха се пробвали да направят за първия цикъл, е следното:
\begin{center}
    \textit{При всяко достигане на проверката за край на цикъла на ред $3$ подмасивът $A[1 \dots i - 1]$ е сортиран.}
\end{center}

Проблемът с това твърдение, е че може много лесно да се измисли алгоритъм, за който това твърдение е изпълнено, и изобщо не сортира елементите в масива:
\lstinputlisting{algorithms/trust-me-it-sorts.txt}

Очевидно този за този алгоритъм горната инвариант е изпълнена, но той е безсмислен.
Получаваме сортиран масив, но за сметка на това губим цялата информация, която сме имали за него.

Нещо друго, което е важно да се направи, е първо да се формулира инвариант за вътрешния цикъл, и после за външния, като тънкият момент тук е, че ще ни трябват допускания за първият инвариант.
Идеята е, че външния цикъл разчита на вътрешния да си свърши работата, и обратно вътрешния разчита (не винаги) на външния преди това да си е свършил работата.

Нека покажем как трябва да станат инвариантите, като доказателството оставяме за упражнение на читателя.
Нека $A^*[1 \dots n]$ е първоначалната стойност на входния масив.
\begin{invariant}[вътрешен цикъл]
    При всяко достигане на проверката за край на цикъла на ред $5$ имаме, че $m$ е индексът на най-малкия елемент в масива $A[i \dots j - 1]$.
\end{invariant}

\begin{invariant}[външен цикъл]
    При всяко достигане на проверката за край на цикъла на ред $2$ имаме, че масивът $A[1 \dots i - 1]$ съдържа сортирани първите $i - 1$ по големина елементи на $A^*[1 \dots n]$, като останалите са в $A[i \dots n]$.
\end{invariant}

Обикновено в доказателството на коректност на алгоритми най-трудното е да се формулира инвариантът.
Ако човек има добре формулирана инвариант, доказателството e на първо място възможно, а на второ -- по-лесно.

\section{Задачи}

\begin{problem}
Да се:
\begin{itemize}
    \item напише алгоритъм, който сумира числата в един масив;
    \item докаже неговата коректност;
    \item изследва сложността му по време и памет.
\end{itemize}
\end{problem}

\begin{problem}
Даден е следният алгоритъм:
\lstinputlisting{algorithms/anon-alg1.txt}

\begin{enumerate}
    \item Какво връща той? Отговорът да се обоснове.
    \item Каква е неговата сложност по време и памет?
\end{enumerate}
\end{problem}

\newpage

\begin{problem}
Даден е следният алгоритъм:
\lstinputlisting{algorithms/fib-iter-linear.txt}

Да се докаже, че $\mathfrak{F}(n)$ връща $n$-тото число на Фибоначи.
\end{problem}

\begin{problem}
Даден е следният: алгоритъм:
\lstinputlisting{algorithms/mult.txt}

Да се докаже, че при вход $n \times n$ матрици $A, B$ и $C$, функцията $\mathtt{Mult}(A, B, C)$ записва в $C$ произведението на $A$ и $B$.
Да се намери сложността му по време и памет.
\end{problem}

\chapter{Рекурентни уравнения}

\section{Защо са ни рекурентни уравнения?}

Те се появяват по естествен път, когато искаме да анализираме сложността на рекурсивни алгоритми.

Нека вземем за пример алгоритъма за двоично търсене:
\lstinputlisting{algorithms/binary-search-rec.txt}

При подаден сортиран целочислен масив $A[1 \dots n]$, негови индекси $l, r$ и цяло число $v$, функцията $\mathtt{BinarySearch}(A[1 \dots n], 1, n, v)$ ще върне индекс на $A[1 \dots n]$, в който се намира $v$, ако има такъв, иначе ще върне $-1$.
Нека помислим каква е сложността на алгоритъма.
Управляващите параметри на рекурсията са $l$ и $r$.
Всеки път разликата между двете намалява двойно (като накрая когато $l = r$ тя ще стане отрицателна).

\newpage

Това означава, че в най-лошия случай сложността на алгоритъма може да се опише със следното рекурентно уравнение:
\begin{align*}
     & T(0) = 2 \text{ // заради ред } 3 \text{ и } 4                                                          \\
     & T(n + 1) = T(\lfloor \frac{n + 1}{2} \rfloor) + 5 \text{ // заради проверките и рекурсивното извикване}
\end{align*}

В този случай лесно се вижда асимптотиката на $T(n)$:
\begin{align*}
    T(n) & = T(\lfloor \frac{n}{2} \rfloor) + 5                                                                                                         \\
         & = T(\lfloor \frac{n}{4} \rfloor) + 5 + 5                                                                                                     \\
         & = T(\lfloor \frac{n}{8} \rfloor) + 5 + 5 + 5 = \dots = T(0) + \underbrace{5 + \dots + 5}_{\text{около } \log(n) \text{ пъти}} \asymp \log(n)
\end{align*}

Така получаваме, че алгоритъмът има сложност $O(\log(n))$.
Обаче в общият случай далеч не е толкова лесно да се намери асимптотичното поведение на дадено рекурентно уравнение.
Целта ни ще бъде да развием по-богат апарат за асимптотичен анализ на рекурентните уравнения.

\section{Начини за намиране на асимптотиката на рекурентни уравнения}

Начините се разделят на два типа:
\begin{itemize}
    \item със решаване на уравнението;
    \item без решаване на уравнението.
\end{itemize}

И двата начина са ценни.
Първият начин ни дава формула във явен вид, което може да ни е от полза.
Понякога обаче формулата във явен вид не е \textit{``красива''}, или изобщо не може да се намери такава.
Тогава идва на помощ вторият начин.
Той директно ни дава някаква \textit{``хубава''} формула, без да трябва да намираме в явен вид решение на рекурентното уравнение.
Проблема е обаче, че асимптотиката понякога е малко лъжлива -- алгоритъм със сложност $2^{2^{2^{1000}}}$ е асимптотично по-бавен от алгоритъм със сложност $n$, но практически вторият е по-бърз.

\newpage

Ще разгледаме следните методи (повечето от които са разглеждани по дискретна математика):
\begin{itemize}
    \item налучкване и доказване
    \item развиване (което преди малко показахме)
    \item методът с характеристичното уравнение
    \item мастър-теоремата
\end{itemize}

Нека разгледаме един пример с налучкване:
\begin{align*}
     & T(0) = 3                   \\
     & T(n + 1) = (n + 1)T(n) - n
\end{align*}

Започваме да разписваме:
\begin{center}
    \begin{tabular}{| c | c | c |}
        \hline
        $n$ & $T(n)$ & $n!$  \\
        \hline
        $0$ & $3$    & $1$   \\
        \hline
        $1$ & $3$    & $1$   \\
        \hline
        $2$ & $5$    & $2$   \\
        \hline
        $3$ & $13$   & $6$   \\
        \hline
        $4$ & $49$   & $24$  \\
        \hline
        $5$ & $241$  & $120$ \\
        \hline
        $6$ & $1441$ & $720$ \\
        \hline
    \end{tabular}
\end{center}

Вече лесно можем да покажем с индукция, че $T(n) = 2(n!) + 1$:
\begin{itemize}
    \item В базата имаме, че $T(0) = 3 = 2 \cdot 1 + 1 = 2 \cdot 0! + 1$.
    \item За индуктивната стъпка:
          \begin{align*}
              T(n + 1) & = (n + 1)T(n) - n \stackrel{\text{(ИП)}}{=} (n + 1)(2(n!) + 1) - n \\
                       & = (n + 1)(2(n!)) + n + 1 - n = 2(n + 1)! + 1
          \end{align*}
\end{itemize}
Накрая получаваме, че $T(n) \asymp n!$

Нека сега да видим как можем да използваме метода на характеристичното уравнение:
\[
    T(n) = 1 + \sum\limits_{i = 0}^{n - 1}T(i)    \text{ // функцията е добре дефинирана и за } 0.
\]
Рекурентното уравнение, зададено в този вид, не може да се реши с този метод.
За това ще трябва да направим преобразувания:
\begin{align*}
    T(0)     & = 1                                                                                                                                      \\
    T(n + 1) & = 1 + \sum\limits_{i = 0}^{n}T(i) = 1 + T(n) + \sum\limits_{i = 0}^{n - 1}T(i)                                                           \\
             & = T(n) + \underbrace{\left( 1 + \sum\limits_{i = 0}^{n - 1}T(i) \right)}_{T(n)} = 2T(n) + 1 = \underbrace{2T(n)}_{\text{хомогенна част}}
\end{align*}

Имаме само хомогенна част, от която получаваме характеристичното уравнение $x - 2 = 0$ с единствен корен $2$.
Така:
\[
    T(n) = A \cdot 2^n \text{ за някоя константи } A.
\]
Вече няма нужда и да се намира константата -- ясно е че $T(n) \asymp 2^n$.
Kато използваме метода на характеристичното уравнение, не е нужно да намираме накрая константите за да разберем каква е асимптотиката.
Достатъчно е да вземем събираемото, която расте най-много. В случая е ясно, че това е $2^n$.

Нека сега разгледаме и последният начин:
\begin{theorem}[Мастър-теорема]
    Нека $a \geq 1, \: b > 1$ и $f \in \F$.
    Нека $T(n) = aT(\frac{n}{b}) + f(n)$, където $\frac{n}{b}$ се интерпретира като $\lfloor \frac{n}{b} \rfloor$ или $\lceil \frac{n}{b} \rceil$.
    Тогава:
    \begin{itemize}
        \item[1 сл.] Ако $f(n) \preceq n^{\log_b(a) - \varepsilon}$ за някое $\varepsilon > 0$, то тогава $T(n) \asymp n^{log_b(a)}$.
        \item[2 сл.] Ако $f(n) \asymp n^{log_b(a)}$, то тогава $T(n) \asymp n^{log_b(a)} \log(n)$.
        \item[3 сл.] Ако са изпълнени следните условия:
              \begin{enumerate}
                  \item $f(n) \succeq n^{\log_b(a) + \varepsilon}$ за някое $\varepsilon > 0$; и
                  \item съществува $0 < c < 1$, за което от някъде нататък $a \cdot f(\frac{n}{b}) \leq c \cdot f(n)$,
              \end{enumerate}
              то тогава $T(n) \asymp f(n)$.
    \end{itemize}
\end{theorem}

Нека разгледаме рекурентното уравнение:
\[
    T(n) = 2T(\frac{n}{2}) + 1.
\]
Тук $a = b = 2$, и $f(n) = 1$.
Също така $\log_b(a) = 1$, откъдето $f(n) = 1 \preceq n^{\log_b(a) - \varepsilon}$, за $\varepsilon \in (0, 1)$.
Така по 1 сл. на мастър-теоремата получаваме, че $T(n) \asymp n$.

\section{Задачи}

\begin{problem}
Да се намери асимптотиката на средната сложност по време на алгоритъма за бързо сортиране т.е. на рекурентното уравнение:
\[
    T(n) = \frac{1}{n} \left(\sum\limits_{i = 1}^{n - 1}T(i) + T(n - i)\right) + n - 1.
\]
\end{problem}

\begin{problem}
Да се намери асимптотиката на следните рекурентни уравнения:
\begin{align*}
    T_1(n) & = 29T_1(\frac{n}{3}) + 2 \sum\limits_{i = 1}^n \frac{1}{i^2} & T_2(n) & = 29T_2(\frac{n}{3}) + 12n + \sqrt{n}                        \\
    T_3(n) & = T_3(n - 1) + \frac{n}{(n + 1)(n - 1)}                      & T_4(n) & = 29T_4(\frac{n}{3}) + (\sum\limits_{i = 1}^n \frac{1}{i})^4 \\
    T_5(n) & = 29T_5(\frac{n}{3}) + 2 \sum\limits_{i = 1}^n i^2           & T_6(n) & = 29T_6(\frac{n}{3}) + n^{\sqrt{n}} + (\sqrt{n})^n           \\
    T_7(n) & = T_7(\sqrt{n}) + n                                          & T_8(n) & = 29T_8(\frac{n}{3}) + \binom{2n}{2}                         \\
    T_9(n) & = 8T_9(n - 1) - T_9(n - 2) + 2n2^{2n} + 3n2^{3n}.            &        &
\end{align*}
\end{problem}

\begin{problem}
Да се намери сложността по време на следния алгоритъм:
\lstinputlisting{algorithms/alg1.txt}
Ще се промени ли нещо ако връщаме $a + 2 \mathfrak{A}_1(n - 1) + \mathfrak{A}_1(n - 2)$?
\end{problem}

\newpage

\begin{problem}
Да се намери сложността по време на следния алгоритъм:
\lstinputlisting{algorithms/alg2.txt}
\end{problem}

\begin{problem}
Да се намери асимптотиката на следните рекурентни уравнения:
\begin{align*}
    T_1(n) & = 2 \sqrt{2} T_1\left(\frac{n}{\sqrt{2}}\right) + n^3              & T_2(n) & = T_2(n - 1) + \frac{1 + n}{n^2}             \\
    T_3(n) & = \sum\limits_{i = 0}^{n - 1}\left(T_3(i) + 2^{\frac{n}{2}}\right) & T_4(n) & = 5 T_4\left(\frac{n}{2}\right) + n^2 \log(n).
\end{align*}
\end{problem}
\chapter{Коректност на рекурсивни алгоритми}

\section{Примери за рекурсивни алгоритми}

Нека разгледаме следният алгоритъм:
\lstinputlisting{algorithms/maximum.txt}

Очевидно при параметър целочислен масив $A[1 \dots n]$, функцията $\mathfrak{M}(A[1 \dots n])$ връща $\max A[1 \dots n]$.
Ще докажем това с индукция по $n$:
\begin{itemize}
    \item В базата имаме, че $\mathfrak{M}(A[1 \dots n]) = -\infty = \max [] = \max A[1 \dots 0]$ където със $[]$ означаваме празният масив.
    \item За индуктивната стъпка имаме, че:
          \begin{align*}
              \mathfrak{M}(A[1 \dots n + 1]) & = \max(A[n + 1], \mathfrak{M}(A[1 \dots n])) \stackrel{\text{(ИП)}}{=} \max(A[n + 1], \max A[1 \dots n]) \\
                                             & = \max A[1 \dots n + 1].
          \end{align*}
\end{itemize}

Тук управляващият параметър на рекурсията $\mathtt{n}$ винаги намалява с $\mathtt{1}$, докато не стигне $\mathtt{0}$, където ще приключи алгоритъмът.
В по нататъчните разсъждения ще смятаме това за очевидно.

Сложността на алгоритъма се описва със рекурентното уравнение:
\[
    T(n) = T(n - 1) + 1 \text{ // базата няма да я пишем}.
\]

Директно се вижда, че $T(n) = \sum\limits_{i = 0}^n 1 = n + 1 \asymp n$.

Да видим един малко по-сложен пример -- за бързо степенуване:
\lstinputlisting{algorithms/exp-rec.txt}

С пълна индукция относно $y$ ще покажем, че $\mathcal{P}(x, y) = x^y$:
\begin{itemize}
    \item $\mathcal{P}(x, 0) = 1 = x^0 = 1$ // тук се уговаряме, че $0^0 = 1$.
    \item $\mathcal{P}(x, 2y + 1) = x \cdot \mathcal{P}(x, y) \cdot \mathcal{P}(x, y) \stackrel{\text{(ИП)}}{=} x \cdot x^y \cdot x^y = x^{2y + 1}$.
    \item $\mathcal{P}(x, 2y + 2) = \mathcal{P}(x, y + 1) \cdot \mathcal{P}(x, y + 1) \stackrel{\text{(ИП)}}{=} x^{y + 1} \cdot x^{y + 1} = x^{2y + 2}$.
\end{itemize}

Сложността на този алгоритъм може да се опише със следното рекурентно уравнение:
\[
    T(n) = T(\frac{n}{2}) + 1.
\]
От мастър-теоремата следва, че:
\[
    T(n) \asymp \log(n).
\]

\section{Трик за бързо пресмятане на членове на някои рекурентни редици}

Нека вземем за пример редицата на Фибоначи:
\begin{align*}
     & F(0) = 0                   \\
     & F(1) = 1                   \\
     & F(n + 2) = F(n + 1) + F(n)
\end{align*}
Човек може да забележи, че:
\[
    \underbrace{\begin{pmatrix}
            1 & 1 \\
            1 & 0
        \end{pmatrix}}_{\mathfrak{F}^*}
    \cdot
    \begin{pmatrix}
        F(n + 1) \\
        F(n)
    \end{pmatrix}
    =
    \begin{pmatrix}
        F(n + 2) \\
        F(n + 1)
    \end{pmatrix}.
\]
След това с индукция лесно се показва, че:
\[
    (\mathfrak{F}^*)^n
    \cdot
    \begin{pmatrix}
        F(1) \\
        F(0)
    \end{pmatrix}
    =
    \begin{pmatrix}
        F(n + 1) \\
        F(n)
    \end{pmatrix}.
\]

За да сметнем $F(n)$, можем да направим бързо степенуване на матрицата, аналогична на тази с числата:
\lstinputlisting{algorithms/fibonacci-fast.txt}

Коректността и сложността на алгоритъма оставяме на читателя (напълно аналогично е на предния алгоритъм).

В общия случай ще имаме рекурентно уравнение от вида:
\[
    T(n + k + 1) = a_k T(n + k) + \dots + a_1 T(n + 1) + a_0 T(n) \text{, където } a_0, \dots, a_k, k \text{ са константи}.
\]
Тогава имаме следната зависимост:
\[
    \begin{pmatrix}
        a_k    & a_{k - 1} & a_{k - 2} & \dots  & a_2    & a_1    & a_0    \\
        1      & 0         & 0         & \dots  & 0      & 0      & 0      \\
        0      & 1         & 0         & \dots  & 0      & 0      & 0      \\
        \vdots & \vdots    & \vdots    & \ddots & \vdots & \vdots & \vdots \\
        0      & 0         & 0         & \dots  & 0      & 1      & 0      \\
    \end{pmatrix}
    \cdot
    \begin{pmatrix}
        T(n + k)     \\
        T(n + k - 1) \\
        T(n + k - 2) \\
        \vdots       \\
        T(n)
    \end{pmatrix}
    =
    \begin{pmatrix}
        T(n + k + 1) \\
        T(n + k)     \\
        T(n + k - 1) \\
        \vdots       \\
        T(n + 1)
    \end{pmatrix}.
\]
Отново с индукция лесно се показва, че:
\[
    \begin{pmatrix}
        a_k    & a_{k - 1} & a_{k - 2} & \dots  & a_2    & a_1    & a_0    \\
        1      & 0         & 0         & \dots  & 0      & 0      & 0      \\
        0      & 1         & 0         & \dots  & 0      & 0      & 0      \\
        \vdots & \vdots    & \vdots    & \ddots & \vdots & \vdots & \vdots \\
        0      & 0         & 0         & \dots  & 0      & 1      & 0      \\
    \end{pmatrix}^n
    \cdot
    \begin{pmatrix}
        T(k)     \\
        T(k - 1) \\
        T(k - 2) \\
        \vdots   \\
        T(0)
    \end{pmatrix}
    =
    \begin{pmatrix}
        T(n + k)     \\
        T(n + k - 1) \\
        T(n + k - 2) \\
        \vdots       \\
        T(n)
    \end{pmatrix}.
\]

\section{Задачи}

\begin{problem}
Да се напише алгоритъм $\mathtt{Calculate}(F[0 \dots k], S[0 \dots k], n)$, който приема два целочислени масива $F[0 \dots k]$ и $S[0 \dots k]$, естествено число $n$, и връща числото $T(n)$, където:
\begin{align*}
     & T(0) = F[0]                                                                         \\
     & T(1) = F[1]                                                                         \\
     & \phantom{00000} \vdots                                                              \\
     & T(k) = F[k]                                                                         \\
     & T(n + k + 1) = S[k] \cdot T(n + k) + \dots + S[1] \cdot T(n + 1) + S[0] \cdot T(n).
\end{align*}
След това да се докаже неговата коректност, и да се изследва сложността му по време.
\end{problem}

\newpage

\begin{problem}
Даден е следният алгоритъм:
\lstinputlisting{algorithms/stooge-sort.txt}
Какво прави $\mathtt{SS}(A[1 \dots n], 1, n)$ и каква е сложността на този алгоритъм по време и памет?
Обосновете отговорите си.
\end{problem}

\begin{problem}
Да се напише алгоритъм $\mathtt{IsDerivationTree}(G = \opair{\Sigma, V, S, R}, T)$, който приема безконтекстна граматика $G$, дърво $T$, и проверява дали $T$ е дърво на извод за $G$.
След това да се докаже неговата коректност и да се изследва сложността му по време и памет.
\end{problem}

\end{document}