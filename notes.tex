\documentclass[openany, 12pt]{book}

\usepackage[utf8]{inputenc}
\usepackage[T2A]{fontenc}
\usepackage[english, bulgarian]{babel}
\usepackage{pgfplots}
\usepackage{amssymb}
\usepackage{hyperref, fancyhdr, lastpage, fancyvrb, tcolorbox, titlesec}
\usepackage{array, tabularx, colortbl}
\usepackage{tikz}
\usepackage{venndiagram}
\usepackage{amsthm, bm}
\usepackage{relsize}
\usepackage{amsmath,physics}
\usepackage{mathtools}
\usepackage{subcaption}
\usepackage{theoremref}
\usepackage{circuitikz}
\usepackage{geometry}
\usepackage{stmaryrd}
\usepackage[symbol]{footmisc}
\usepackage{minted}
\usepackage{enumitem}
\usepackage{listings}
\usepackage{systeme}
\usepackage{forest}
\useforestlibrary{linguistics}

\pgfplotsset{width=10cm,compat=1.9}

\newcommand{\N}{\mathbb{N}}
\newcommand{\Z}{\mathbb{Z}}
\newcommand{\R}{\mathbb{R}}
\newcommand{\T}{\mathbb{T}}
\newcommand{\F}{\mathbb{F}}
\newcommand{\calF}{\mathcal{F}}
\newcommand{\calT}{\mathcal{T}}
\newcommand{\NP}{\textbf{NP}}
\newcommand{\arr}{\operatorname{array}(\mathbb{Z})}


\ExplSyntaxOn
\NewDocumentCommand{\opair}{m}
 {
  \langle\mspace{2mu}
  \clist_set:Nn \l_tmpa_clist { #1 }
  \clist_use:Nn \l_tmpa_clist {,\mspace{3mu plus 1mu minus 1mu}\allowbreak}
  \mspace{2mu}\rangle
}
\ExplSyntaxOff

\hypersetup{
    colorlinks=true,
    linktoc=all,
    linkcolor=blue
}

\lstset{basicstyle=\ttfamily,
        breaklines=true,mathescape=true,numbers=left,
        inputencoding=utf8,extendedchars=true,frame=single}

\theoremstyle{definition}
\newtheorem*{definition}{Дефиниция}
\newtheorem*{warning}{\textcolor{red}{Внимание}}
\theoremstyle{plain}
\newtheorem{theorem}{Теорема}[section]
\newtheorem{invariant}{Инвариант}
\newtheorem{claim}[theorem]{Твърдение}
\newtheorem{axiom}[theorem]{Аксиома}
\newtheorem{lemma}[theorem]{Лема}
\newtheorem{corollary}[theorem]{Следствие}
\theoremstyle{remark}
\newtheorem*{remark}{Забележка}
\newtheorem{problem}{Задача}[chapter]
\newtheorem*{solution}{Решение}
\theoremstyle{definition}

\pagestyle{fancy}

\lhead{\leftmark}
\rhead{}

\setlength\parindent{0pt}

\begin{document}

\begin{titlepage}
    \title{Записки по ДАА}
    \author{Тодор Дуков}
    \date{}
\end{titlepage}

\maketitle

\tableofcontents

\chapter{Въведение в алгоритмите и асимптотичния анализ}

\section{Що е то алгоритъм?}

Алгоритмите се срещат навсякъде около нас:
\begin{itemize}
  \item рецептите са алгоритми за готвене;
  \item сутрешното приготвяне;
  \item придвижването от точка A до точка B;
  \item търсенето на книга в библиотеката.
\end{itemize}

Въпреки това е трудно да се даде формална дефиниция на това какво точно е алгоритъм.
На ниво интуиция, човек може да си мисли, че това просто е някакъв последователен списък от стъпки/инструкции, които човек/машина трябва да изпълни.
Други начини човек да си мисли за алгоритмите, са:
\begin{itemize}
  \item програми -- обикновено така се реализират алгоритми;
  \item машини на Тюринг, крайни (стекови) автомати или формални граматики;
  \item частично рекурсивни функции.
\end{itemize}

Един програмист в ежедневието си постоянно пише алгоритми за да решава различни задачи/проблеми.
Една задача може да се решава по много начини, някои по-добри от други.
Добрият програмист, освен че ще намери решение на проблема, той ще намери най-доброто решение (или поне достатъчно добро за неговите цели).

\section{Какво означава добро решение?}

Хубаво е човек да се води по следните (неизчерпателни) критерии:
\begin{itemize}
  \item решението трябва да е коректно -- ако алгоритъмът работи само през 50\% от времето, най-вероятно можем да се справим по-добре;
  \item решението трябва да е бързо -- ако алгоритъмът ще завърши работа след като всички звезди са измрели, то той практически не ни върши работа;
  \item решението трябва да заема малко памет -- ако алгоритъмът по време на своята работа се нуждае от повече памет, колкото компютърът може да предостави, за нас този алгоритъм е безполезен;
  \item решението трябва да е просто -- това е може би най-маловажният критерии от тези, но въпреки това е хубаво когато човек може, да пише чист и разбираем код, който лесно се разширява.
\end{itemize}

За да можем да сравняваме алгоритми в зависимост от това колко големи ресурси (време и памет) използват, трябва първо да можем да ``измерваме'' тези ресурси.

\section{Как мерим времето и паметта?}

Когато пишем алгоритми, имаме няколко базови инструкции (за които предварително сме се уговорили), които ще наричаме \textbf{атомарни инструкции}.
Тяхното извикване ще отнеме една единица време.
\textbf{Време за изпълнение} ще наричаме броят на извикванията на атомарните инструкции по време на изпълнение на програмата.
Също така числата и символите ще бъдат нашите \textbf{атомарни типове данни}, и ще заемат една единица памет.
\textbf{Паметта}, която една програма заема, ще наричаме максималния брой на единици от атомарни типове данни по време на изпълнение, без да броим входните данни.
Обикновено времето и паметта зависят от размера на подадените входни данни.
Това означава, че можем да си мислим за времето и паметта като функции на размера на входа.
Подходът, който ще изберем, е да сравняваме функциите за време/памет на различните алгоритми асимптотично.
Интересуваме се не толкова от конкретните стойности, а от поведението им, когато размерът на входа клони към безкрайност.

\section{Основни дефиниции}
\footnotetext[1]{точност до константен множител и константно събираемо}
Множеството от функции, които ще анализираме, е
\[
  \calF = \{ f \mid f : \R^{\geq 0} \rightarrow \R \: \& \: (\exists n_0 > 0) (\forall n \geq n_0) (f(n) > 0) \}.
\]


\begin{definition}
  За всяка функция $f \in \calF$ дефинираме:
  \begin{align*}
    \Theta(f) = \{ g \in \calF \: \mid & (\exists c_1 > 0)(\exists c_2 > 0)                                                        \\
                                       & (\exists n_0 \in \N)(\forall n \geq n_0)(c_1 \cdot f(n) \leq g(n) \leq c_2 \cdot f(n))\}.
  \end{align*}

\end{definition}
Може да тълкуваме $\Theta(f)$ като:
\begin{center}
  \textit{``множеството от функциите, които растат\footnotemark[1] със скоростта на $f$''.}
\end{center}


Нека вземем за пример $f(n) = 3n + 1$ и $g(n) = n + 200$:

\begin{tikzpicture}
  \begin{axis}[axis lines = left, xlabel = \(n\), ylabel = {\(h(n)\)}]
    \addplot[domain = 0:1000, color=red]{x+200};
    \addlegendentry{$g(n)$};
    \addplot[domain = 0:1000, color=blue]{(3*x)+1};
    \addlegendentry{$f(n) (c_2 = 1)$};
    \addplot[domain = 0:1000, color=green]{((3*x)+1)/4};
    \addlegendentry{$\frac{f(n)}{4} (c_1 = \frac{1}{4})$};
  \end{axis}
\end{tikzpicture}

На картинката се вижда как от един момент нататък, функцията $f$ остава ``заключена`` между $c_1 \cdot g$ и $c_2 \cdot g$.
Точно заради това $g \in \Theta(f)$.
\begin{remark}
  Вместо да пишем $g \in \Theta(f)$, ще пишем $g = \Theta(f)$ или $g \asymp f$.
\end{remark}

\newpage

\begin{definition}
  За всяка функция $f \in \calF$ дефинираме:
  \begin{align*}
    O(f) = \{ g \in \calF \: \mid (\exists c > 0)(\exists n_0 \in \N)(\forall n \geq n_0)(g(n) \leq c \cdot f(n))\}.
  \end{align*}
\end{definition}
Може да тълкуваме $O(f)$ като:
\begin{center}
  \textit{``множеството от функциите, които не растат\footnotemark[1] по-бързо от $f$''.}
\end{center}


Тук заслабваме условията от $\Theta(f)$ като искаме само горната граница.

За пример човек може да вземе $f(n) = n^2$ и $g(n) = n$:

\begin{tikzpicture}
  \begin{axis}[axis lines = left, xlabel = \(n\), ylabel = {\(h(n)\)}]
    \addplot[domain = 0:10, color=red]{x*x};
    \addlegendentry{$f(n)$};
    \addplot[domain = 0:10, color=blue]{x};
    \addlegendentry{$g(n)$};
  \end{axis}
\end{tikzpicture}

\begin{remark}
  Вместо да пишем $g \in O(f)$, ще пишем $g = O(f)$ или $g \preceq f$.
\end{remark}

\begin{definition}
  За всяка функция $f \in \calF$ дефинираме:
  \begin{align*}
    o(f) = \{ g \in \calF \: \mid (\forall c > 0)(\exists n_0 \in \N)(\forall n \geq n_0)(g(n) < c \cdot f(n))\}.
  \end{align*}
\end{definition}
Може да тълкуваме $o(f)$ като:
\begin{center}
  \textit{``множеството от функциите, които растат\footnotemark[1] по-бавно от $f$''.}
\end{center}
Разликата между $O(f)$ и $o(f)$ е строгото неравенство и универсалният квантор в началото.
Лесно се вижда, че $o(f) \subseteq O(f)$.
Тук изключваме функциите от същия порядък.
\begin{remark}
  Вместо да пишем $g \in o(f)$, ще пишем $g = o(f)$ или $g \prec f$.
\end{remark}

\begin{definition}
  За всяка функция $f \in \calF$ дефинираме:
  \begin{align*}
    \Omega(f) = \{ g \in \calF \: \mid (\exists c > 0)(\exists n_0 \in \N)(\forall n \geq n_0)(c \cdot f(n) \leq g(n))\}.
  \end{align*}
\end{definition}
Може да тълкуваме $\Omega(f)$ като:
\begin{center}
  \textit{``множеството от функциите, които не растат\footnotemark[1] по-бавно от $f$''.}
\end{center}
Това е дуалното множество на $O(f)$.
\begin{remark}
  Вместо да пишем $g \in \Omega(f)$, ще пишем $g = \Omega(f)$ или $g \succeq f$.
\end{remark}

\begin{definition}
  За всяка функция $f \in \calF$ дефинираме:
  \begin{align*}
    \omega(f) = \{ g \in \calF \: \mid (\forall c > 0)(\exists n_0 \in \N)(\forall n \geq n_0)(c \cdot f(n) < g(n))\}.
  \end{align*}
\end{definition}
Може да тълкуваме $\omega(f)$ като:
\begin{center}
  \textit{``множеството от функциите, които растат\footnotemark[1] по-бързо от $f$''.}
\end{center}
Това е дуалното множество на $o(f)$.
\begin{remark}
  Вместо да пишем $g \in \omega(f)$, ще пишем $g = \omega(f)$ или $g \succ f$.
\end{remark}

\begin{warning}
  Не всички функции от $\calF$ са сравними по релациите $\prec, \preceq$ или $\asymp$.

  За пример човек може да вземе функциите $f(n) = n$ и $g(n) = n^{1 + \sin(n)}$.
  Лесно се вижда, че функцията $g(n)$ ``плава'' между $n^0 = 1$ и $n^2$ т.е. няма нито как да расте по-бързо, нито как да расте по-бавно.
\end{warning}

Въпреки това, тези релации са сравнително хубави.
\begin{claim}
  Следните свойства са в сила:
  \begin{itemize}
    \item $\asymp$ е релация на еквивалентност;
    \item $\prec$ и $\succ$ са транзитивни и антирефлексивни;
    \item $\preceq$ и $\succeq$ са транзитивни и рефлексивни.
  \end{itemize}
\end{claim}

Доказателството на това твърдение оставяме за упражнение на читателя.
То е една елементарна разходка из дефинициите.

\newpage

\section{Полезни свойства}

Тук ще изброим няколко свойства, които много често се ползват в задачите:
\begin{itemize}
  \item Нека $f, g \in \calF$ и $\lim\limits_{n \rightarrow \infty} \frac{f(n)}{g(n)} = l$ (тук искаме границата да съществува).
        Тогава: \\
        --- ако $l = 0$, то $f \prec g$; \\
        --- ако $l = \infty$, то $f \succ g$; \\
        --- в останалите случаи $f \asymp g$.
  \item $f + g \asymp \max\{f, g\}$ за всяко $f, g \in \calF$.
  \item $c \cdot f \asymp f$ за всяко $f \in F$ и $c > 0$.
  \item $f \asymp g \iff f^c \asymp g^c$ за всяко $f, g \in F$ и $c > 0$.
  \item $O(f) \cap \Omega(f) = \Theta(f)$ за всяко $f \in \calF$.
  \item $o(f) \cap \omega(f) = O(f) \cap \omega(f) = o(f) \cap \Omega(f) = \varnothing$ за всяко $f \in \calF$.
  \item $f \prec g \iff g \succ f$ и $f \preceq g \iff g \succeq f$ за всяко $f, g \in \calF$.
  \item ако $f \prec g$, то $c^f \prec c^g$ за всяко $f, g \in \calF$ и $c > 1$.
  \item ако $\log_c(f) \prec \log_c(g)$, то $f \prec g$ за всяко $f, g \in \calF$ и $c > 1$.
  \item ако $c^f \asymp c^g$, то $f \asymp g$ за всяко $f, g \in \calF$ и $c > 1$.
  \item ако $f \asymp g$, то $\log_c(f) \asymp \log_c(g)$ за всяко $f, g \in \calF$ и $c > 1$.
  \item тъй като $\log_a(n) = \frac{\log_b(n)}{\log_b(a)}$, то $\log_a(n) \asymp \log_b(n)$ -- вече ще пишем само $\log(n)$ като ще имаме предвид $\log_2(n)$.
  \item $n! \asymp \sqrt{n} \frac{n^n}{e^n}$ - апроксимация на Стирлинг.
  \item $\log(n!) \asymp n \log(n)$.
  \item $\log(n) \prec n^k \prec 2^n \prec n! \prec n^n \prec 2^{n^2}$ за всяко $k \geq 1$.
\end{itemize}

\section{Задачи}

\begin{problem}
Да се сравнят асимптотично следните двойки функции:
\begin{enumerate}
  \item $f(n) = \log(\log(n))$ и $g(n) = \log(n)$;
  \item $f(n) = 5n^3$ и $g(n) = n \sqrt{n^9 + n^5}$;
  \item $f(n) = n 5^n$ и $g(n) = n^ 2 3^n$;
  \item $f(n) = n^n$ и $g(n) = 3^{n^2}$;
  \item $f(n) = 3^{n^2}$ и $g(n) = 2^{n^3}$.
\end{enumerate}
\end{problem}

\begin{problem}
Да се докаже, че $\sum\limits_{i = 0}^n i^k \asymp n^{k+1}$.
\end{problem}

\begin{problem}
Да се подредят по асимптотично нарастване следните функции:
\begin{align*}
  f_1(n) & = n^2                                       & f_2(n)    & = \sqrt{n}       & f_3(n)    & = \log^2(n)        & f_4(n)    & = \sqrt{\log(n)!}                         \\
  f_5(n) & = \sum\limits_{k = 2}^{\log(n)} \frac{1}{k} & f_6(n)    & = \log(\log(n))  & f_7(n)    & = 2^{2^{\sqrt{n}}} & f_8(n)    & = \binom{\binom{n}{3}}{2}                 \\
  f_9(n) & =2^{n^2}                                    & f_{10}(n) & = 3^{n \sqrt{n}} & f_{11}(n) & =2^{\binom{n}{2}}  & f_{12}(n) & = \sum\limits_{k = 1}^{n^2} \frac{1}{2^k}.
\end{align*}
\end{problem}
\chapter{Анализ на сложността на итеративни алгоритми}

\section{Как анализираме един алгоритъм по сложност?}

Нека започнем с един прост пример:
\lstinputlisting{algorithms/find.txt}

Да кажем, че искаме да проверим броя на инструкциите, която тази функция ще изпълни, преди да приключи работата си.
Точен отговор не може да се даде.
В зависимост от това къде се намира $v$ във $A[1 \dots n]$, алгоритъмът може да приключи много бързо или много бавно.
Можем да дадем горна и долна граница на бързодействието.

Ако $v$ се намира в началото, то ще сме направили само следните $4$ операции:
\begin{itemize}
  \item да инициализираме променливата $i$ със $0$;
  \item да проверим верността на $i < n$;
  \item да проверим верността на $A[i] = v$;
  \item да върнем $i$ т.е. $1$.
\end{itemize}

Нека сега да помислим какво ще стане в най-лошия случай (обикновено от тези ще се интересуваме) -- $v$ не участва в $A[1 \dots n]$.
Тогава $n$ пъти ще изпълним следните $3$ операции:
\begin{itemize}
  \item проверяваме верността на $i \leq n$;
  \item проверяваме верността на $A[i] = v$;
  \item увеличаваме $i$ с $1$.
\end{itemize}
Освен тези $3n$ операции, преди всичко трябва да инициализираме променливата $i$ със $1$, да се направи последната проверка на верността на $i \leq n$ (която ще ни изкара от цикъла), и да върнем $-1$.
Общо излизат $3n + 3$ операции.

Така виждаме, че в зависимост от входните данни, алгоритъмът приключва работа за поне $4$ стъпки и най-много $3n + 3$ стъпки.
Такъв алгоритъм ще казваме, че има сложност по време $O(n)$.
Разбира се, няма да е грешно и да кажем, че алгоритъмът има сложност по време $\Omega(1)$, но това не ни дава никаква информация, защото всеки алгоритъм има такава сложност.
Също така, понеже не използваме допълнителни променливи, алгоритъмът ни има константна сложност по памет или сложност по памет $\Theta(1)$.

По-общо казано, се интересуваме от асимптотиката на $T(n)$, където $T(n)$ е броят елементарни инструкции, които алгоритъмът извиква по време на своето изпълнение, при вход с размер $n$ в най-лошият случай.

Тук вход с големина $n$ може да означава различни неща.
Ако входът е някакъв масив или множество, то под размер ще разбираме броят на елементи.
Ако пък входът е число, то под размер можем да разбираме самата стойност на числото или дължината на двоичния запис.

\section{Предимствата и недостатъците на този вид анализ}

Най-голямото предимство на асимптотичния анализ, е неговата простота.
Вместо да влачим някакви константни множители и събираеми, имаме колкото се може по-проста формула, която да описва сложността на нашия алгоритъм.
Това дали един алгоритъм работи със две или три стъпки по-бързо/бавно не ни интересува особено много.
При много голям вход те ще работят практически еднакво.
В някакъв смисъл това ни помага да виждаме по-голямата картинка.
Един алгоритъм може да бъде по-бърз от друг, но от по-бърз алгоритъм до по-бърз алгоритъм има голяма разлика.

Нека вземем за пример следната таблица:
\begin{center}
  \begin{tabular}{|c|c|c|c|c|}
    \hline
    $n$       & $\lceil\log_2(n)\rceil$ & $n$       & $n^2$           & $2^n$                    \\
    \hline
    $1$       & $0$                     & $1$       & $1$             & $2$                      \\
    \hline
    $10$      & $4$                     & $10$      & $100$           & $1024$                   \\
    \hline
    $100$     & $7$                     & $100$     & $10000$         & число със $31$ цифри     \\
    \hline
    $10000$   & $13$                    & $10000$   & $100000000$     & число със $3011$ цифри   \\
    \hline
    $1000000$ & $20$                    & $1000000$ & $1000000000000$ & число със $301030$ цифри \\
    \hline
  \end{tabular}
\end{center}

Алгоритъм със сложността $n^2$ ще е по-бавен от алгоритъм със сложност $n$, обаче скока в бързината е много по-малък от този между $2^n$ и $n^2$.

Този подход обаче си има своите недостатъци.
Нека разгледаме два алгоритъма със сложности по време съответно $n$ и $2^{2^{2^{2^{1024}}}}$.
Ние ведната ще се втурнем да кажем, че първият алгоритъм е по-лош.
Той е с линейна сложност, а вторият алгоритъм има константна сложност.
Обаче преди вторият алгоритъм даде отговор, всички звезди ще умрат т.е. няма да доживем да чуем този отговор.
Разбира се, от някъде нататък, за много голями входни данни, първият алгоритъм наистина ще работи по-бавно, но ние никога няма да работим с толкова големи данни.
Тогава на практика, първият алгоритъм е по-добър, нищо че асимптотично се води по-лош.
Нас това няма да ни интересува в курса по ДАА.

\section{Сложност по време на някои алгоритми}

Нека видим сложността на алгоритъма за сортиране по метода на мехурчето:
\lstinputlisting{algorithms/bubble-sort.txt}

В най-лошия случай сложността $T(n)$ на функцията {\tt Sort} е следната:
\[
  T(n) = \sum\limits_{i = 1}^{n - 1} \sum\limits_{j = 1}^{n - i - 1} 1 = \sum\limits_{i = 1}^{n - 1} (n - i - 1) = (n - 2) + (n - 3) + \dots + 0 = \frac{(n - 2)(n - 1)}{2} \asymp n^2.
\]

По принцип $T(n)$ трябва да е сума от $4$, а не от $1$, но такъв константен брой операции, дори и приложени неконстантен брой пъти, не влияят на асимптотичното поведение.

Нека сега разгледаме следният алгоритъм за степенуване:
\lstinputlisting{algorithms/exp.txt}

Той се възползва от простата идея, че за да сметнем да кажем $3^8$, можем вместо $8$ пъти да умножаваме числото $3$, да представим $3^8$ като $3^4 \cdot 3^4$.
Тогава $3^4$ можем да сметнем веднъж, и да го умножим със себе си.
Пак можем да представим $3^4$ като $3^2 \cdot 3^2$ и да пресметнем $3^2$ само веднъж и да го умножим със себе си.
Така при по-голяма стойност на $y$ си спестяваме много работа.
С уговорката, че умножението е атомарна операция, сложността по време $T(n)$ ($n$ е стойността на $y$) на функцията {\tt Exp} е следната:
\[
  T(n) = \sum\limits_{\substack{i = n \\ i \leftarrow \frac{i}{2}}}^1 1 = \underbrace{1 + \dots + 1}_{\substack{\text{колкото пъти} \\\text{можем да} \\ \text{делим целочислено} \\ n \text{ на } 2 \text{ преди} \\ \text{да получим } 0}} = \underbrace{1 + \dots + 1}_{\text{около } \log(n) \text{ пъти}} \asymp \log(n).
\]

\newpage

\section{Задачи}

\begin{problem}
Да се определи сложността по време за функцията:
\lstinputlisting{algorithms/task1.txt}
\end{problem}

\begin{problem}
Да се определи сложността по време за функцията:
\lstinputlisting{algorithms/task2.txt}
\end{problem}

\begin{problem}
Да се определи сложността по време за функцията:
\lstinputlisting{algorithms/task3.txt}
\end{problem}

\end{document}