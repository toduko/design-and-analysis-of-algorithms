\documentclass{article}

\usepackage[utf8]{inputenc}
\usepackage[T2A]{fontenc}
\usepackage[english, bulgarian]{babel}
\usepackage{pgfplots}
\usepackage{amssymb}
\usepackage{hyperref, fancyhdr, lastpage, fancyvrb, tcolorbox, titlesec}
\usepackage{array, tabularx, colortbl}
\usepackage{tikz}
\usepackage{venndiagram}
\usepackage{amsthm, bm}
\usepackage{relsize}
\usepackage{amsmath,physics}
\usepackage{mathtools}
\usepackage{subcaption}
\usepackage{theoremref}
\usepackage{circuitikz}
\usepackage[a4paper, left=0.50in, right=0.50in, top=0.5in, bottom=1.0in]{geometry}
\usepackage{stmaryrd}
\usepackage[symbol]{footmisc}
\usepackage{minted}
\usepackage{enumitem}

\pgfplotsset{width=10cm,compat=1.9}

\newcommand{\N}{\mathbb{N}}
\newcommand{\R}{\mathbb{R}}
\newcommand{\F}{\mathcal{F}}

\ExplSyntaxOn
\NewDocumentCommand{\opair}{m}
{
  \langle\mspace{2mu}
  \clist_set:Nn \l_tmpa_clist { #1 }
  \clist_use:Nn \l_tmpa_clist {,\mspace{3mu plus 1mu minus 1mu}\allowbreak}
  \mspace{2mu}\rangle
}
\ExplSyntaxOff

\hypersetup{
  colorlinks=true,
  linktoc=all,
  linkcolor=blue
}

\theoremstyle{definition}
\newtheorem{definition}{Дефиниция}[section]
\newtheorem*{warning}{\textcolor{red}{Внимание}}
\theoremstyle{plain}
\newtheorem*{theorem}{Теорема}
\newtheorem*{invariant}{Инварианта}
\newtheorem{claim}[definition]{Твърдение}
\newtheorem{axiom}[definition]{Аксиома}
\newtheorem{lemma}[definition]{Лема}
\newtheorem{corollary}[definition]{Следствие}
\theoremstyle{remark}
\newtheorem*{remark}{Забележка}
\newtheorem{problem}{Задача}
\newtheorem*{solution}{Решение}
\theoremstyle{definition}

\setlength\parindent{0pt}

\title{Коректност на рекурсивни алгоритми}
\author{Тодор Дуков}
\date{}

\begin{document}
\maketitle

\section*{Примери за рекурсивни алгоритми}

Нека разгледаме следният алгоритъм:
\inputminted[linenos]{c++}{algorithms/maximum.cpp}

Очевидно при параметри масив $\mathtt{arr}$ с размер поне $\mathtt{n}$, функцията $\mathtt{maximum(arr, n)}$ връща $\mathtt{\max arr[0 \dots n - 1]}$.
Ще докажем това с индукция по $\mathtt{n}$:
\begin{itemize}
    \item за $\mathtt{n = 0}$ имаме, че:
          \[
              \mathtt{maximum(arr, n) = -\infty = \max [] = \max arr[0 \dots 0 - 1] \text{, където под } [] \text{ се има предвид празният масив}}
          \]
    \item за $\mathtt{n + 1}$ имаме, че:
          \[
              \mathtt{maximum(arr, n + 1) = \max(arr[n], maximum(arr, n)) \stackrel{\text{(ИП)}}{=} \max(arr[n], \max arr[0 \dots n - 1]) = \max arr[0 \dots n]}
          \]
\end{itemize}

Тук управляващият параметър на рекурсията $\mathtt{n}$ винаги намалява с $\mathtt{1}$, докато не стигне $\mathtt{0}$, където ще приключи алгоритъмът.
В по нататъчните разсъждения ще смятаме това за очевидно.

Сложността на алгоритъма се описва със рекурентното уравнение:
\[
    T(n) = T(n - 1) + 1 \text{ // базата няма да я пишем}
\]

Директно се вижда, че $T(n) = \sum\limits_{i = 0}^n 1 = n + 1 \asymp n$.

Да видим един малко по-сложен пример -- за бързо степенуване:
\inputminted[linenos]{c++}{algorithms/exp.cpp}

С пълна (от тук нататък няма да го пишем) индукция относно $\mathtt{power}$ ще покажем, че $\mathtt{exp(base, power) = base^{power}}$:
\begin{itemize}
    \item $\mathtt{exp(base, 0) = 1 = base^0}$ // тук се уговаряме, че $\mathtt{0^0 = 1}$
    \item $\mathtt{exp(base, 2n + 1) = base \cdot exp(base, n) \cdot exp(base, n) \stackrel{\text{(ИП)}}{=} base \cdot base^n \cdot base^n = base^{2n + 1}}$
    \item $\mathtt{exp(base, 2n + 2) = exp(base, n + 1) \cdot exp(base, n + 1) \stackrel{\text{(ИП)}}{=} base^{n + 1} \cdot base^{n + 1} = base^{2n + 2}}$
\end{itemize}

Сложността на този алгоритъм може да се опише със следното рекурентно уравнение:
\[
    T(n) = T(\frac{n}{2}) + 1
\]
От мастър-теоремата следва, че:
\[
    T(n) \asymp \log(n)
\]

\pagebreak

\section*{Трик за бързо пресмятане на членове на някои рекурентни редици}

Нека вземем за пример редицата на Фибоначи:
\begin{align*}
     & F(0) = 0                   \\
     & F(1) = 1                   \\
     & F(n + 2) = F(n + 1) + F(n)
\end{align*}
Човек може да забележи, че:
\[
    \begin{pmatrix}
        1 & 1 \\
        1 & 0
    \end{pmatrix}
    \cdot
    \begin{pmatrix}
        F(n + 1) \\
        F(n)
    \end{pmatrix}
    =
    \begin{pmatrix}
        F(n + 2) \\
        F(n + 1)
    \end{pmatrix}
\]
След това с индукция лесно се показва, че:
\[
    \begin{pmatrix}
        1 & 1 \\
        1 & 0
    \end{pmatrix}^n
    \cdot
    \begin{pmatrix}
        F(1) \\
        F(0)
    \end{pmatrix}
    =
    \begin{pmatrix}
        F(n + 1) \\
        F(n)
    \end{pmatrix}
\]

За да сметнем $F(n)$, можем да направим бързо степенуване на матрицата, аналогична на тази с числата:
\inputminted[linenos]{c++}{algorithms/fibonacci.cpp}

Коректността и сложността на алгоритъма оставяме на читателя (напълно аналогично е на предния алгоритъм).

\pagebreak

В общия случай ще имаме рекурентно уравнение от вида:
\[
    T(n + k + 1) = a_k T(n + k) + \dots + a_1 T(n + 1) + a_0 T(n) \text{, където } a_0, \dots, a_k, k \text{ са константи}
\]
Тогава имаме следната зависимост:
\[
    \begin{pmatrix}
        a_k    & a_{k - 1} & a_{k - 2} & \dots  & a_0    \\
        1      & 0         & 0         & \dots  & 0      \\
        0      & 1         & 0         & \dots  & 0      \\
        \vdots & \vdots    & \vdots    & \ddots & \vdots \\
        0      & \dots     & 0         & 1      & 0      \\
    \end{pmatrix}
    \cdot
    \begin{pmatrix}
        T(n + k)     \\
        T(n + k - 1) \\
        T(n + k - 2) \\
        \vdots       \\
        T(n)
    \end{pmatrix}
    =
    \begin{pmatrix}
        T(n + k + 1) \\
        T(n + k)     \\
        T(n + k - 1) \\
        \vdots       \\
        T(n + 1)
    \end{pmatrix}
\]
Отново с индукция лесно се показва, че:
\[
    \begin{pmatrix}
        a_k    & a_{k - 1} & a_{k - 2} & \dots  & a_0    \\
        1      & 0         & 0         & \dots  & 0      \\
        0      & 1         & 0         & \dots  & 0      \\
        \vdots & \vdots    & \vdots    & \ddots & \vdots \\
        0      & \dots     & 0         & 1      & 0      \\
    \end{pmatrix}^n
    \cdot
    \begin{pmatrix}
        T(k)     \\
        T(k - 1) \\
        T(k - 2) \\
        \vdots   \\
        T(0)
    \end{pmatrix}
    =
    \begin{pmatrix}
        T(n + k)     \\
        T(n + k - 1) \\
        T(n + k - 2) \\
        \vdots       \\
        T(n)
    \end{pmatrix}
\]

\section*{Задачи}

\begin{problem}
Да се напише алгоритъм $\mathtt{calculate(first\_members, step, k, n)}$, който приема два масива от цели числа $\mathtt{first\_members}$ и $\mathtt{step}$ с размер $\mathtt{k + 1}$, естествено число $\mathtt{n}$, и връща $\mathtt{n}$-тия член на следната рекурентна редица:
\begin{align*}
     & T(0) = \mathtt{first\_members[0]}                                                                                      \\
     & T(1) = \mathtt{first\_members[1]}                                                                                      \\
     & \phantom{0000000000} \vdots                                                                                            \\
     & T(k) = \mathtt{first\_members[k]}                                                                                      \\
     & T(n + k + 1) = \mathtt{step[k]} \cdot T(n + k) + \dots + \mathtt{step[1]} \cdot T(n + 1) + \mathtt{step[0]} \cdot T(n)
\end{align*}
След това да се докаже неговата коректност, и да се изследва сложността му по време.
\end{problem}

\begin{problem}
Даден е следният алгоритъм:
\inputminted[linenos]{c++}{algorithms/stooge_sort.cpp}
Да се докаже, че при подаден масив $\mathtt{arr}$ с размер $\mathtt{n}$, функцията $\mathtt{stooge\_sort(arr, 0, n - 1)}$ ще сортира $\mathtt{arr}$.
След това да се направи анализ на сложността по време.
\end{problem}

\end{document}
